\documentclass[letterpaper,12pt,twoside,]{pinp}

%% Some pieces required from the pandoc template
\providecommand{\tightlist}{%
  \setlength{\itemsep}{0pt}\setlength{\parskip}{0pt}}

% Use the lineno option to display guide line numbers if required.
% Note that the use of elements such as single-column equations
% may affect the guide line number alignment.

\usepackage[T1]{fontenc}
\usepackage[utf8]{inputenc}

% pinp change: the geometry package layout settings need to be set here, not in pinp.cls
\geometry{layoutsize={0.95588\paperwidth,0.98864\paperheight},%
  layouthoffset=0.02206\paperwidth, layoutvoffset=0.00568\paperheight}

\definecolor{pinpblue}{HTML}{185FAF}  % imagecolorpicker on blue for new R logo
\definecolor{pnasbluetext}{RGB}{101,0,0} %



\title{Logistic Regression in R}

\author[a]{EPIB607 - Inferential Statistics}

  \affil[a]{Fall 2020, McGill University}

\setcounter{secnumdepth}{5}

% Please give the surname of the lead author for the running footer
\leadauthor{Bhatnagar}

% Keywords are not mandatory, but authors are strongly encouraged to provide them. If provided, please include two to five keywords, separated by the pipe symbol, e.g:
 

\begin{abstract}

\end{abstract}

\dates{This version was compiled on \today} 

% initially we use doi so keep for backwards compatibility
% new name is doi_footer


\begin{document}

% Optional adjustment to line up main text (after abstract) of first page with line numbers, when using both lineno and twocolumn options.
% You should only change this length when you've finalised the article contents.
\verticaladjustment{-2pt}

\maketitle
\thispagestyle{firststyle}
\ifthenelse{\boolean{shortarticle}}{\ifthenelse{\boolean{singlecolumn}}{\abscontentformatted}{\abscontent}}{}

% If your first paragraph (i.e. with the \dropcap) contains a list environment (quote, quotation, theorem, definition, enumerate, itemize...), the line after the list may have some extra indentation. If this is the case, add \parshape=0 to the end of the list environment.


\hypertarget{fitting-a-logistic-regression-model-with-a-single-predictor}{%
\section{Fitting a Logistic Regression Model with a Single
Predictor}\label{fitting-a-logistic-regression-model-with-a-single-predictor}}

The \textbf{\texttt{glm()}} function is used to fit logistic regression
models. It has the following generic structure:

\begin{Shaded}
\begin{Highlighting}[]
\KeywordTok{glm}\NormalTok{(y }\OperatorTok{~}\StringTok{ }\NormalTok{x, data, }\DataTypeTok{family =} \KeywordTok{binomial}\NormalTok{(}\DataTypeTok{link =} \StringTok{"logit"}\NormalTok{))}
\end{Highlighting}
\end{Shaded}

where the first argument specifies the variables used in the model; in
this example, the model regresses a response variable \texttt{y} against
an explanatory variable \texttt{x}. The second argument is used only
when the dataframe name is not already specified in the first argument.
Running the function creates an \emph{object} (of class `\texttt{lm}'
and `\texttt{glm}') that contains several components, such as the model
coefficients. The model coefficients are directly displayed upon running
\texttt{glm()}, while other components can be accessed through either
the \texttt{\$} notation or specific functions like \texttt{summary()}.
The argument \texttt{family = binomial(link = "logit")} is specific to
logistic regression; the texttt\{glm()\} function is capable of running
families of general linear models that are not discussed in this course.

The following example shows fitting a linear model that predicts the
estimated log odds of death before discharge from resting heart rate,
using data from \texttt{icu}.

\begin{Shaded}
\begin{Highlighting}[]
\CommentTok{#load the data}
\KeywordTok{library}\NormalTok{(aplore3)}
\KeywordTok{data}\NormalTok{(}\StringTok{"icu"}\NormalTok{)}

\CommentTok{#fitting logistic model}
\KeywordTok{glm}\NormalTok{(sta }\OperatorTok{~}\StringTok{ }\NormalTok{hra, }\DataTypeTok{data =}\NormalTok{ icu, }\DataTypeTok{family =} \KeywordTok{binomial}\NormalTok{(}\DataTypeTok{link =} \StringTok{"logit"}\NormalTok{))}
\end{Highlighting}
\end{Shaded}

\begin{ShadedResult}
\begin{verbatim}
#  
#  Call:  glm(formula = sta ~ hra, family = binomial(link = "logit"), data = icu)
#  
#  Coefficients:
#  (Intercept)          hra  
#    -1.679129     0.002941  
#  
#  Degrees of Freedom: 199 Total (i.e. Null);  198 Residual
#  Null Deviance:       200.2 
#  Residual Deviance: 200   AIC: 204
\end{verbatim}
\end{ShadedResult}

To fit a linear model that predicts the estimated log odds of survival
to discharge from resting heart rate, it is necessary to relevel the
factor \texttt{sta} such that a \texttt{1} corresponds to individuals
who survived to discharge. This can be accomplished with
\textbf{\texttt{factor()}} and \textbf{\texttt{rev()}}. The
\texttt{rev()} function reverses elements. In the example below,
applying \texttt{rev)} to a vector \{1, 2, 3\} produces a vector \{3, 2,
1\}.

\begin{Shaded}
\begin{Highlighting}[]
\CommentTok{#check levels}
\KeywordTok{levels}\NormalTok{(icu}\OperatorTok{$}\NormalTok{sta)}
\end{Highlighting}
\end{Shaded}

\begin{ShadedResult}
\begin{verbatim}
#  [1] "Lived" "Died"
\end{verbatim}
\end{ShadedResult}

\begin{Shaded}
\begin{Highlighting}[]
\CommentTok{#relevel survival}
\NormalTok{icu}\OperatorTok{$}\NormalTok{sta =}\StringTok{ }\KeywordTok{factor}\NormalTok{(icu}\OperatorTok{$}\NormalTok{sta, }\DataTypeTok{levels =} \KeywordTok{rev}\NormalTok{(}\KeywordTok{levels}\NormalTok{(icu}\OperatorTok{$}\NormalTok{sta)))}

\CommentTok{#check levels}
\KeywordTok{levels}\NormalTok{(icu}\OperatorTok{$}\NormalTok{sta)}
\end{Highlighting}
\end{Shaded}

\begin{ShadedResult}
\begin{verbatim}
#  [1] "Died"  "Lived"
\end{verbatim}
\end{ShadedResult}

\begin{Shaded}
\begin{Highlighting}[]
\CommentTok{#example of using rev()}
\NormalTok{a =}\StringTok{ }\KeywordTok{c}\NormalTok{(}\DecValTok{1}\NormalTok{, }\DecValTok{2}\NormalTok{, }\DecValTok{3}\NormalTok{)}
\KeywordTok{rev}\NormalTok{(a)}
\end{Highlighting}
\end{Shaded}

\begin{ShadedResult}
\begin{verbatim}
#  [1] 3 2 1
\end{verbatim}
\end{ShadedResult}

The following example shows outputting the model summary, selectively
outputting model coefficients from the model fit, and extracting the
numeric value of a coefficient.

\begin{Shaded}
\begin{Highlighting}[]
\CommentTok{#name the model}
\NormalTok{model.hra =}\StringTok{ }\KeywordTok{glm}\NormalTok{(sta }\OperatorTok{~}\StringTok{ }\NormalTok{hra, }\DataTypeTok{data =}\NormalTok{ icu, }\DataTypeTok{family =} \KeywordTok{binomial}\NormalTok{(}\DataTypeTok{link =} \StringTok{"logit"}\NormalTok{))}

\CommentTok{#model summary}
\KeywordTok{summary}\NormalTok{(model.hra)}
\end{Highlighting}
\end{Shaded}

\begin{ShadedResult}
\begin{verbatim}
#  
#  Call:
#  glm(formula = sta ~ hra, family = binomial(link = "logit"), data = icu)
#  
#  Deviance Residuals: 
#      Min       1Q   Median       3Q      Max  
#  -1.8524   0.6339   0.6579   0.6784   0.7533  
#  
#  Coefficients:
#               Estimate Std. Error z value Pr(>|z|)  
#  (Intercept)  1.679129   0.679863   2.470   0.0135 *
#  hra         -0.002941   0.006552  -0.449   0.6535  
#  ---
#  Signif. codes:  0 '***' 0.001 '**' 0.01 '*' 0.05 '.' 0.1 ' ' 1
#  
#  (Dispersion parameter for binomial family taken to be 1)
#  
#      Null deviance: 200.16  on 199  degrees of freedom
#  Residual deviance: 199.96  on 198  degrees of freedom
#  AIC: 203.96
#  
#  Number of Fisher Scoring iterations: 4
\end{verbatim}
\end{ShadedResult}

\begin{Shaded}
\begin{Highlighting}[]
\CommentTok{#model summary of coefficients}
\KeywordTok{summary}\NormalTok{(model.hra)}\OperatorTok{$}\NormalTok{coef}
\end{Highlighting}
\end{Shaded}

\begin{ShadedResult}
\begin{verbatim}
#                  Estimate  Std. Error    z value   Pr(>|z|)
#  (Intercept)  1.679128937 0.679862734  2.4698058 0.01351864
#  hra         -0.002941381 0.006552235 -0.4489127 0.65349464
\end{verbatim}
\end{ShadedResult}

\begin{Shaded}
\begin{Highlighting}[]
\CommentTok{#extract value of slope coefficient}
\KeywordTok{coef}\NormalTok{(model.hra)[}\DecValTok{2}\NormalTok{]}
\end{Highlighting}
\end{Shaded}

\begin{ShadedResult}
\begin{verbatim}
#           hra 
#  -0.002941381
\end{verbatim}
\end{ShadedResult}

As in linear regression, the \texttt{predict()} function can be used to
evaluate the regression equation for specific values of a predictor
variable. The following example shows predicting the estimated log odds
of survival to discharge for an individual with resting heart rate of 98
bpm.

\begin{Shaded}
\begin{Highlighting}[]
\KeywordTok{predict}\NormalTok{(model.hra, }\DataTypeTok{newdata =} \KeywordTok{data.frame}\NormalTok{(}\DataTypeTok{hra =} \DecValTok{98}\NormalTok{))}
\end{Highlighting}
\end{Shaded}

\begin{ShadedResult}
\begin{verbatim}
#         1 
#  1.390874
\end{verbatim}
\end{ShadedResult}

\newpage

\hypertarget{multiple-logistic-regression}{%
\section{Multiple Logistic
Regression}\label{multiple-logistic-regression}}

\hypertarget{working-with-several-predictors}{%
\subsection{Working with Several
Predictors}\label{working-with-several-predictors}}

The \textbf{\texttt{glm()}} function is used to fit linear models. It
has the following generic structure:

\begin{Shaded}
\begin{Highlighting}[]
\KeywordTok{glm}\NormalTok{(y }\OperatorTok{~}\StringTok{ }\NormalTok{x1 }\OperatorTok{+}\StringTok{ }\NormalTok{x2, data, }\DataTypeTok{family =} \KeywordTok{binomial}\NormalTok{(}\DataTypeTok{link =} \StringTok{"logit"}\NormalTok{))}
\end{Highlighting}
\end{Shaded}

where the first argument specifies the variables used in the model; in
this example, the model regresses a response variable \texttt{y} against
two explanatory variables \texttt{x1} and \texttt{x2}. Additional
predictor variables can be added to the model formula with the
\texttt{+} symbol, and an interaction between two variables is specified
with the \texttt{*} symbol.

The following example shows fitting a linear model that predicts the
estimated log odds of survival to discharge from age and gender, and a
linear model that predicts the estimated log odds of survival to
discharge from age, gender, and their interaction.

\begin{Shaded}
\begin{Highlighting}[]
\CommentTok{#fitting model with age and gender}
\KeywordTok{glm}\NormalTok{(sta }\OperatorTok{~}\StringTok{ }\NormalTok{age }\OperatorTok{+}\StringTok{ }\NormalTok{gender, }\DataTypeTok{data =}\NormalTok{ icu, }\DataTypeTok{family =} \KeywordTok{binomial}\NormalTok{(}\DataTypeTok{link =} \StringTok{"logit"}\NormalTok{))}
\end{Highlighting}
\end{Shaded}

\begin{ShadedResult}
\begin{verbatim}
#  
#  Call:  glm(formula = sta ~ age + gender, family = binomial(link = "logit"), 
#      data = icu)
#  
#  Coefficients:
#   (Intercept)           age  genderFemale  
#       3.05669      -0.02758       0.01131  
#  
#  Degrees of Freedom: 199 Total (i.e. Null);  197 Residual
#  Null Deviance:       200.2 
#  Residual Deviance: 192.3     AIC: 198.3
\end{verbatim}
\end{ShadedResult}

\begin{Shaded}
\begin{Highlighting}[]
\CommentTok{#fitting model with age, gender, and an interaction term}
\KeywordTok{glm}\NormalTok{(sta }\OperatorTok{~}\StringTok{ }\NormalTok{age}\OperatorTok{*}\NormalTok{gender, }\DataTypeTok{data =}\NormalTok{ icu, }\DataTypeTok{family =} \KeywordTok{binomial}\NormalTok{(}\DataTypeTok{link =} \StringTok{"logit"}\NormalTok{))}
\end{Highlighting}
\end{Shaded}

\begin{ShadedResult}
\begin{verbatim}
#  
#  Call:  glm(formula = sta ~ age * gender, family = binomial(link = "logit"), 
#      data = icu)
#  
#  Coefficients:
#       (Intercept)               age      genderFemale  age:genderFemale  
#         3.0762954        -0.0279007        -0.0388512         0.0007774  
#  
#  Degrees of Freedom: 199 Total (i.e. Null);  196 Residual
#  Null Deviance:       200.2 
#  Residual Deviance: 192.3     AIC: 200.3
\end{verbatim}
\end{ShadedResult}

\hypertarget{calculating-aic}{%
\subsection{Calculating AIC}\label{calculating-aic}}

The AIC of a logistic model can be extracted from \texttt{summary()} or
computed via the \textbf{\texttt{AIC()}} function.

The following example shows how to output the AIC from the model
predicting estimated odds of survival to discharge from resting heart
rate.

\begin{Shaded}
\begin{Highlighting}[]
\CommentTok{#use summary()$aic}
\KeywordTok{summary}\NormalTok{(model.hra)}\OperatorTok{$}\NormalTok{aic}
\end{Highlighting}
\end{Shaded}

\begin{ShadedResult}
\begin{verbatim}
#  [1] 203.9604
\end{verbatim}
\end{ShadedResult}

\begin{Shaded}
\begin{Highlighting}[]
\CommentTok{#use AIC()}
\KeywordTok{AIC}\NormalTok{(model.hra)}
\end{Highlighting}
\end{Shaded}

\begin{ShadedResult}
\begin{verbatim}
#  [1] 203.9604
\end{verbatim}
\end{ShadedResult}

\hypertarget{collapsing-factor-levels}{%
\subsection{Collapsing Factor Levels}\label{collapsing-factor-levels}}

The \texttt{factor()} function can also be used to collapse levels of a
factor.

The following example shows the re-defining of the levels of
\texttt{loc}; the variable initially has three levels (\texttt{Nothing},
\texttt{Stupor}, and \texttt{Coma}). The levels \texttt{Stupor} and
\texttt{Coma} can be combined into a single level \texttt{Unconscious},
while the level \texttt{Nothing} is renamed \texttt{Conscious}.

\begin{Shaded}
\begin{Highlighting}[]
\CommentTok{#view levels of loc}
\KeywordTok{levels}\NormalTok{(icu}\OperatorTok{$}\NormalTok{loc)}
\end{Highlighting}
\end{Shaded}

\begin{ShadedResult}
\begin{verbatim}
#  [1] "Nothing" "Stupor"  "Coma"
\end{verbatim}
\end{ShadedResult}

\begin{Shaded}
\begin{Highlighting}[]
\CommentTok{#create the loc.binary variable}
\NormalTok{icu}\OperatorTok{$}\NormalTok{loc.binary =}\StringTok{ }\NormalTok{icu}\OperatorTok{$}\NormalTok{loc}

\CommentTok{#redefine the factor levels of loc.binary}
\KeywordTok{levels}\NormalTok{(icu}\OperatorTok{$}\NormalTok{loc.binary) =}\StringTok{ }\KeywordTok{list}\NormalTok{(}\StringTok{"Conscious"}\NormalTok{ =}\StringTok{ "Nothing"}\NormalTok{, }
                              \StringTok{"Unconscious"}\NormalTok{ =}\StringTok{ }\KeywordTok{c}\NormalTok{(}\StringTok{"Stupor"}\NormalTok{, }\StringTok{"Coma"}\NormalTok{))}

\CommentTok{#view levels of loc.binary}
\KeywordTok{levels}\NormalTok{(icu}\OperatorTok{$}\NormalTok{loc.binary)}
\end{Highlighting}
\end{Shaded}

\begin{ShadedResult}
\begin{verbatim}
#  [1] "Conscious"   "Unconscious"
\end{verbatim}
\end{ShadedResult}

\begin{Shaded}
\begin{Highlighting}[]
\CommentTok{#compare tables}
\KeywordTok{table}\NormalTok{(icu}\OperatorTok{$}\NormalTok{loc); }\KeywordTok{table}\NormalTok{(icu}\OperatorTok{$}\NormalTok{loc.binary)}
\end{Highlighting}
\end{Shaded}

\begin{ShadedResult}
\begin{verbatim}
#  
#  Nothing  Stupor    Coma 
#      185       5      10
\end{verbatim}
\end{ShadedResult}
\begin{ShadedResult}
\begin{verbatim}
#  
#    Conscious Unconscious 
#          185          15
\end{verbatim}
\end{ShadedResult}

%\showmatmethods


\bibliography{pinp}
\bibliographystyle{jss}



\end{document}

