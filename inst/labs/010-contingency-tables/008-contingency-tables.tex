\documentclass[letterpaper,12pt,twoside,]{pinp}

%% Some pieces required from the pandoc template
\providecommand{\tightlist}{%
  \setlength{\itemsep}{0pt}\setlength{\parskip}{0pt}}

% Use the lineno option to display guide line numbers if required.
% Note that the use of elements such as single-column equations
% may affect the guide line number alignment.

\usepackage[T1]{fontenc}
\usepackage[utf8]{inputenc}

% pinp change: the geometry package layout settings need to be set here, not in pinp.cls
\geometry{layoutsize={0.95588\paperwidth,0.98864\paperheight},%
  layouthoffset=0.02206\paperwidth, layoutvoffset=0.00568\paperheight}

\definecolor{pinpblue}{HTML}{185FAF}  % imagecolorpicker on blue for new R logo
\definecolor{pnasbluetext}{RGB}{101,0,0} %



\title{Lab 008 - Contingency Tables and Difference of two Proportions}

\author[a]{EPIB607 - Inferential Statistics}

  \affil[a]{Fall 2020, McGill University}

\setcounter{secnumdepth}{5}

% Please give the surname of the lead author for the running footer
\leadauthor{Bhatnagar}

% Keywords are not mandatory, but authors are strongly encouraged to provide them. If provided, please include two to five keywords, separated by the pipe symbol, e.g:
 

\begin{abstract}

\end{abstract}

\dates{This version was compiled on \today} 

% initially we use doi so keep for backwards compatibility
% new name is doi_footer


\begin{document}

% Optional adjustment to line up main text (after abstract) of first page with line numbers, when using both lineno and twocolumn options.
% You should only change this length when you've finalised the article contents.
\verticaladjustment{-2pt}

\maketitle
\thispagestyle{firststyle}
\ifthenelse{\boolean{shortarticle}}{\ifthenelse{\boolean{singlecolumn}}{\abscontentformatted}{\abscontent}}{}

% If your first paragraph (i.e. with the \dropcap) contains a list environment (quote, quotation, theorem, definition, enumerate, itemize...), the line after the list may have some extra indentation. If this is the case, add \parshape=0 to the end of the list environment.


\tableofcontents

In the setting where a binary outcome is recorded for a single group of
participants, inference about the binomial probability of success
provides information about a population proportion \(p\). Just as
inference can be done for the difference of two population means,
inference can also be done in the setting of comparing two population
proportions \(p_1\) and \(p_2\). This lab then generalizes inference for
binomial proportions to the setting of two-way contingency tables.
Hypothesis testing in a two-way table assesses whether the two variables
of interest are associated; this approach can be applied to settings
with two or more groups and for responses that have two or more
categories. Measures of association in two-by-two tables are also
discussed.

\hypertarget{difference-of-two-proportions}{%
\section{Difference of two
proportions}\label{difference-of-two-proportions}}

\textbf{Recap: Inference for a single proportion}

Suppose that \(X\) is binomial with parameters \(n\) (total number of
trials) and \(p\), the population parameter of success, where \(x\)
represents the number of successes. Inference about \(p\) is based on
the sample proportion \(\hat{p}\), where \(\hat{p} = x/n\); \(\hat{p}\)
is the point estimate of \(p\).

Inference for \(p\) can be made using the normal approximation to the
binomial, or directly using the binomial distribution.

\begin{itemize}
\item
  \emph{Inference with the normal approximation}

  \begin{itemize}
  \item
    The sampling distribution of \(\hat{p}\) is approximately normal
    when 1) the sample observations are independent, 2) \(np \geq 10\),
    \(n(1 - p) \geq 10\).\footnote{The second condition is commonly referred to as the \textbf{success-failure condition}, since it can be effectively restated as the number of successes is greater than 10 and the number of failures is greater than 10.}
    Under these conditions, the sampling distribution of \(\hat{p}\) is
    approximately normally distributed with mean \(p\) and standard
    deviation \(\sqrt{\frac{p(1-p)}{n}}\). For confidence intervals,
    substitute \(\hat{p}\) for \(p\); for hypothesis testing, substitute
    \(p_0\) for \(p\).
  \item
    The approximate two-sided 95\% confidence interval for \(p\) is
    given by \[\hat{p} \pm 1.96 \sqrt{\dfrac{\hat{p}(1 - \hat{p})}{n}}\]
  \item
    The test statistic \(z\) for the null hypothesis \(H_0: p = p_0\)
    based on a sample size of \(n\) is
    \[z = \dfrac{\hat{p} - p_0}{\sqrt{\dfrac{(p_0)(1-p_0)}{n}}} \]
  \end{itemize}
\item
  \emph{Inference with exact methods}

  \begin{itemize}
  \item
    Confidence intervals and \(p\)-values based on the binomial
    distribution are best calculated via \textsf{R}.
  \item
    The logic behind calculating a \(p\)-value from the binomial
    distribution: Let \(X\) be a binomial random variable with
    parameters \(n\) and \(p_0\), where \(\hat{p} = x/n\) and \(x\) is
    the observed number of events. For a test of \(H_0: p = p_0\) versus
    \(H_A: p \neq p_0\), the \(p\)-value equals
    \(2 \times P(X \geq x)\).
  \end{itemize}
\end{itemize}

\textbf{Inference for the difference of two proportions}

The normal model can be applied to \(\hat{p}_1 - \hat{p}_2\) if the
sampling distribution for each sample proportion is nearly normal, and
if the samples are independent random samples from the relevant
populations and independent of each other.

Each sample proportion approximately follows a normal model when
\(n_1p_1\), \(n_1(1 - p_1)\), \(n_2p_2\), and \(n_2(1-p_2)\) are all
\(\geq 10\). To check success-failure in the context of a confidence
interval, use \(\hat{p}_1 and \hat{p}_2\).

The standard error of the difference in sample proportions is
\[\sqrt{\dfrac{p_1(1-p_1)}{n_1} + \dfrac{p_2(1-p_2)}{n_2}}. \]

For hypothesis testing, an estimate of \(p\) is used to compute the
standard error of \(\hat{p}_1 - \hat{p}_2\): \(\hat{p}\), the weighted
average of the sample proportions \(\hat{p}_1\) and \(\hat{p}_2\),
\[\hat{p} = \dfrac{n_1\hat{p}_1 + n_2\hat{p}_2}{n_1 + n_2} = \dfrac{x_1 + x_2}{n_1 + n_2}. \]

To check success-failure in the context of hypothesis testing, check
that \(\hat{p}n_1\) and \(\hat{p}n_2\) are both \(\geq 10\).

\begin{enumerate}
\def\labelenumi{\arabic{enumi}.}
\item
  The use of screening mammograms for breast cancer has been
  controversial for decades because the overall benefit on breast cancer
  mortality is uncertain. A 30-year study to investigate the
  effectiveness of mammograms versus a standard non-mammogram breast
  cancer exam was conducted in Canada with 89,835 female
  participants.\footnote{Miller AB. 2014. \emph{Twenty five year follow-up for breast cancer incidence and mortality of the Canadian National Breast Screening Study: randomised screening trial}. \textit{BMJ} 348 (2014): g366. }
  Each woman was randomized to receive either annual mammograms or
  standard physical exams for breast cancer over a 5-year screening
  period.

  By the end of the 25 year follow-up period, 1,005 women died from
  breast cancer. The results are summarized in the following
  table.\footnote{During the 25 years following the screening period, each woman was screened for breast cancer according to the standard of care at her health care center. }

  \begin{table}[h]
   \centering
   \begin{tabular}{rrcc}
       & \multicolumn{3}{c}{Death from breast cancer?} \\
       \cline{2-4}
       & \ \hspace{3mm}\ & Yes & No \\
       \hline
       Mammogram Group && 500 & 44,425 \\
       Control Group && 505 & 44,405 \\
       \hline
   \end{tabular}
  \end{table}

  \begin{enumerate}
  \def\labelenumii{\alph{enumii})}
  \item
    Calculate \(\hat{p}_1\) and \(\hat{p}_2\), the two sample
    proportions of interest.
  \item
    Analyze the results; do the data suggest that annual mammography
    results in a reduction in breast cancer mortality relative to
    standard exams? Be sure to check the assumptions for using the
    normal approximation.
  \item
    Calculate and interpret a 95\% confidence interval for the
    difference in proportions of deaths from breast cancer. Be sure to
    check the assumptions for using the normal approximation.
  \end{enumerate}
\item
  Remdesivir is an antiviral drug previously tested in animal models
  infected with coronaviruses like SARS and MERS. As of May 2020,
  remdesivir had temporary approval from the FDA for use in severely ill
  COVID-19 patients and was the subject of numerous ongoing studies.

  A randomized controlled trial conducted in China enrolled 236 patients
  with severe COVID-19; 158 were assigned to receive remdesivir and 78
  to receive a placebo. In the remdesivir group, 103 patients showed
  clinical improvement; in the placebo group, 45 patients showed
  clinical
  improvement.\footnote{Wang, Y, et al. Remdesivir in adults with severe COVID-19: a randomised, double-blind, placebo-controlled, multi-centre trial. \textit{Lancet} 395(10236). 16 May 2020.}

  \begin{enumerate}
  \def\labelenumii{\alph{enumii})}
  \item
    Calculate \(\hat{p}_1\) and \(\hat{p}_2\), the two sample
    proportions of interest.
  \item
    Conduct a formal comparison of the clinical improvement rates and
    summarize your findings. Be sure to check the assumptions for using
    the normal approximation.
  \item
    Report and interpret an appropriate interval estimate. Be sure to
    check the assumptions for using the normal approximation.
  \end{enumerate}
\end{enumerate}

\newpage

\hypertarget{contingency-tables}{%
\section{Contingency Tables}\label{contingency-tables}}

\textbf{The \(\chi^2\) test of independence}

In the \(\chi^2\) test of independence, the observed number of cell
counts are compared to the number of \textbf{expected} cell counts,
where the expected counts are calculated under the null hypothesis.

\begin{itemize}
\item
  \(H_0\): the row and column variables are not associated
\item
  \(H_A\): the row and column variables are associated
\end{itemize}

The expected count for the \(i^{th}\) row and \(j^{th}\) column is
\[E_{i, j} = \dfrac{(\text{row $i$ total}) \times (\text{column $j$ total}) }{n}, \]
where \(n\) is the total number of observations.

Assumptions for the \(\chi^2\) test:

\begin{itemize}
\item
  \emph{Independence}. Each case that contributes a count to the table
  must be independent of all other cases in the table.
\item
  \emph{Sample size}. Each expected cell count must be greater than or
  equal to 10. For tables larger than \(2 \times 2\), it is appropriate
  to use the test if no more than 1/5 of the expected counts are less
  than 5, and all expected counts are greater than 1.
\end{itemize}

The \textbf{\(\chi^2\) test statistic} is calculated as
\[\chi^2 = \sum_{i = 1}^r \sum_{j = 1}^c \dfrac{(O_{i, j} - E_{i, j})^2}{E_{i, j}}, \]
and is approximately distributed \(\chi^2\) with degrees of freedom
\((r - 1)(c - 1)\), where \(r\) is the number of rows and \(c\) is the
number of columns. \(O_{i, j}\) represents the observed count in row
\(i\), column \(j\).

For each cell in a table, the \textbf{residual} equals
\[\dfrac{O_{i, j} - E_{i, j}}{\sqrt{E_{i,j}}}. \] Residuals with a large
magnitude contribute the most to the \(\chi^2\) statistic. If a residual
is positive, the observed value is greater than the expected value; if a
residual is negative, the observed value is less than the expected.

\vspace{0.5cm}

\textbf{Fisher's exact test}

When the expected counts in a two-way table are less than 10, Fisher's
exact test is used to compute a \(p\)-value without relying on the
normal approximation. In this course, only the logic behind Fisher's
exact test for a \(2 \times 2\) table is discussed. In the
\(2 \times 2\) table case, the hypotheses for Fisher's exact test can be
expressed in the same way as for a two-sample test of proportions; the
null hypothesis is \(H_0: p_1 = p_2\).

The \(p\)-value is the probability of observing results as or more
extreme than those observed under the assumption that the null
hypothesis is true.

\begin{itemize}
\item
  Thus, the \(p\)-value is calculated by adding together the individual
  conditional probabilities of obtaining each table that is as or more
  extreme than the one observed, under the null hypothesis and given
  that the marginal totals are considered fixed.
\item
  When the marginal totals are held constant, the value of any one cell
  in the table determines the rest of entries. When marginal totals are
  considered fixed, each table represents a unique set of results.
\item
  Extreme tables are those which contradict \(H_0: p_1 = p_2\).
\item
  A two-sided \(p\)-value can be calculated by doubling the smaller of
  the possible one-sided \(p\)-values; this method is typically used
  when calculating \(p\)-values by hand. Another common method is to
  classify ``more extreme'' tables as all tables with probabilities less
  than that of the observed table, in both directions; the \(p\)-value
  is the sum of probabilities for the qualifying table.
\end{itemize}

The probability of a particular table (i.e., set of results) can be
calculated with the \textbf{hypergeometric distribution}.

Let \(X\) represent the number of successes in a series of repeated
Bernoulli trials, where sampling is done without replacement. Suppose
that in the population of size \(N\), there are \(m\) total successes.
What is the probability of observing exactly \(k\) successes when
drawing a sample of size \(n\)?

For example, imagine an urn with \(m\) white balls and \(N - m\) black
balls (thus, there are \(N\) total balls). Draw \(n\) balls without
replacement (i.e., a sample of \(n\) balls). What is the probability of
observing \(k\) white balls in the sample?

The possible results of a sample can be organized in a \(2 \times 2\)
table:

\begin{table*}[h!]
\begin{center}
\begin{tabular}{l|cc|c} 
   & \textbf{White Ball} & \textbf{Black Ball} & \textbf{Total}\\ \hline
  \textbf{Sampled} & $k$ & \textcolor{gray}{$n - k$}  & $n$  \\
  \textbf{Not Sampled} & \textcolor{gray}{$m - k$} & \textcolor{gray}{$N - n - (m - k)$} & \textcolor{gray}{$N - n$} \\ \hline
  \textbf{Total} & $m$ & $N - m$ & $N$  \\ 
\end{tabular}\\
\end{center}
\end{table*}

The probability of observing exactly \(k\) successses in a sample of
size \(n\) (i.e., \(n\) dependent trials) is given by
\[P(X = k) = \dfrac{{m \choose k} {N - m \choose n-k}}{{N \choose n}}. \]

Hypergeometric probabilities are calculated in \textsf{R} with the use
of \texttt{dhyper()} and \texttt{phyper()}. The following code shows how
to calculate \(P(X = 5)\), \(P(X \leq 5)\), and \(P(X > 5)\) for
\(X \sim \text{HGeom}(10, 15, 8)\), where \(m = 10\), \(N - m = 15\),
and \(n = 8\).

\begin{Shaded}
\begin{Highlighting}[]
\CommentTok{#probability X equals 5}
\KeywordTok{dhyper}\NormalTok{(}\DecValTok{5}\NormalTok{, }\DecValTok{10}\NormalTok{, }\DecValTok{15}\NormalTok{, }\DecValTok{8}\NormalTok{)}
\end{Highlighting}
\end{Shaded}

\begin{ShadedResult}
\begin{verbatim}
#  [1] 0.1060121
\end{verbatim}
\end{ShadedResult}

\begin{Shaded}
\begin{Highlighting}[]
\CommentTok{#probability X is less than or equal to 5}
\KeywordTok{phyper}\NormalTok{(}\DecValTok{5}\NormalTok{, }\DecValTok{10}\NormalTok{, }\DecValTok{15}\NormalTok{, }\DecValTok{8}\NormalTok{)}
\end{Highlighting}
\end{Shaded}

\begin{ShadedResult}
\begin{verbatim}
#  [1] 0.9779072
\end{verbatim}
\end{ShadedResult}

\begin{Shaded}
\begin{Highlighting}[]
\CommentTok{#probability X is greater than 5}
\KeywordTok{phyper}\NormalTok{(}\DecValTok{5}\NormalTok{, }\DecValTok{10}\NormalTok{, }\DecValTok{15}\NormalTok{, }\DecValTok{8}\NormalTok{, }\DataTypeTok{lower.tail =} \OtherTok{FALSE}\NormalTok{)}
\end{Highlighting}
\end{Shaded}

\begin{ShadedResult}
\begin{verbatim}
#  [1] 0.02209278
\end{verbatim}
\end{ShadedResult}

\vspace{0.5cm}

\textbf{Measures of association in two-by-two tables}

Chapter 1 introduced the \textbf{relative risk (RR)}, a measure of the
risk of a certain event occurring in one group relative to the risk of
the event occurring in another group, as a numerical summary for
two-by-two (\(2 \times 2\)) tables. The relative risk can also be
thought of as a measure of association.

Consider the following hypothetical two-by-two table. The relative risk
of Outcome A can be calculated by using either Group 1 or Group 2 as the
reference group:

\begin{table}[h!]
    \centering
    \begin{tabular}{r|rrr}
        \hline
        & Outcome A & Outcome B & Sum\\ 
        \hline
        Group 1 & $a$ & $b$ & $a + b$ \\ 
        Group 2 & $c$ & $d$ & $c + d$ \\
        Sum & $a + c$ & $b + d$ & $a + b + c + d = n$ \\
        \hline
    \end{tabular}   
    \caption{A hypothetical two-by-two table of outcome by group.}
\end{table}

\[RR_{\text{A, comparing Group 1 to Group 2}} = \dfrac{a/(a + b)}{c/(c+d)} \]
\[RR_{\text{A, comparing Group 2 to Group 1}} = \dfrac{c/(c + d)}{a/(a+b)} \]

The relative risk is only valid for tables where the proportions
\(a/(a + b)\) and \(c/(c + d)\) represent the incidence of Outcome A
within the populations from which Groups 1 and 2 are sampled.

The \textbf{odds ratio (OR)} is a measure of association that remains
applicable even when it is not possible to estimate incidence of an
outcome from the sample data. The \textbf{odds} of Outcome A in Group 1
are \(a/b\), while the odds of Outcome A in Group 2 are \(c/d\).

\[OR_{\text{A, comparing Group 1 to Group 2}} = \dfrac{a/b}{c/d} = \dfrac{ad}{bc} \]
\[OR_{\text{A, comparing Group 2 to Group 1}} = \dfrac{c/d}{a/b} = \dfrac{bc}{ad} \]

\vspace{0.5cm}

\hypertarget{the-chi2-test-of-independence}{%
\section{\texorpdfstring{The \(\chi^2\) test of
independence}{The \textbackslash chi\^{}2 test of independence}}\label{the-chi2-test-of-independence}}

\begin{enumerate}
\def\labelenumi{\arabic{enumi}.}
\setcounter{enumi}{2}
\item
  In resource-limited settings, single-dose nevirapine (NVP) is given to
  an HIV-positive woman during birth to prevent mother-to-child
  transmission of the virus. Exposure of the infant to NVP may foster
  the growth of more virulent strains of the virus in the child.

  If a child is HIV-positive, should they be treated with NVP or a more
  expensive drug, lopinavir (LPV)? In this setting, success means
  preventing a growth of the virus in the child (i.e., preventing
  virologic failure). The following table contains data from a 2012
  study conducted in six African countries and
  India.\footnote{A. Violari, et al. "Nevirapine versus ritonavir-boosted lopinavir for HIV-infected children." \textit{NEJM} 366: 2380-2389.}

  \begin{center}
  \begin{tabular}{l|cc|c} 
  & \textbf{NVP} & \textbf{LPV} & \textbf{Total}\\ \hline
    \textbf{Virologic Failure} & 60 & 27 & 87  \\
    \textbf{Stable Disease} & 87 & 113 & 200 \\ \hline
    \textbf{Total} & 147 & 140 & 287  \\ 
  \end{tabular}\\
  \end{center}

  \vspace{0.5cm}

  \begin{enumerate}
  \def\labelenumii{\alph{enumii})}
  \item
    State the null and alternative hypotheses.
  \item
    Calculate the expected cell counts.
  \item
    Check the assumptions for using the \(\chi^2\) test.
  \item
    Calculate the \(\chi^2\) test statistic.
  \item
    Calculate the \(p\)-value for the test statistic using
    \texttt{pchisq()}. The \(p\)-value represents the probability of
    observing a result as or more extreme than the sample data.
  \item
    Confirm the results from parts c) and d) using
    \texttt{chisq.test()}. Note that the value of the test statistic
    will be slightly different because \textsf{R} is applying a
    `continuity correction'.
  \item
    Summarize the conclusions; be sure to include which drug is
    recommended for treatment, based on the data.
  \item
    Repeat the analysis using inference for the difference of two
    proportions and confirm that the results are the same.
  \end{enumerate}
\item
  In the PREVEND study introduced in Chapter 6, researchers measured
  various features of study participants, including data on statin use
  and highest level of education attained. From the data in
  \texttt{prevend.samp}, is there evidence of an association between
  statin use and educational level? Summarize the results.
\end{enumerate}

\hypertarget{fishers-exact-test}{%
\section{Fisher's exact test}\label{fishers-exact-test}}

\begin{enumerate}
\def\labelenumi{\arabic{enumi}.}
\setcounter{enumi}{4}
\item
  \textit{Clostridium difficile} is a bacterium that causes inflammation
  of the colon. Antibiotic treatment is typically not effective,
  particularly for patients who experience multiple recurrences of
  infection. Infusion of feces from healthy donors has been reported as
  an effective treatment for recurrent infection. A randomized trial was
  conducted to compare the efficacy of donor-feces infusion versus
  vancomycin, the antibiotic typically prescribed to treat
  \textit{C. difficile }infection. The results of the trial are shown in
  the following table.

  \begin{table}[h]
   \centering
   \begin{tabular}{rrr|r}
       \hline
       & Cured & Uncured & Sum \\ 
       \hline
       Fecal Infusion & 13 & 3 & 16 \\ 
       Vancomycin & 4 & 9 & 13 \\ 
       \hline
       Sum & 17 & 12 & 29 \\ 
       \hline
   \end{tabular}
   \end{table}

  \begin{enumerate}
  \def\labelenumii{\alph{enumii})}
  \item
    Can a \(\chi^2\) test be used to analyze these results?
  \item
    Researchers are interested in understanding whether fecal infusion
    is a more effective treatment than vancomycin. Write the null
    hypothesis and appropriate one-sided alternative hypothesis.
  \item
    Under the assumption that the marginal totals are fixed, enumerate
    all possible sets of results that are more extreme than what was
    observed, in the same direction.
  \item
    Calculate the probability of the observed results.
  \item
    Calculate the probability of each set of results enumerated in part
    c).
  \item
    Based on the answers in parts d) and e), compute the one-sided
    \(p\)-value and interpret the results.
  \item
    Use \texttt{fisher.test( )} to confirm the calculations in part f)
    and to calculate the two-sided \(p\)-value.
  \end{enumerate}
\item
  Psychologists conducted an experiment to investigate the effect of
  anxiety on a person's desire to be alone or in the company of others
  (Schacter 1959; Lehmann 1975). A group of 30 individuals were randomly
  assigned into two groups; one group was designated the ``high
  anxiety'' group and the other the ``low anxiety'' group. Those in the
  high-anxiety group were told that in the ``upcoming experiment'', they
  would be subjected to painful electric shocks, while those in the
  low-anxiety group were told that the shocks would be mild and
  painless.\footnote{Individuals were not actually subjected to electric shocks of any kind}.
  All individuals were informed that there would be a 10 minute wait
  before the experiment began, and that they could choose whether to
  wait alone or with other participants.

  The following table summarizes the results:

  \begin{table}[h]
   \centering
   \begin{tabular}{rrr|r}
       \hline
       & Wait Together & Wait Alone & Sum \\ 
       \hline
       High-Anxiety & 12 & 5 & 17 \\ 
       Low-Anxiety & 4 & 9 & 13 \\ 
       \hline
       Sum & 16 & 14 & 30 \\ 
       \hline
   \end{tabular}
   \end{table}

  \begin{enumerate}
  \def\labelenumii{\alph{enumii})}
  \item
    Under the null hypothesis of no association, what are the expected
    cell counts?
  \item
    Under the assumption that the marginal totals are fixed and the null
    hypothesis is true, what is the probability of the observed set of
    results?
  \item
    Enumerate the tables that are more extreme than what was observed,
    in the same direction.
  \item
    Conduct a formal test of association for the results and summarize
    your findings. Let \(\alpha = 0.05\).
  \end{enumerate}
\end{enumerate}

\hypertarget{measures-of-association-in-two-by-two-tables}{%
\section{Measures of association in two-by-two
tables}\label{measures-of-association-in-two-by-two-tables}}

\begin{enumerate}
\def\labelenumi{\arabic{enumi}.}
\setcounter{enumi}{6}
\item
  Suppose a study is conducted to assess the association between smoking
  and cardiovascular disease (CVD). Researchers recruited a group of 231
  study participants then categorized them according to smoking and
  disease status: 111 are smokers, while 40 smokers and 32 non-smokers
  have CVD. Calculate and interpret the relative risk of CVD.
\item
  Suppose another study is conducted to assess the association between
  smoking and CVD, but researchers use a different design: 90
  individuals with CVD and 110 individuals without CVD are recruited. 40
  of the individuals with CVD are smokers, and 80 of the individuals
  without CVD are non-smokers.

  \begin{enumerate}
  \def\labelenumii{\alph{enumii})}
  \item
    Is relative risk an appropriate measure of association for these
    data? Explain your answer.
  \item
    Calculate the odds of CVD among smokers and the odds of CVD among
    non-smokers.
  \item
    Calculate and interpret the odds ratio of CVD, comparing smokers to
    non-smokers.
  \item
    What would an odds ratio of CVD (comparing smokers to non-smokers)
    equal to 1 represent, in terms of the association between smoking
    and CVD? What would an odds ratio of CVD less than 1 represent?
  \end{enumerate}
\end{enumerate}

%\showmatmethods


\bibliography{pinp}
\bibliographystyle{jss}



\end{document}

