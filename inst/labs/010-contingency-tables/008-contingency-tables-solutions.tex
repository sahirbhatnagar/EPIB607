\documentclass[letterpaper,12pt,twoside,]{pinp}

%% Some pieces required from the pandoc template
\providecommand{\tightlist}{%
  \setlength{\itemsep}{0pt}\setlength{\parskip}{0pt}}

% Use the lineno option to display guide line numbers if required.
% Note that the use of elements such as single-column equations
% may affect the guide line number alignment.

\usepackage[T1]{fontenc}
\usepackage[utf8]{inputenc}

% pinp change: the geometry package layout settings need to be set here, not in pinp.cls
\geometry{layoutsize={0.95588\paperwidth,0.98864\paperheight},%
  layouthoffset=0.02206\paperwidth, layoutvoffset=0.00568\paperheight}

\definecolor{pinpblue}{HTML}{185FAF}  % imagecolorpicker on blue for new R logo
\definecolor{pnasbluetext}{RGB}{101,0,0} %



\title{Lab 008 - Contingency Tables and Difference of two Proportions Solutions}

\author[a]{EPIB607 - Inferential Statistics}

  \affil[a]{Fall 2020, McGill University}

\setcounter{secnumdepth}{5}

% Please give the surname of the lead author for the running footer
\leadauthor{Bhatnagar}

% Keywords are not mandatory, but authors are strongly encouraged to provide them. If provided, please include two to five keywords, separated by the pipe symbol, e.g:
 

\begin{abstract}

\end{abstract}

\dates{This version was compiled on \today} 

% initially we use doi so keep for backwards compatibility
% new name is doi_footer


\begin{document}

% Optional adjustment to line up main text (after abstract) of first page with line numbers, when using both lineno and twocolumn options.
% You should only change this length when you've finalised the article contents.
\verticaladjustment{-2pt}

\maketitle
\thispagestyle{firststyle}
\ifthenelse{\boolean{shortarticle}}{\ifthenelse{\boolean{singlecolumn}}{\abscontentformatted}{\abscontent}}{}

% If your first paragraph (i.e. with the \dropcap) contains a list environment (quote, quotation, theorem, definition, enumerate, itemize...), the line after the list may have some extra indentation. If this is the case, add \parshape=0 to the end of the list environment.


\hypertarget{inference-for-the-difference-of-two-proportions}{%
\section{Inference for the difference of two
proportions}\label{inference-for-the-difference-of-two-proportions}}

\begin{enumerate}
\def\labelenumi{\arabic{enumi}.}
\item
  The use of screening mammograms for breast cancer has been
  controversial for decades because the overall benefit on breast cancer
  mortality is uncertain. A 30-year study to investigate the
  effectiveness of mammograms versus a standard non-mammogram breast
  cancer exam was conducted in Canada with 89,835 female
  participants.\footnote{Miller AB. 2014. \emph{Twenty five year follow-up for breast cancer incidence and mortality of the Canadian National Breast Screening Study: randomised screening trial}. \textit{BMJ} 348 (2014): g366. }
  Each woman was randomized to receive either annual mammograms or
  standard physical exams for breast cancer over a 5-year screening
  period.

  By the end of the 25 year follow-up period, 1,005 women died from
  breast cancer. The results are summarized in the following
  table.\footnote{During the 25 years following the screening period, each woman was screened for breast cancer according to the standard of care at her health care center. }

  \begin{table}[h]
   \centering
   \begin{tabular}{rrcc}
       & \multicolumn{3}{c}{Death from breast cancer?} \\
       \cline{2-4}
       & \ \hspace{3mm}\ & Yes & No \\
       \hline
       Mammogram Group && 500 & 44,425 \\
       Control Group && 505 & 44,405 \\
       \hline
   \end{tabular}
  \end{table}

  \begin{enumerate}
  \def\labelenumii{\alph{enumii})}
  \item
    Calculate \(\hat{p}_1\) and \(\hat{p}_2\), the two sample
    proportions of interest.

    \textcolor{blue}{The two sample proportions of interest are the proportion of breast cancer deaths in the mammogram group and the proportion of breast cancer deaths in the control group: $\hat{p}_M = 500/(500 + 44425) = 0.0111$, $\hat{p}_C = 505/(505 + 44405) = 0.0112$.}
  \end{enumerate}

\begin{Shaded}
\begin{Highlighting}[]
\CommentTok{#use r as a calculator}
\NormalTok{x =}\StringTok{ }\KeywordTok{c}\NormalTok{(}\DecValTok{500}\NormalTok{, }\DecValTok{505}\NormalTok{)}
\NormalTok{n =}\StringTok{ }\KeywordTok{c}\NormalTok{(}\DecValTok{500} \OperatorTok{+}\StringTok{ }\DecValTok{44425}\NormalTok{, }\DecValTok{505} \OperatorTok{+}\StringTok{ }\DecValTok{44405}\NormalTok{)}
\NormalTok{p.hat.vector =}\StringTok{ }\NormalTok{x}\OperatorTok{/}\NormalTok{n}
\NormalTok{p.hat.vector}
\end{Highlighting}
\end{Shaded}

  \begin{ShadedResult}
   \begin{verbatim}
   #  [1] 0.01112966 0.01124471
   \end{verbatim}
   \end{ShadedResult}

  \begin{enumerate}
  \def\labelenumii{\alph{enumii})}
  \setcounter{enumii}{1}
  \item
    Analyze the results; do the data suggest that annual mammography
    results in a reduction in breast cancer mortality relative to
    standard exams? Be sure to check the assumptions for using the
    normal approximation.

    \color{blue}

    Since the participants were randomly assigned to each group, the
    groups can be treated as independent, and it is reasonable to assume
    independence of patients within each group.

    The pooled proportion \(\hat{p}\) is
    \[\hat{p} = \dfrac{x_1 + x_2}{n_1 + n_2} = 0.0112\]

    The success-failure condition is met; the expected number of
    successes is about 503 in both groups (and expected number of
    failures is naturally much larger, given that both \(\hat{p}\) is
    less than 0.50 and \(n\) is very large).

    Test \(H_0: p_M = p_C\) against \(H_A: p_M \neq p_C\). Let
    \(\alpha = 0.05\).

    The two-sided \(p\)-value is 0.895. Since \(p > \alpha\), there is
    insufficient evidence to reject the null hypothesis; the observed
    difference in proportions of breast cancer deaths is reasonably
    explained by chance. The results do not suggest that annual
    mammography results in a reduction in breast cancer mortality
    relative to standard exams.

    \color{black}
  \end{enumerate}

\begin{Shaded}
\begin{Highlighting}[]
\CommentTok{#use r as a calculator}
\NormalTok{p.hat.pooled =}\StringTok{ }\KeywordTok{sum}\NormalTok{(x)}\OperatorTok{/}\KeywordTok{sum}\NormalTok{(n)}
\NormalTok{p.hat.pooled}
\end{Highlighting}
\end{Shaded}

  \begin{ShadedResult}
   \begin{verbatim}
   #  [1] 0.01118718
   \end{verbatim}
   \end{ShadedResult}

\begin{Shaded}
\begin{Highlighting}[]
\CommentTok{#check success-failure}
\NormalTok{n}\OperatorTok{*}\NormalTok{p.hat.pooled}
\end{Highlighting}
\end{Shaded}

  \begin{ShadedResult}
   \begin{verbatim}
   #  [1] 502.5839 502.4161
   \end{verbatim}
   \end{ShadedResult}

\begin{Shaded}
\begin{Highlighting}[]
\NormalTok{n}\OperatorTok{*}\NormalTok{(}\DecValTok{1} \OperatorTok{-}\StringTok{ }\NormalTok{p.hat.pooled)}
\end{Highlighting}
\end{Shaded}

  \begin{ShadedResult}
   \begin{verbatim}
   #  [1] 44422.42 44407.58
   \end{verbatim}
   \end{ShadedResult}

\begin{Shaded}
\begin{Highlighting}[]
\CommentTok{#conduct inference}
\KeywordTok{prop.test}\NormalTok{(}\DataTypeTok{x =}\NormalTok{ x, }\DataTypeTok{n =}\NormalTok{ n)}\OperatorTok{$}\NormalTok{p.val}
\end{Highlighting}
\end{Shaded}

  \begin{ShadedResult}
   \begin{verbatim}
   #  [1] 0.8948174
   \end{verbatim}
   \end{ShadedResult}

  \begin{enumerate}
  \def\labelenumii{\alph{enumii})}
  \setcounter{enumii}{2}
  \item
    Calculate and interpret a 95\% confidence interval for the
    difference in proportions of deaths from breast cancer. Be sure to
    check the assumptions for using the normal approximation.

    \color{blue}

    The success-failure condition should be checked for each sample:
    from the data, the number of successes and failures are both well
    over 10 in each group.

    The 95\% confidence interval is (-0.0015, 0.0013). With 95\%
    confidence, the difference in probability of death is within the
    interval (-0.15\%, 0.13\%); i.e., 0.15\% lower in the mammogram
    group to 0.13\% higher in the mammogram group. As expected from the
    large \(p\)-value, the confidence interval contains the null value
    0.

    \color{black}
  \end{enumerate}

\begin{Shaded}
\begin{Highlighting}[]
\CommentTok{#conduct inference}
\KeywordTok{prop.test}\NormalTok{(}\DataTypeTok{x =}\NormalTok{ x, }\DataTypeTok{n =}\NormalTok{ n)}\OperatorTok{$}\NormalTok{conf.int}
\end{Highlighting}
\end{Shaded}

  \begin{ShadedResult}
   \begin{verbatim}
   #  [1] -0.001512853  0.001282751
   #  attr(,"conf.level")
   #  [1] 0.95
   \end{verbatim}
   \end{ShadedResult}
\end{enumerate}

\newpage

\begin{enumerate}
\def\labelenumi{\arabic{enumi}.}
\setcounter{enumi}{1}
\item
  Remdesivir is an antiviral drug previously tested in animal models
  infected with coronaviruses like SARS and MERS. As of May 2020,
  remdesivir had temporary approval from the FDA for use in severely ill
  COVID-19 patients and was the subject of numerous ongoing studies.

  A randomized controlled trial conducted in China enrolled 236 patients
  with severe COVID-19; 158 were assigned to receive remdesivir and 78
  to receive a placebo. In the remdesivir group, 103 patients showed
  clinical improvement; in the placebo group, 45 patients showed
  clinical
  improvement.\footnote{Wang, Y, et al. Remdesivir in adults with severe COVID-19: a randomised, double-blind, placebo-controlled, multi-centre trial. \textit{Lancet} 395(10236). 16 May 2020.}

  \begin{enumerate}
  \def\labelenumii{\alph{enumii})}
  \item
    Calculate \(\hat{p}_1\) and \(\hat{p}_2\), the two sample
    proportions of interest.

    \textcolor{blue}{The two sample proportions of interest are 0.652 and 0.577, the proportion of individuals in each treatment group that showed clinical improvement.}
  \end{enumerate}

\begin{Shaded}
\begin{Highlighting}[]
\CommentTok{#use r as a calculator}
\NormalTok{x =}\StringTok{ }\KeywordTok{c}\NormalTok{(}\DecValTok{103}\NormalTok{, }\DecValTok{45}\NormalTok{)}
\NormalTok{n =}\StringTok{ }\KeywordTok{c}\NormalTok{(}\DecValTok{158}\NormalTok{, }\DecValTok{78}\NormalTok{)}
\NormalTok{p.hat.vector =}\StringTok{ }\NormalTok{x}\OperatorTok{/}\NormalTok{n}
\NormalTok{p.hat.vector}
\end{Highlighting}
\end{Shaded}

  \begin{ShadedResult}
   \begin{verbatim}
   #  [1] 0.6518987 0.5769231
   \end{verbatim}
   \end{ShadedResult}

  \begin{enumerate}
  \def\labelenumii{\alph{enumii})}
  \setcounter{enumii}{1}
  \item
    Conduct a formal comparison of the clinical improvement rates and
    summarize your findings. Be sure to check the assumptions for using
    the normal approximation.

    \color{blue}

    Since the participants were randomly assigned to each group, the
    groups can be treated as independent, and it is reasonable to assume
    independence of patients within each group.

    The pooled proportion \(\hat{p}\) is
    \[\hat{p} = \dfrac{x_1 + x_2}{n_1 + n_2} = 0.627\]

    The success-failure condition is met; the expected number of
    successes and failures are all larger than 10.

    Test \(H_0: p_1 = p_2\) against \(H_A: p_1 \neq p_2\), where \(p_1\)
    represents the population proportion of clinical improvement in
    COVID-19 patients treated with remdesivir and \(p_2\) represents the
    population proportion of clinical improvement in COVID-19 patients
    treated with placebo. Let \(\alpha = 0.05\). The \(p\)-value is
    0.328, which is greater than \(\alpha\); there is insufficient
    evidence to reject the null hypothesis of no difference. Even though
    the proportion of patients who experienced clinical improvement
    about 7\% higher in the remdesivir group, this difference is not
    extreme enough to represent sufficient evidence that remdesivir is
    more effective than placebo.

    \color{black}
  \end{enumerate}

\begin{Shaded}
\begin{Highlighting}[]
\CommentTok{#use r as a calculator}
\NormalTok{p.hat.pooled =}\StringTok{ }\KeywordTok{sum}\NormalTok{(x)}\OperatorTok{/}\KeywordTok{sum}\NormalTok{(n)}
\NormalTok{p.hat.pooled}
\end{Highlighting}
\end{Shaded}

  \begin{ShadedResult}
   \begin{verbatim}
   #  [1] 0.6271186
   \end{verbatim}
   \end{ShadedResult}

\begin{Shaded}
\begin{Highlighting}[]
\CommentTok{#check success-failure}
\NormalTok{n}\OperatorTok{*}\NormalTok{p.hat.pooled}
\end{Highlighting}
\end{Shaded}

  \begin{ShadedResult}
   \begin{verbatim}
   #  [1] 99.08475 48.91525
   \end{verbatim}
   \end{ShadedResult}

\begin{Shaded}
\begin{Highlighting}[]
\NormalTok{n}\OperatorTok{*}\NormalTok{(}\DecValTok{1} \OperatorTok{-}\StringTok{ }\NormalTok{p.hat.pooled)}
\end{Highlighting}
\end{Shaded}

  \begin{ShadedResult}
   \begin{verbatim}
   #  [1] 58.91525 29.08475
   \end{verbatim}
   \end{ShadedResult}

\begin{Shaded}
\begin{Highlighting}[]
\CommentTok{#conduct inference}
\KeywordTok{prop.test}\NormalTok{(}\DataTypeTok{x =}\NormalTok{ x, }\DataTypeTok{n =}\NormalTok{ n)}\OperatorTok{$}\NormalTok{p.val}
\end{Highlighting}
\end{Shaded}

  \begin{ShadedResult}
   \begin{verbatim}
   #  [1] 0.3284038
   \end{verbatim}
   \end{ShadedResult}

  \begin{enumerate}
  \def\labelenumii{\alph{enumii})}
  \setcounter{enumii}{2}
  \item
    Report and interpret an appropriate interval estimate. Be sure to
    check the assumptions for using the normal approximation.

    \textcolor{blue}{The success-failure condition should be checked for each sample: from the data, the number of successes and failures are both well over 10 in each group. The 95\% confidence interval is (-0.067, 0.217); with 95\% confidence, this interval captures the difference in population proportion of clinical mortality between COVID-19 patients treated with remdesivir and those treated with placebo. The interval contains 0, which is consistent with no statistically significant evidence of a difference. The interval reflects the lack of precision around the effect estimate that is characteristic of an insufficiently large sample size.}
  \end{enumerate}

\begin{Shaded}
\begin{Highlighting}[]
\KeywordTok{prop.test}\NormalTok{(}\DataTypeTok{x =}\NormalTok{ x, }\DataTypeTok{n =}\NormalTok{ n)}\OperatorTok{$}\NormalTok{conf.int}
\end{Highlighting}
\end{Shaded}

  \begin{ShadedResult}
   \begin{verbatim}
   #  [1] -0.06703113  0.21698245
   #  attr(,"conf.level")
   #  [1] 0.95
   \end{verbatim}
   \end{ShadedResult}
\end{enumerate}

\newpage

\hypertarget{contingency-tables}{%
\section{Contingency Tables}\label{contingency-tables}}

\textbf{The \(\chi^2\) test of independence}

In the \(\chi^2\) test of independence, the observed number of cell
counts are compared to the number of \textbf{expected} cell counts,
where the expected counts are calculated under the null hypothesis.

\begin{itemize}
\item
  \(H_0\): the row and column variables are not associated
\item
  \(H_A\): the row and column variables are associated
\end{itemize}

The expected count for the \(i^{th}\) row and \(j^{th}\) column is
\[E_{i, j} = \dfrac{(\text{row $i$ total}) \times (\text{column $j$ total}) }{n}, \]
where \(n\) is the total number of observations.

Assumptions for the \(\chi^2\) test:

\begin{itemize}
\item
  \emph{Independence}. Each case that contributes a count to the table
  must be independent of all other cases in the table.
\item
  \emph{Sample size}. Each expected cell count must be greater than or
  equal to 10. For tables larger than \(2 \times 2\), it is appropriate
  to use the test if no more than 1/5 of the expected counts are less
  than 5, and all expected counts are greater than 1.
\end{itemize}

The \textbf{\(\chi^2\) test statistic} is calculated as
\[\chi^2 = \sum_{i = 1}^r \sum_{j = 1}^c \dfrac{(O_{i, j} - E_{i, j})^2}{E_{i, j}}, \]
and is approximately distributed \(\chi^2\) with degrees of freedom
\((r - 1)(c - 1)\), where \(r\) is the number of rows and \(c\) is the
number of columns. \(O_{i, j}\) represents the observed count in row
\(i\), column \(j\).

For each cell in a table, the \textbf{residual} equals
\[\dfrac{O_{i, j} - E_{i, j}}{\sqrt{E_{i,j}}}. \] Residuals with a large
magnitude contribute the most to the \(\chi^2\) statistic. If a residual
is positive, the observed value is greater than the expected value; if a
residual is negative, the observed value is less than the expected.

\vspace{0.5cm}

\emph{Fisher's exact test}

When the expected counts in a two-way table are less than 10, Fisher's
exact test is used to compute a \(p\)-value without relying on the
normal approximation. In this course, only the logic behind Fisher's
exact test for a \(2 \times 2\) table is discussed. In the
\(2 \times 2\) table case, the hypotheses for Fisher's exact test can be
expressed in the same way as for a two-sample test of proportions; the
null hypothesis is \(H_0: p_1 = p_2\).

The \(p\)-value is the probability of observing results as or more
extreme than those observed under the assumption that the null
hypothesis is true.

\begin{itemize}
\item
  Thus, the \(p\)-value is calculated by adding together the individual
  conditional probabilities of obtaining each table that is as or more
  extreme than the one observed, under the null hypothesis and given
  that the marginal totals are considered fixed.
\item
  When the marginal totals are held constant, the value of any one cell
  in the table determines the rest of entries. When marginal totals are
  considered fixed, each table represents a unique set of results.
\item
  Extreme tables are those which contradict \(H_0: p_1 = p_2\).
\item
  A two-sided \(p\)-value can be calculated by doubling the smaller of
  the possible one-sided \(p\)-values; this method is typically used
  when calculating \(p\)-values by hand. Another common method is to
  classify ``more extreme'' tables as all tables with probabilities less
  than that of the observed table, in both directions; the \(p\)-value
  is the sum of probabilities for the qualifying table.
\end{itemize}

The probability of a particular table (i.e., set of results) can be
calculated with the \textbf{hypergeometric distribution}.

Let \(X\) represent the number of successes in a series of repeated
Bernoulli trials, where sampling is done without replacement. Suppose
that in the population of size \(N\), there are \(m\) total successes.
What is the probability of observing exactly \(k\) successes when
drawing a sample of size \(n\)?

For example, imagine an urn with \(m\) white balls and \(N - m\) black
balls (thus, there are \(N\) total balls). Draw \(n\) balls without
replacement (i.e., a sample of \(n\) balls). What is the probability of
observing \(k\) white balls in the sample?

The possible results of a sample can be organized in a \(2 \times 2\)
table:

\begin{table*}[h!]
\begin{center}
\begin{tabular}{l|cc|c} 
   & \textbf{White Ball} & \textbf{Black Ball} & \textbf{Total}\\ \hline
  \textbf{Sampled} & $k$ & \textcolor{gray}{$n - k$}  & $n$  \\
  \textbf{Not Sampled} & \textcolor{gray}{$m - k$} & \textcolor{gray}{$N - n - (m - k)$} & \textcolor{gray}{$N - n$} \\ \hline
  \textbf{Total} & $m$ & $N - m$ & $N$  \\ 
\end{tabular}\\
\end{center}
\end{table*}

The probability of observing exactly \(k\) successses in a sample of
size \(n\) (i.e., \(n\) dependent trials) is given by
\[P(X = k) = \dfrac{{m \choose k} {N - m \choose n-k}}{{N \choose n}}. \]

Hypergeometric probabilities are calculated in \textsf{R} with the use
of \texttt{dhyper()} and \texttt{phyper()}. The following code shows how
to calculate \(P(X = 5)\), \(P(X \leq 5)\), and \(P(X > 5)\) for
\(X \sim \text{HGeom}(10, 15, 8)\), where \(m = 10\), \(N - m = 15\),
and \(n = 8\).

\begin{Shaded}
\begin{Highlighting}[]
\CommentTok{#probability X equals 5}
\KeywordTok{dhyper}\NormalTok{(}\DecValTok{5}\NormalTok{, }\DecValTok{10}\NormalTok{, }\DecValTok{15}\NormalTok{, }\DecValTok{8}\NormalTok{)}
\end{Highlighting}
\end{Shaded}

\begin{ShadedResult}
\begin{verbatim}
#  [1] 0.1060121
\end{verbatim}
\end{ShadedResult}

\begin{Shaded}
\begin{Highlighting}[]
\CommentTok{#probability X is less than or equal to 5}
\KeywordTok{phyper}\NormalTok{(}\DecValTok{5}\NormalTok{, }\DecValTok{10}\NormalTok{, }\DecValTok{15}\NormalTok{, }\DecValTok{8}\NormalTok{)}
\end{Highlighting}
\end{Shaded}

\begin{ShadedResult}
\begin{verbatim}
#  [1] 0.9779072
\end{verbatim}
\end{ShadedResult}

\begin{Shaded}
\begin{Highlighting}[]
\CommentTok{#probability X is greater than 5}
\KeywordTok{phyper}\NormalTok{(}\DecValTok{5}\NormalTok{, }\DecValTok{10}\NormalTok{, }\DecValTok{15}\NormalTok{, }\DecValTok{8}\NormalTok{, }\DataTypeTok{lower.tail =} \OtherTok{FALSE}\NormalTok{)}
\end{Highlighting}
\end{Shaded}

\begin{ShadedResult}
\begin{verbatim}
#  [1] 0.02209278
\end{verbatim}
\end{ShadedResult}

\vspace{0.5cm}

\emph{Measures of association in two-by-two tables}

Chapter 1 introduced the \textbf{relative risk (RR)}, a measure of the
risk of a certain event occurring in one group relative to the risk of
the event occurring in another group, as a numerical summary for
two-by-two (\(2 \times 2\)) tables. The relative risk can also be
thought of as a measure of association.

Consider the following hypothetical two-by-two table. The relative risk
of Outcome A can be calculated by using either Group 1 or Group 2 as the
reference group:

\begin{table}[h!]
    \centering
    \begin{tabular}{r|rrr}
        \hline
        & Outcome A & Outcome B & Sum\\ 
        \hline
        Group 1 & $a$ & $b$ & $a + b$ \\ 
        Group 2 & $c$ & $d$ & $c + d$ \\
        Sum & $a + c$ & $b + d$ & $a + b + c + d = n$ \\
        \hline
    \end{tabular}   
    \caption{A hypothetical two-by-two table of outcome by group.}
\end{table}

\[RR_{\text{A, comparing Group 1 to Group 2}} = \dfrac{a/(a + b)}{c/(c+d)} \]
\[RR_{\text{A, comparing Group 2 to Group 1}} = \dfrac{c/(c + d)}{a/(a+b)} \]

The relative risk is only valid for tables where the proportions
\(a/(a + b)\) and \(c/(c + d)\) represent the incidence of Outcome A
within the populations from which Groups 1 and 2 are sampled.

The \textbf{odds ratio (OR)} is a measure of association that remains
applicable even when it is not possible to estimate incidence of an
outcome from the sample data. The \textbf{odds} of Outcome A in Group 1
are \(a/b\), while the odds of Outcome A in Group 2 are \(c/d\).

\[OR_{\text{A, comparing Group 1 to Group 2}} = \dfrac{a/b}{c/d} = \dfrac{ad}{bc} \]
\[OR_{\text{A, comparing Group 2 to Group 1}} = \dfrac{c/d}{a/b} = \dfrac{bc}{ad} \]

\vspace{0.5cm}

\hypertarget{the-chi2-test-of-independence}{%
\section{\texorpdfstring{The \(\chi^2\) test of
independence}{The \textbackslash chi\^{}2 test of independence}}\label{the-chi2-test-of-independence}}

\begin{enumerate}
\def\labelenumi{\arabic{enumi}.}
\setcounter{enumi}{2}
\item
  In resource-limited settings, single-dose nevirapine (NVP) is given to
  an HIV-positive woman during birth to prevent mother-to-child
  transmission of the virus. Exposure of the infant to NVP may foster
  the growth of more virulent strains of the virus in the child.

  If a child is HIV-positive, should they be treated with NVP or a more
  expensive drug, lopinavir (LPV)? In this setting, success means
  preventing a growth of the virus in the child (i.e., preventing
  virologic failure). The following table contains data from a 2012
  study conducted in six African countries and
  India.\footnote{A. Violari, et al. "Nevirapine versus ritonavir-boosted lopinavir for HIV-infected children." \textit{NEJM} 366: 2380-2389.}

  \begin{center}
  \begin{tabular}{l|cc|c} 
  & \textbf{NVP} & \textbf{LPV} & \textbf{Total}\\ \hline
    \textbf{Virologic Failure} & 60 & 27 & 87  \\
    \textbf{Stable Disease} & 87 & 113 & 200 \\ \hline
    \textbf{Total} & 147 & 140 & 287  \\ 
  \end{tabular}\\
  \end{center}

  \vspace{0.5cm}

  \begin{enumerate}
  \def\labelenumii{\alph{enumii})}
  \item
    State the null and alternative hypotheses.

    \color{blue}

    The null hypothesis is that there is no association between
    treatment and outcome; i.e., treatment and outcome are independent.

    The alternative hypothesis is that there is an association between
    treatment and outcome; i.e., treatment and outcome are not
    independent.

    \color{black}
  \item
    Calculate the expected cell counts.

    \color{blue}

    The expected cell counts are shown in parentheses next to the
    observed cell counts.

    \begin{center}
      \begin{tabular}{l|cc|c} 
    & \textbf{NVP} & \textbf{LPV} & \textbf{Total}\\ \hline
      \textbf{Virologic Failure} & 60 (44.56) & 27 (42.44) & 87  \\ 
      \textbf{Stable Disease} & 87 (102.44) & 113 (97.56)& 200 \\ \hline
      \textbf{Total} & 147 & 140 & 287  \\ 
      \end{tabular}\\
     \end{center}

    \color{black}
  \end{enumerate}

\begin{Shaded}
\begin{Highlighting}[]
\CommentTok{#use r as a calculator}

\CommentTok{#set parameters}
\NormalTok{n =}\StringTok{ }\DecValTok{287}
\NormalTok{row.}\FloatTok{1.}\NormalTok{total =}\StringTok{ }\DecValTok{87}
\NormalTok{row.}\FloatTok{2.}\NormalTok{total =}\StringTok{ }\DecValTok{200}
\NormalTok{col.}\FloatTok{1.}\NormalTok{total =}\StringTok{ }\DecValTok{147}
\NormalTok{col.}\FloatTok{2.}\NormalTok{total =}\StringTok{ }\DecValTok{140}

\CommentTok{#calculate expected values}
\NormalTok{exp.}\FloatTok{1.1}\NormalTok{ =}\StringTok{ }\NormalTok{(row.}\FloatTok{1.}\NormalTok{total }\OperatorTok{*}\StringTok{ }\NormalTok{col.}\FloatTok{1.}\NormalTok{total)}\OperatorTok{/}\NormalTok{n}
\NormalTok{exp.}\FloatTok{1.1}
\end{Highlighting}
\end{Shaded}

  \begin{ShadedResult}
   \begin{verbatim}
   #  [1] 44.56098
   \end{verbatim}
   \end{ShadedResult}

\begin{Shaded}
\begin{Highlighting}[]
\NormalTok{exp.}\FloatTok{1.2}\NormalTok{ =}\StringTok{ }\NormalTok{(row.}\FloatTok{1.}\NormalTok{total }\OperatorTok{*}\StringTok{ }\NormalTok{col.}\FloatTok{2.}\NormalTok{total)}\OperatorTok{/}\NormalTok{n}
\NormalTok{exp.}\FloatTok{1.2}
\end{Highlighting}
\end{Shaded}

  \begin{ShadedResult}
   \begin{verbatim}
   #  [1] 42.43902
   \end{verbatim}
   \end{ShadedResult}

\begin{Shaded}
\begin{Highlighting}[]
\NormalTok{exp.}\FloatTok{2.1}\NormalTok{ =}\StringTok{ }\NormalTok{(row.}\FloatTok{2.}\NormalTok{total }\OperatorTok{*}\StringTok{ }\NormalTok{col.}\FloatTok{1.}\NormalTok{total)}\OperatorTok{/}\NormalTok{n}
\NormalTok{exp.}\FloatTok{2.1}
\end{Highlighting}
\end{Shaded}

  \begin{ShadedResult}
   \begin{verbatim}
   #  [1] 102.439
   \end{verbatim}
   \end{ShadedResult}

\begin{Shaded}
\begin{Highlighting}[]
\NormalTok{exp.}\FloatTok{2.2}\NormalTok{ =}\StringTok{ }\NormalTok{(row.}\FloatTok{2.}\NormalTok{total }\OperatorTok{*}\StringTok{ }\NormalTok{col.}\FloatTok{2.}\NormalTok{total)}\OperatorTok{/}\NormalTok{n}
\NormalTok{exp.}\FloatTok{2.2}
\end{Highlighting}
\end{Shaded}

  \begin{ShadedResult}
   \begin{verbatim}
   #  [1] 97.56098
   \end{verbatim}
   \end{ShadedResult}

  \begin{enumerate}
  \def\labelenumii{\alph{enumii})}
  \setcounter{enumii}{2}
  \item
    Check the assumptions for using the \(\chi^2\) test.

    \textcolor{blue}{Independence holds, since this is a randomized study. All expected counts are greater than 10.}
  \item
    Calculate the \(\chi^2\) test statistic.

    \color{blue}

    \[\chi^2 = \sum \dfrac{(\text{obs - exp})^2}{\text{exp}} = \dfrac{(60-44.56)^2}{44.56} + \dfrac{(27-42.44)^2}{42.44} +  \dfrac{(87-102.44)^2}{102.44} + \dfrac{(113 - 97.56)^2}{97.56} = 15.74\]

    \color{black}
  \end{enumerate}

\begin{Shaded}
\begin{Highlighting}[]
\CommentTok{#use r as a calculator}
\NormalTok{obs.}\FloatTok{1.1}\NormalTok{ =}\StringTok{ }\DecValTok{60}
\NormalTok{chi.sq.}\FloatTok{1.1}\NormalTok{ =}\StringTok{ }\NormalTok{((obs.}\FloatTok{1.1} \OperatorTok{-}\StringTok{ }\NormalTok{exp.}\FloatTok{1.1}\NormalTok{)}\OperatorTok{^}\DecValTok{2}\NormalTok{)}\OperatorTok{/}\NormalTok{exp.}\FloatTok{1.1}

\NormalTok{obs.}\FloatTok{1.2}\NormalTok{ =}\StringTok{ }\DecValTok{27}
\NormalTok{chi.sq.}\FloatTok{1.2}\NormalTok{ =}\StringTok{ }\NormalTok{((obs.}\FloatTok{1.2} \OperatorTok{-}\StringTok{ }\NormalTok{exp.}\FloatTok{1.2}\NormalTok{)}\OperatorTok{^}\DecValTok{2}\NormalTok{)}\OperatorTok{/}\NormalTok{exp.}\FloatTok{1.2}

\NormalTok{obs.}\FloatTok{2.1}\NormalTok{ =}\StringTok{ }\DecValTok{87}
\NormalTok{chi.sq.}\FloatTok{2.1}\NormalTok{ =}\StringTok{ }\NormalTok{((obs.}\FloatTok{2.1} \OperatorTok{-}\StringTok{ }\NormalTok{exp.}\FloatTok{2.1}\NormalTok{)}\OperatorTok{^}\DecValTok{2}\NormalTok{)}\OperatorTok{/}\NormalTok{exp.}\FloatTok{2.1}

\NormalTok{obs.}\FloatTok{2.2}\NormalTok{ =}\StringTok{ }\DecValTok{113}
\NormalTok{chi.sq.}\FloatTok{2.2}\NormalTok{ =}\StringTok{ }\NormalTok{((obs.}\FloatTok{2.2} \OperatorTok{-}\StringTok{ }\NormalTok{exp.}\FloatTok{2.2}\NormalTok{)}\OperatorTok{^}\DecValTok{2}\NormalTok{)}\OperatorTok{/}\NormalTok{exp.}\FloatTok{2.2}

\NormalTok{chi.sq =}\StringTok{ }\NormalTok{chi.sq.}\FloatTok{1.1} \OperatorTok{+}\StringTok{ }\NormalTok{chi.sq.}\FloatTok{1.2} \OperatorTok{+}\StringTok{ }\NormalTok{chi.sq.}\FloatTok{2.1} \OperatorTok{+}\StringTok{ }\NormalTok{chi.sq.}\FloatTok{2.2}
\NormalTok{chi.sq}
\end{Highlighting}
\end{Shaded}

  \begin{ShadedResult}
   \begin{verbatim}
   #  [1] 15.73587
   \end{verbatim}
   \end{ShadedResult}

  \begin{enumerate}
  \def\labelenumii{\alph{enumii})}
  \setcounter{enumii}{4}
  \item
    Calculate the \(p\)-value for the test statistic using
    \texttt{pchisq()}. The \(p\)-value represents the probability of
    observing a result as or more extreme than the sample data.

    \textcolor{blue}{The $p$-value for the test statistic is $7.28 \times 10^{-5}$.}
  \end{enumerate}

\begin{Shaded}
\begin{Highlighting}[]
\CommentTok{#use pchisq()}
\KeywordTok{pchisq}\NormalTok{(chi.sq, }\DataTypeTok{df =}\NormalTok{ (}\DecValTok{2} \OperatorTok{-}\StringTok{ }\DecValTok{1}\NormalTok{)}\OperatorTok{*}\NormalTok{(}\DecValTok{2} \OperatorTok{-}\StringTok{ }\DecValTok{1}\NormalTok{), }\DataTypeTok{lower.tail =} \OtherTok{FALSE}\NormalTok{)}
\end{Highlighting}
\end{Shaded}

  \begin{enumerate}
  \def\labelenumii{\alph{enumii})}
  \setcounter{enumii}{5}
  \item
    Confirm the results from parts c) and d) using
    \texttt{chisq.test()}. Note that the value of the test statistic
    will be slightly different because \textsf{R} is applying a
    `continuity correction'.

    \textcolor{blue}{From \texttt{chisq.test()}, the $\chi^2$ statistic is 14.73 and the associated $p$-value is 0.0001.}
  \end{enumerate}

\begin{Shaded}
\begin{Highlighting}[]
\CommentTok{#enter the data as a table}
\NormalTok{hiv.table =}\StringTok{ }\KeywordTok{matrix}\NormalTok{(}\KeywordTok{c}\NormalTok{(}\DecValTok{60}\NormalTok{, }\DecValTok{27}\NormalTok{, }\DecValTok{87}\NormalTok{, }\DecValTok{113}\NormalTok{), }
              \DataTypeTok{nrow =} \DecValTok{2}\NormalTok{, }\DataTypeTok{ncol =} \DecValTok{2}\NormalTok{, }\DataTypeTok{byrow =}\NormalTok{ T)}

\CommentTok{#add labels and confirm the table was entered correctly}
\KeywordTok{dimnames}\NormalTok{(hiv.table) =}\StringTok{ }\KeywordTok{list}\NormalTok{(}\StringTok{"Outcome"}\NormalTok{ =}\StringTok{ }\KeywordTok{c}\NormalTok{(}\StringTok{"Virologic Failure"}\NormalTok{, }\StringTok{"Stable Disease"}\NormalTok{),}
                        \StringTok{"Drug"}\NormalTok{ =}\StringTok{ }\KeywordTok{c}\NormalTok{(}\StringTok{"NVP"}\NormalTok{, }\StringTok{"LPV"}\NormalTok{))}
\NormalTok{hiv.table}
\end{Highlighting}
\end{Shaded}

  \begin{ShadedResult}
   \begin{verbatim}
   #                     Drug
   #  Outcome             NVP LPV
   #    Virologic Failure  60  27
   #    Stable Disease     87 113
   \end{verbatim}
   \end{ShadedResult}

\begin{Shaded}
\begin{Highlighting}[]
\CommentTok{#use chisq.test()}
\KeywordTok{chisq.test}\NormalTok{(hiv.table)}
\end{Highlighting}
\end{Shaded}

  \begin{ShadedResult}
   \begin{verbatim}
   #  
   #   Pearson's Chi-squared test with Yates' continuity correction
   #  
   #  data:  hiv.table
   #  X-squared = 14.733, df = 1, p-value = 0.0001238
   \end{verbatim}
   \end{ShadedResult}

  \begin{enumerate}
  \def\labelenumii{\alph{enumii})}
  \setcounter{enumii}{6}
  \item
    Summarize the conclusions; be sure to include which drug is
    recommended for treatment, based on the data.

    \color{blue}

    There is sufficient evidence at the \(\alpha = 0.05\) significance
    level to reject the null hypothesis and accept the alternative
    hypothesis that treatment and outcome are associated.

    From comparing the expected and observed cell counts (or looking at
    the residuals), it is possible to determine the direction of the
    association. When treated with lopinarvir, fewer children than
    expected experience virologic failure (27 observed versus
    \textasciitilde42 expected), and more than expected experience
    stable disease (113 observed versus \textasciitilde98 expected). In
    contrast, when treated with nevirapine, more children than expected
    experience virologic failure (60 observed versus \textasciitilde45
    expected), and fewer children than expected experience stable
    disease (87 observed versus \textasciitilde102 expected).

    The data suggest that HIV-positive children should be treated with
    lopinarvir.

    \color{black}
  \end{enumerate}

\begin{Shaded}
\begin{Highlighting}[]
\CommentTok{#look at residuals}
\KeywordTok{chisq.test}\NormalTok{(hiv.table)}\OperatorTok{$}\NormalTok{resid}
\end{Highlighting}
\end{Shaded}

  \begin{ShadedResult}
   \begin{verbatim}
   #                     Drug
   #  Outcome                   NVP       LPV
   #    Virologic Failure  2.312824 -2.369939
   #    Stable Disease    -1.525412  1.563082
   \end{verbatim}
   \end{ShadedResult}

\begin{Shaded}
\begin{Highlighting}[]
\CommentTok{#to view the expected values, use $expected}
\KeywordTok{chisq.test}\NormalTok{(hiv.table)}\OperatorTok{$}\NormalTok{expected}
\end{Highlighting}
\end{Shaded}

  \begin{ShadedResult}
   \begin{verbatim}
   #                     Drug
   #  Outcome                   NVP      LPV
   #    Virologic Failure  44.56098 42.43902
   #    Stable Disease    102.43902 97.56098
   \end{verbatim}
   \end{ShadedResult}

  \begin{enumerate}
  \def\labelenumii{\alph{enumii})}
  \setcounter{enumii}{7}
  \item
    Repeat the analysis using inference for the difference of two
    proportions and confirm that the results are the same.

    \textcolor{blue}{The proportion of successes on nevirapine is 0.59 and the proportion of successes on lopinarvir is 0.81. The $p$-value is 0.0012; there is sufficient evidence to reject the null of no difference and conclude that stable disease is associated with lopinarvir.}
  \end{enumerate}

\begin{Shaded}
\begin{Highlighting}[]
\CommentTok{#use prop.test( )}
\KeywordTok{prop.test}\NormalTok{(}\DataTypeTok{x =} \KeywordTok{c}\NormalTok{(}\DecValTok{87}\NormalTok{, }\DecValTok{113}\NormalTok{), }\DataTypeTok{n =} \KeywordTok{c}\NormalTok{(}\DecValTok{147}\NormalTok{, }\DecValTok{140}\NormalTok{))}
\end{Highlighting}
\end{Shaded}

  \begin{ShadedResult}
   \begin{verbatim}
   #  
   #   2-sample test for equality of proportions with continuity correction
   #  
   #  data:  c(87, 113) out of c(147, 140)
   #  X-squared = 14.733, df = 1, p-value = 0.0001238
   #  alternative hypothesis: two.sided
   #  95 percent confidence interval:
   #   -0.3251572 -0.1054551
   #  sample estimates:
   #     prop 1    prop 2 
   #  0.5918367 0.8071429
   \end{verbatim}
   \end{ShadedResult}
\item
  In the PREVEND study introduced in Chapter 6, researchers measured
  various features of study participants, including data on statin use
  and highest level of education attained. From the data in
  \texttt{prevend.samp}, is there evidence of an association between
  statin use and educational level? Summarize the results.

  \color{blue}

  Test the null hypothesis that statin use and education level are not
  associated against the alternative hypothesis that statin use and
  education level are associated. Let \(\alpha = 0.05\).

  The \(p\)-value of the \(\chi^2\) statistic is 0.0003. The results are
  highly significant, and there is evidence to support accepting the
  alternative hypothesis that statin use and education level are
  associated.

  The largest deviations from independence occur in the primary school
  group and university group. There are more statin users than expected
  in the primary school group and fewer statin users than expected in
  the university group. There is an observable overall trend; as highest
  educational level attained increases, the proportion of statin users
  goes from higher than expected to lower than expected.

  \color{black}

\begin{Shaded}
\begin{Highlighting}[]
\CommentTok{#load the data}
\KeywordTok{library}\NormalTok{(oibiostat)}
\KeywordTok{data}\NormalTok{(}\StringTok{"prevend.samp"}\NormalTok{)}

\CommentTok{#convert variables to factors}
\NormalTok{prevend.samp}\OperatorTok{$}\NormalTok{Statin =}\StringTok{ }\KeywordTok{factor}\NormalTok{(prevend.samp}\OperatorTok{$}\NormalTok{Statin, }\DataTypeTok{levels =} \KeywordTok{c}\NormalTok{(}\DecValTok{0}\NormalTok{, }\DecValTok{1}\NormalTok{),}
                         \DataTypeTok{labels =} \KeywordTok{c}\NormalTok{(}\StringTok{"NonUser"}\NormalTok{, }\StringTok{"User"}\NormalTok{))}

\NormalTok{prevend.samp}\OperatorTok{$}\NormalTok{Education =}\StringTok{ }\KeywordTok{factor}\NormalTok{(prevend.samp}\OperatorTok{$}\NormalTok{Education, }\DataTypeTok{levels =} \DecValTok{0}\OperatorTok{:}\DecValTok{3}\NormalTok{,}
                            \DataTypeTok{labels =} \KeywordTok{c}\NormalTok{(}\StringTok{"Primary"}\NormalTok{, }\StringTok{"LowerSec"}\NormalTok{,}
                                       \StringTok{"UpperSec"}\NormalTok{, }\StringTok{"Univ"}\NormalTok{))}

\CommentTok{#create a table}
\NormalTok{statin.edu.table =}\StringTok{ }\KeywordTok{table}\NormalTok{(prevend.samp}\OperatorTok{$}\NormalTok{Statin, prevend.samp}\OperatorTok{$}\NormalTok{Education)}

\CommentTok{#run chi-squared test}
\KeywordTok{chisq.test}\NormalTok{(statin.edu.table)}
\end{Highlighting}
\end{Shaded}

  \begin{ShadedResult}
   \begin{verbatim}
   #  
   #   Pearson's Chi-squared test
   #  
   #  data:  statin.edu.table
   #  X-squared = 19.054, df = 3, p-value = 0.0002665
   \end{verbatim}
   \end{ShadedResult}

\begin{Shaded}
\begin{Highlighting}[]
\KeywordTok{chisq.test}\NormalTok{(statin.edu.table)}\OperatorTok{$}\NormalTok{residuals}
\end{Highlighting}
\end{Shaded}

  \begin{ShadedResult}
   \begin{verbatim}
   #           
   #               Primary   LowerSec   UpperSec       Univ
   #    NonUser -1.3196995 -0.8994999  0.3760673  1.3000955
   #    User     2.4146629  1.6458208 -0.6880929 -2.3787932
   \end{verbatim}
   \end{ShadedResult}
\end{enumerate}

\newpage

\hypertarget{fishers-exact-test}{%
\section{Fisher's exact test}\label{fishers-exact-test}}

\begin{enumerate}
\def\labelenumi{\arabic{enumi}.}
\setcounter{enumi}{4}
\item
  \textit{Clostridium difficile} is a bacterium that causes inflammation
  of the colon. Antibiotic treatment is typically not effective,
  particularly for patients who experience multiple recurrences of
  infection. Infusion of feces from healthy donors has been reported as
  an effective treatment for recurrent infection. A randomized trial was
  conducted to compare the efficacy of donor-feces infusion versus
  vancomycin, the antibiotic typically prescribed to treat
  \textit{C. difficile }infection. The results of the trial are shown in
  the following table.

  \begin{table}[h]
   \centering
   \begin{tabular}{rrr|r}
       \hline
       & Cured & Uncured & Sum \\ 
       \hline
       Fecal Infusion & 13 & 3 & 16 \\ 
       Vancomycin & 4 & 9 & 13 \\ 
       \hline
       Sum & 17 & 12 & 29 \\ 
       \hline
   \end{tabular}
   \end{table}

  \begin{enumerate}
  \def\labelenumii{\alph{enumii})}
  \item
    Can a \(\chi^2\) test be used to analyze these results?

    \textcolor{blue}{A $\chi^2$ test is not advisable since there are expected counts less than 10.}
  \end{enumerate}

\begin{Shaded}
\begin{Highlighting}[]
\NormalTok{infusion.table =}\StringTok{ }\KeywordTok{matrix}\NormalTok{(}\KeywordTok{c}\NormalTok{(}\DecValTok{13}\NormalTok{, }\DecValTok{3}\NormalTok{, }\DecValTok{4}\NormalTok{, }\DecValTok{9}\NormalTok{), }\DataTypeTok{nrow =} \DecValTok{2}\NormalTok{, }\DataTypeTok{ncol =} \DecValTok{2}\NormalTok{, }\DataTypeTok{byrow =}\NormalTok{ T)}
\KeywordTok{dimnames}\NormalTok{(infusion.table) =}\StringTok{ }\KeywordTok{list}\NormalTok{(}\StringTok{"Outcome"}\NormalTok{ =}\StringTok{ }\KeywordTok{c}\NormalTok{(}\StringTok{"Cured"}\NormalTok{, }\StringTok{"Uncured"}\NormalTok{),}
                            \StringTok{"Treatment"}\NormalTok{ =}\StringTok{ }\KeywordTok{c}\NormalTok{(}\StringTok{"Fecal Infusion"}\NormalTok{, }
                                            \StringTok{"Vancomycin"}\NormalTok{))}

\KeywordTok{chisq.test}\NormalTok{(infusion.table)}\OperatorTok{$}\NormalTok{expected}
\end{Highlighting}
\end{Shaded}

  \begin{ShadedResult}
   \begin{verbatim}
   #           Treatment
   #  Outcome   Fecal Infusion Vancomycin
   #    Cured          9.37931    6.62069
   #    Uncured        7.62069    5.37931
   \end{verbatim}
   \end{ShadedResult}

  \begin{enumerate}
  \def\labelenumii{\alph{enumii})}
  \setcounter{enumii}{1}
  \item
    Researchers are interested in understanding whether fecal infusion
    is a more effective treatment than vancomycin. Write the null
    hypothesis and appropriate one-sided alternative hypothesis.

    \textcolor{blue}{Let $p_1$ represent the population proportion of individuals cured on the fecal infusion treatment and $p_2$ represent the population proportion of individuals cured on the vancomycin treatment. Under the null hypothesis, the proportion cured is equal between the two treatment groups: $H_0: p_1 = p_2$. The appropriate alternative hypothesis of interest is $H_0: p_1 > p_2$. }
  \item
    Under the assumption that the marginal totals are fixed, enumerate
    all possible sets of results that are more extreme than what was
    observed, in the same direction.

    \textcolor{blue}{Results more extreme than what was observed are those that constitute stronger evidence in favor of the fecal treatment group being more effective than vancomyin; i.e., results where either 14, 15, or 16 of the cured patients were in the fecal infusion group.}

    \color{blue}

    \begin{table}[h]
     \centering
     \color{gray}
     \begin{tabular}{r|cc|c}
     \hline
     & Cured & Uncured & Sum \\ 
     \hline
     Fecal Infusion & \textcolor{black}{14} & \textcolor{black}{2} & 16 \\ 
     Vancomycin & \textcolor{black}{3} & \textcolor{black}{10} & 13 \\ 
     \hline
     Sum & 17 & 12 & 29 \\ 
     \hline
     \end{tabular}
     \end{table}

    \begin{table}[h]
     \centering
     \color{gray}
     \begin{tabular}{r|cc|c}
     \hline
     & Cured & Uncured & Sum \\ 
     \hline
     Fecal Infusion & \textcolor{black}{15} & \textcolor{black}{1} & 16 \\ 
     Vancomycin & \textcolor{black}{2} & \textcolor{black}{11} & 13 \\ 
     \hline
     Sum & 17 & 12 & 29 \\ 
     \hline
     \end{tabular}
     \end{table}

    \begin{table}[h]
     \centering
     \color{gray}
     \begin{tabular}{r|cc|c}
     \hline
     & Cured & Uncured & Sum \\ 
     \hline
     Fecal Infusion & \textcolor{black}{16} & \textcolor{black}{0} & 16 \\ 
     Vancomycin & \textcolor{black}{1} & \textcolor{black}{12} & 13 \\ 
     \hline
     Sum & 17 & 12 & 29 \\ 
     \hline
     \end{tabular}
       \end{table}
  \end{enumerate}

  \color{black}

  \begin{enumerate}
  \def\labelenumii{\alph{enumii})}
  \setcounter{enumii}{3}
  \item
    Calculate the probability of the observed results.

    \textcolor{blue}{Use the hypergeometric distribution with parameters $N = 29$, $m = 17$, and $n = 16$; calculate $P(X = 13)$. Consider the "successes" to be the individuals cured and the "sample size" to be the number of individuals in the fecal infusion group. The probability that 13 of the cured individuals were in the fecal infusion group, given the table margins are fixed, is 0.0077.}
  \end{enumerate}

\begin{Shaded}
\begin{Highlighting}[]
\CommentTok{#probability of the observed results}
\KeywordTok{dhyper}\NormalTok{(}\DecValTok{13}\NormalTok{, }\DecValTok{17}\NormalTok{, }\DecValTok{29} \OperatorTok{-}\StringTok{ }\DecValTok{17}\NormalTok{, }\DecValTok{16}\NormalTok{)}
\end{Highlighting}
\end{Shaded}

  \begin{ShadedResult}
   \begin{verbatim}
   #  [1] 0.007715441
   \end{verbatim}
   \end{ShadedResult}

  \begin{enumerate}
  \def\labelenumii{\alph{enumii})}
  \setcounter{enumii}{4}
  \item
    Calculate the probability of each set of results enumerated in part
    c).

    \textcolor{blue}{The probabilities of observing 14, 15, or 16 cured individuals in the fecal infusion group, respectively, are $6.61 \times 10^{-4}$, $2.41 \times 10^{-5}$, and $2.51 \times 10^{-7}$. }
  \end{enumerate}

\begin{Shaded}
\begin{Highlighting}[]
\CommentTok{#probability of results more extreme than observed}
\KeywordTok{dhyper}\NormalTok{(}\DecValTok{14}\NormalTok{, }\DecValTok{17}\NormalTok{, }\DecValTok{29} \OperatorTok{-}\StringTok{ }\DecValTok{17}\NormalTok{, }\DecValTok{16}\NormalTok{)}
\end{Highlighting}
\end{Shaded}

  \begin{ShadedResult}
   \begin{verbatim}
   #  [1] 0.0006613235
   \end{verbatim}
   \end{ShadedResult}

\begin{Shaded}
\begin{Highlighting}[]
\KeywordTok{dhyper}\NormalTok{(}\DecValTok{15}\NormalTok{, }\DecValTok{17}\NormalTok{, }\DecValTok{29} \OperatorTok{-}\StringTok{ }\DecValTok{17}\NormalTok{, }\DecValTok{16}\NormalTok{)}
\end{Highlighting}
\end{Shaded}

  \begin{ShadedResult}
   \begin{verbatim}
   #  [1] 2.404813e-05
   \end{verbatim}
   \end{ShadedResult}

\begin{Shaded}
\begin{Highlighting}[]
\KeywordTok{dhyper}\NormalTok{(}\DecValTok{16}\NormalTok{, }\DecValTok{17}\NormalTok{, }\DecValTok{29} \OperatorTok{-}\StringTok{ }\DecValTok{17}\NormalTok{, }\DecValTok{16}\NormalTok{)}
\end{Highlighting}
\end{Shaded}

  \begin{ShadedResult}
   \begin{verbatim}
   #  [1] 2.505013e-07
   \end{verbatim}
   \end{ShadedResult}

  \begin{enumerate}
  \def\labelenumii{\alph{enumii})}
  \setcounter{enumii}{5}
  \item
    Based on the answers in parts d) and e), compute the one-sided
    \(p\)-value and interpret the results.

    \textcolor{blue}{The probability of observing results as or more extreme than the observed table is 0.0084. Since $p < 0.05$, there is sufficient evidence to reject the null hypothesis at significance level $\alpha = 0.05$; the data support fecal infusion as a more effective treatment for \textit{C. difficile} infection than vancomycin.}
  \end{enumerate}

\begin{Shaded}
\begin{Highlighting}[]
\CommentTok{#summing previous probabilities}
\NormalTok{p.observed =}\StringTok{ }\KeywordTok{dhyper}\NormalTok{(}\DecValTok{13}\NormalTok{, }\DecValTok{17}\NormalTok{, }\DecValTok{29} \OperatorTok{-}\StringTok{ }\DecValTok{17}\NormalTok{, }\DecValTok{16}\NormalTok{)}
\NormalTok{p.more.extreme =}\StringTok{ }\KeywordTok{dhyper}\NormalTok{(}\DecValTok{14}\NormalTok{, }\DecValTok{17}\NormalTok{, }\DecValTok{29} \OperatorTok{-}\StringTok{ }\DecValTok{17}\NormalTok{, }\DecValTok{16}\NormalTok{) }\OperatorTok{+}\StringTok{ }\KeywordTok{dhyper}\NormalTok{(}\DecValTok{15}\NormalTok{, }\DecValTok{17}\NormalTok{, }\DecValTok{29} \OperatorTok{-}\StringTok{ }\DecValTok{17}\NormalTok{, }\DecValTok{16}\NormalTok{) }\OperatorTok{+}\StringTok{ }
\StringTok{  }\KeywordTok{dhyper}\NormalTok{(}\DecValTok{16}\NormalTok{, }\DecValTok{17}\NormalTok{, }\DecValTok{29} \OperatorTok{-}\StringTok{ }\DecValTok{17}\NormalTok{, }\DecValTok{16}\NormalTok{)}
\NormalTok{p.observed }\OperatorTok{+}\StringTok{ }\NormalTok{p.more.extreme}
\end{Highlighting}
\end{Shaded}

  \begin{ShadedResult}
   \begin{verbatim}
   #  [1] 0.008401063
   \end{verbatim}
   \end{ShadedResult}

\begin{Shaded}
\begin{Highlighting}[]
\CommentTok{#using phyper}
\KeywordTok{phyper}\NormalTok{(}\DecValTok{12}\NormalTok{, }\DecValTok{17}\NormalTok{, }\DecValTok{29} \OperatorTok{-}\StringTok{ }\DecValTok{17}\NormalTok{, }\DecValTok{16}\NormalTok{, }\DataTypeTok{lower.tail =} \OtherTok{FALSE}\NormalTok{)}
\end{Highlighting}
\end{Shaded}

  \begin{ShadedResult}
   \begin{verbatim}
   #  [1] 0.008401063
   \end{verbatim}
   \end{ShadedResult}

  \begin{enumerate}
  \def\labelenumii{\alph{enumii})}
  \setcounter{enumii}{6}
  \item
    Use \texttt{fisher.test( )} to confirm the calculations in part f)
    and to calculate the two-sided \(p\)-value.

    \textcolor{blue}{The two-sided $p$-value is 0.0095.}
  \end{enumerate}

\begin{Shaded}
\begin{Highlighting}[]
\CommentTok{#one-sided p-value}
\KeywordTok{fisher.test}\NormalTok{(infusion.table, }\DataTypeTok{alternative =} \StringTok{"greater"}\NormalTok{)}\OperatorTok{$}\NormalTok{p.val}
\end{Highlighting}
\end{Shaded}

  \begin{ShadedResult}
   \begin{verbatim}
   #  [1] 0.008401063
   \end{verbatim}
   \end{ShadedResult}

\begin{Shaded}
\begin{Highlighting}[]
\CommentTok{#two-sided p-value}
\KeywordTok{fisher.test}\NormalTok{(infusion.table, }\DataTypeTok{alternative =} \StringTok{"two.sided"}\NormalTok{)}\OperatorTok{$}\NormalTok{p.val}
\end{Highlighting}
\end{Shaded}

  \begin{ShadedResult}
   \begin{verbatim}
   #  [1] 0.009530323
   \end{verbatim}
   \end{ShadedResult}
\item
  Psychologists conducted an experiment to investigate the effect of
  anxiety on a person's desire to be alone or in the company of others
  (Schacter 1959; Lehmann 1975). A group of 30 individuals were randomly
  assigned into two groups; one group was designated the ``high
  anxiety'' group and the other the ``low anxiety'' group. Those in the
  high-anxiety group were told that in the ``upcoming experiment'', they
  would be subjected to painful electric shocks, while those in the
  low-anxiety group were told that the shocks would be mild and
  painless.\footnote{Individuals were not actually subjected to electric shocks of any kind}.
  All individuals were informed that there would be a 10 minute wait
  before the experiment began, and that they could choose whether to
  wait alone or with other participants.

  The following table summarizes the results:

  \begin{table}[h]
   \centering
   \begin{tabular}{rrr|r}
       \hline
       & Wait Together & Wait Alone & Sum \\ 
       \hline
       High-Anxiety & 12 & 5 & 17 \\ 
       Low-Anxiety & 4 & 9 & 13 \\ 
       \hline
       Sum & 16 & 14 & 30 \\ 
       \hline
   \end{tabular}
   \end{table}

  \begin{enumerate}
  \def\labelenumii{\alph{enumii})}
  \item
    Under the null hypothesis of no association, what are the expected
    cell counts?

    \textcolor{blue}{Under the null hypothesis of no association, the expected cell counts are 9.07 and 7.93 in the wait together and wait alone groups, respectively, for those considered "high anxiety" and 6.93 and 6.07 in the wait together and wait alone groups, respectively, for those considered "low anxiety".}
  \end{enumerate}

\begin{Shaded}
\begin{Highlighting}[]
\CommentTok{#enter the data}
\NormalTok{anxiety.table =}\StringTok{ }\KeywordTok{matrix}\NormalTok{(}\KeywordTok{c}\NormalTok{(}\DecValTok{12}\NormalTok{, }\DecValTok{5}\NormalTok{, }\DecValTok{4}\NormalTok{, }\DecValTok{9}\NormalTok{), }\DataTypeTok{nrow =} \DecValTok{2}\NormalTok{, }\DataTypeTok{ncol =} \DecValTok{2}\NormalTok{, }\DataTypeTok{byrow =}\NormalTok{ T)}
\KeywordTok{dimnames}\NormalTok{(anxiety.table) =}\StringTok{ }\KeywordTok{list}\NormalTok{(}\StringTok{"Treatment"}\NormalTok{ =}\StringTok{ }\KeywordTok{c}\NormalTok{(}\StringTok{"High Anxiety"}\NormalTok{, }\StringTok{"Low Anxiety"}\NormalTok{),}
                           \StringTok{"Outcome"}\NormalTok{ =}\StringTok{ }\KeywordTok{c}\NormalTok{(}\StringTok{"Wait Together"}\NormalTok{, }\StringTok{"Wait Alone"}\NormalTok{))}

\KeywordTok{chisq.test}\NormalTok{(anxiety.table)}\OperatorTok{$}\NormalTok{expected}
\end{Highlighting}
\end{Shaded}

  \begin{ShadedResult}
   \begin{verbatim}
   #                Outcome
   #  Treatment      Wait Together Wait Alone
   #    High Anxiety      9.066667   7.933333
   #    Low Anxiety       6.933333   6.066667
   \end{verbatim}
   \end{ShadedResult}

  \begin{enumerate}
  \def\labelenumii{\alph{enumii})}
  \setcounter{enumii}{1}
  \item
    Under the assumption that the marginal totals are fixed and the null
    hypothesis is true, what is the probability of the observed set of
    results?

    \textcolor{blue}{Use the hypergeometric distribution with parameters $N = 30$, $m = 16$, and $n = 17$; calculate $P(X = 12)$. Consider the "successes" to be the individuals who wait together, and the "number sampled" to be the people randomized to the high-anxiety group. The probability of the observed set of results, assuming the marginal totals are fixed and the null hypothesis is true, is 0.0304.}
  \end{enumerate}

\begin{Shaded}
\begin{Highlighting}[]
\KeywordTok{dhyper}\NormalTok{(}\DecValTok{12}\NormalTok{, }\DecValTok{16}\NormalTok{, }\DecValTok{30} \OperatorTok{-}\StringTok{ }\DecValTok{16}\NormalTok{, }\DecValTok{17}\NormalTok{)}
\end{Highlighting}
\end{Shaded}

  \begin{ShadedResult}
   \begin{verbatim}
   #  [1] 0.03042455
   \end{verbatim}
   \end{ShadedResult}

  \begin{enumerate}
  \def\labelenumii{\alph{enumii})}
  \setcounter{enumii}{2}
  \item
    Enumerate the tables that are more extreme than what was observed,
    in the same direction.

    \textcolor{blue}{More individuals than expected in the high-anxiety group were observed to wait together; thus, tables that are more extreme in the same direction also consist of those where more people in the high-anxiety group wait together than observed. These are tables in which 13, 14, 15, or 16 individuals in the high-anxiety group wait together.}

    \begin{table}[h]
     \centering
     \color{gray}
     \begin{tabular}{r|cc|c}
     \hline
     & Wait Together & Wait Alone & Sum \\ 
     \hline
     High-Anxiety & \textcolor{black}{13} & \textcolor{black}{4} & 17 \\ 
     Low-Anxiety & \textcolor{black}{3} & \textcolor{black}{10} & 13 \\ 
     \hline
     Sum & 16 & 14 & 30 \\ 
     \hline
     \end{tabular}
     \end{table}

    \begin{table}[h!]
     \centering
     \color{gray}
     \begin{tabular}{r|cc|c}
     \hline
     & Wait Together & Wait Alone & Sum \\ 
     \hline
     High-Anxiety & \textcolor{black}{14} & \textcolor{black}{3} & 17 \\ 
     Low-Anxiety & \textcolor{black}{2} & \textcolor{black}{11} & 13 \\ 
     \hline
     Sum & 16 & 14 & 30 \\ 
     \hline
     \end{tabular}
     \end{table}

    \begin{table}[h!]
     \centering
     \color{gray}
     \begin{tabular}{r|cc|c}
     \hline
     & Wait Together & Wait Alone & Sum \\ 
     \hline
     High-Anxiety & \textcolor{black}{15} & \textcolor{black}{2} & 17 \\ 
     Low-Anxiety & \textcolor{black}{1} & \textcolor{black}{12} & 13 \\ 
     \hline
     Sum & 16 & 14 & 30 \\ 
     \hline
     \end{tabular}
     \end{table}

    \begin{table}[h!]
     \centering
     \color{gray}
     \begin{tabular}{r|cc|c}
     \hline
     & Wait Together & Wait Alone & Sum \\ 
     \hline
     High-Anxiety & \textcolor{black}{16} & \textcolor{black}{1} & 17 \\ 
     Low-Anxiety & \textcolor{black}{0} & \textcolor{black}{13} & 13 \\ 
     \hline
     Sum & 16 & 14 & 30 \\ 
     \hline
     \end{tabular}
     \end{table}
  \end{enumerate}

  \newpage

  \begin{enumerate}
  \def\labelenumii{\alph{enumii})}
  \setcounter{enumii}{3}
  \item
    Conduct a formal test of association for the results and summarize
    your findings. Let \(\alpha = 0.05\).

    \textcolor{blue}{Let $p_1$ represent the population proportion of individuals waiting together in the high-anxiety group and $p_2$ represent the population proportion of individuals waiting together in the low-anxiety group. Test $H_0: p_1 = p_2$ against $H_A: p_1 \neq p_2$. Let $\alpha = 0.05$. The two-sided $p$-value is 0.063. There is insufficient evidence to reject the null hypothesis; the data do not suggest there is an association between high anxiety and a person's desire to be in the company of others.}

    \textcolor{blue}{Note that the results are borderline; a one-sided $p$-value rejects the null hypothesis with $p = 0.036$. The choice of two-sided hypothesis is more impartial.}
  \end{enumerate}

\begin{Shaded}
\begin{Highlighting}[]
\CommentTok{#two-sided test}
\KeywordTok{fisher.test}\NormalTok{(anxiety.table)}\OperatorTok{$}\NormalTok{p.val}
\end{Highlighting}
\end{Shaded}

  \begin{ShadedResult}
   \begin{verbatim}
   #  [1] 0.06335838
   \end{verbatim}
   \end{ShadedResult}

\begin{Shaded}
\begin{Highlighting}[]
\CommentTok{#one-sided test}
\KeywordTok{fisher.test}\NormalTok{(anxiety.table, }\DataTypeTok{alternative =} \StringTok{"greater"}\NormalTok{)}\OperatorTok{$}\NormalTok{p.val}
\end{Highlighting}
\end{Shaded}

  \begin{ShadedResult}
   \begin{verbatim}
   #  [1] 0.03548226
   \end{verbatim}
   \end{ShadedResult}
\end{enumerate}

\newpage

\hypertarget{measures-of-association-in-two-by-two-tables}{%
\section{Measures of association in two-by-two
tables}\label{measures-of-association-in-two-by-two-tables}}

\begin{enumerate}
\def\labelenumi{\arabic{enumi}.}
\setcounter{enumi}{6}
\item
  Suppose a study is conducted to assess the association between smoking
  and cardiovascular disease (CVD). Researchers recruited a group of 231
  study participants then categorized them according to smoking and
  disease status: 111 are smokers, while 40 smokers and 32 non-smokers
  have CVD. Calculate and interpret the relative risk of CVD.

  \textcolor{blue}{The relative risk of CVD comparing smokers to non-smokers is 1.35. Smoking is associated with a 35\% increase in the probability of CVD. In other words, the risk of CVD is 35\% greater in smokers compared to non-smokers. }

\begin{Shaded}
\begin{Highlighting}[]
\CommentTok{#use r as a calculator}
\NormalTok{risk.smokers =}\StringTok{ }\DecValTok{40}\OperatorTok{/}\DecValTok{111}
\NormalTok{risk.nonsmokers =}\StringTok{ }\DecValTok{32}\OperatorTok{/}\NormalTok{(}\DecValTok{231-111}\NormalTok{)}

\NormalTok{risk.smokers }\OperatorTok{/}\StringTok{ }\NormalTok{risk.nonsmokers}
\end{Highlighting}
\end{Shaded}

  \begin{ShadedResult}
   \begin{verbatim}
   #  [1] 1.351351
   \end{verbatim}
   \end{ShadedResult}
\item
  Suppose another study is conducted to assess the association between
  smoking and CVD, but researchers use a different design: 90
  individuals with CVD and 110 individuals without CVD are recruited. 40
  of the individuals with CVD are smokers, and 80 of the individuals
  without CVD are non-smokers.

  \begin{enumerate}
  \def\labelenumii{\alph{enumii})}
  \item
    Is relative risk an appropriate measure of association for these
    data? Explain your answer.

    \textcolor{blue}{No, relative risk should not be calculated for these observations. Since the number of individuals with and without CVD is fixed by the study design, the proportion of individuals with CVD within a certain group (smokers or non-smokers) as calculated from the data is not a measure of CVD risk for that population.}
  \item
    Calculate the odds of CVD among smokers and the odds of CVD among
    non-smokers.

    \textcolor{blue}{Since there are 110 individuals without CVD and 80 of those are non-smokers, there are 30 individuals without CVD who smoke. Thus, there are $30 + 40 = 70$ individuals who smoke, and $(110 + 90) - 70 = 130$ individuals who do smoke. Of the 130 non-smokers, 80 do not have CVD; thus, 50 non-smokers have CVD.}

    \textcolor{blue}{The odds of CVD among smokers is the number of smokers with CVD divided by the number of smokers without CVD: $40/30 = 1.33$. The odds of CVD among non-smokers is the number of non-smokers with CVD divided by the number of non-smokers without CVD: $50/80 = 0.625$.}
  \end{enumerate}

\begin{Shaded}
\begin{Highlighting}[]
\CommentTok{#use r as a calculator}
\NormalTok{odds.smokers =}\StringTok{ }\DecValTok{40}\OperatorTok{/}\DecValTok{30}
\NormalTok{odds.nonsmokers =}\StringTok{ }\DecValTok{50}\OperatorTok{/}\DecValTok{80}

\NormalTok{odds.smokers}
\end{Highlighting}
\end{Shaded}

  \begin{ShadedResult}
   \begin{verbatim}
   #  [1] 1.333333
   \end{verbatim}
   \end{ShadedResult}

\begin{Shaded}
\begin{Highlighting}[]
\NormalTok{odds.nonsmokers}
\end{Highlighting}
\end{Shaded}

  \begin{ShadedResult}
   \begin{verbatim}
   #  [1] 0.625
   \end{verbatim}
   \end{ShadedResult}

  \begin{enumerate}
  \def\labelenumii{\alph{enumii})}
  \setcounter{enumii}{2}
  \item
    Calculate and interpret the odds ratio of CVD, comparing smokers to
    non-smokers.

    \textcolor{blue}{The odds ratio of CVD, comparing smokers to non-smokers is 2.13. The odds of CVD in smokers are approximately twice as large as the odds of CVD in smokers. The data suggest that smoking is associated with CVD.}
  \end{enumerate}

\begin{Shaded}
\begin{Highlighting}[]
\CommentTok{#use r as a calculator}
\NormalTok{odds.smokers }\OperatorTok{/}\StringTok{ }\NormalTok{odds.nonsmokers}
\end{Highlighting}
\end{Shaded}

  \begin{ShadedResult}
   \begin{verbatim}
   #  [1] 2.133333
   \end{verbatim}
   \end{ShadedResult}

  \begin{enumerate}
  \def\labelenumii{\alph{enumii})}
  \setcounter{enumii}{3}
  \item
    What would an odds ratio of CVD (comparing smokers to non-smokers)
    equal to 1 represent, in terms of the association between smoking
    and CVD? What would an odds ratio of CVD less than 1 represent?

    \textcolor{blue}{An odds ratio equal to 1 would represent no association between smoking and CVD. An odds ratio less than 1 would represent an association between not smoking and CVD; i.e., that the odds of CVD in non-smokers were higher than the odds of CVD in smokers.}
  \end{enumerate}
\end{enumerate}

%\showmatmethods


\bibliography{pinp}
\bibliographystyle{jss}



\end{document}

