\documentclass[letterpaper,11pt,twoside,]{pinp}

%% Some pieces required from the pandoc template
\providecommand{\tightlist}{%
  \setlength{\itemsep}{0pt}\setlength{\parskip}{0pt}}

% Use the lineno option to display guide line numbers if required.
% Note that the use of elements such as single-column equations
% may affect the guide line number alignment.

\usepackage[T1]{fontenc}
\usepackage[utf8]{inputenc}

% pinp change: the geometry package layout settings need to be set here, not in pinp.cls
\geometry{layoutsize={0.95588\paperwidth,0.98864\paperheight},%
  layouthoffset=0.02206\paperwidth, layoutvoffset=0.00568\paperheight}

\definecolor{pinpblue}{HTML}{185FAF}  % imagecolorpicker on blue for new R logo
\definecolor{pnasbluetext}{RGB}{101,0,0} %


\usepackage{booktabs}
\usepackage{longtable}
\usepackage{array}
\usepackage{multirow}
\usepackage{wrapfig}
\usepackage{float}
\usepackage{colortbl}
\usepackage{pdflscape}
\usepackage{tabu}
\usepackage{threeparttable}
\usepackage{threeparttablex}
\usepackage[normalem]{ulem}
\usepackage{makecell}

\title{Lab 006 - Inference for rates.}

\author[a]{EPIB607 - Inferential Statistics}

  \affil[a]{Fall 2020, McGill University}

\setcounter{secnumdepth}{5}

% Please give the surname of the lead author for the running footer
\leadauthor{Bhatnagar}

% Keywords are not mandatory, but authors are strongly encouraged to provide them. If provided, please include two to five keywords, separated by the pipe symbol, e.g:
 \keywords{  Sampling distribution |  Standard error |  Normal distribution |  Quantiles |  Percentiles |  Z-scores  }  

\begin{abstract}
In this exercise you will practice doing inference for one-sample rates.
\end{abstract}

\dates{This version was compiled on \today} 

% initially we use doi so keep for backwards compatibility
% new name is doi_footer

\pinpfootercontents{in-class exercise on rates}

\begin{document}

% Optional adjustment to line up main text (after abstract) of first page with line numbers, when using both lineno and twocolumn options.
% You should only change this length when you've finalised the article contents.
\verticaladjustment{-2pt}

\maketitle
\thispagestyle{firststyle}
\ifthenelse{\boolean{shortarticle}}{\ifthenelse{\boolean{singlecolumn}}{\abscontentformatted}{\abscontent}}{}

% If your first paragraph (i.e. with the \dropcap) contains a list environment (quote, quotation, theorem, definition, enumerate, itemize...), the line after the list may have some extra indentation. If this is the case, add \parshape=0 to the end of the list environment.


\begin{table}[h]
\begin{center}
\begin{tabular}{|r|rr|rr|rr|}
\hline
$y$ & \multicolumn{2}{c}{95\%} & \multicolumn{2}{c}{90\%} & \multicolumn{2}{c}{80\%} \\
\hline
0 & 0.00 & 3.69 & 0.00 & 3.00 & 0.00 & 2.30 \\
1 & 0.03 & 5.57 & 0.05 & 4.74 & 0.11 & 3.89 \\
2 & 0.24 & 7.22 & 0.36 & 6.30 & 0.53 & 5.32 \\
3 & 0.62 & 8.77 & 0.82 & 7.75 & 1.10 & 6.68 \\
4 & 1.09 & 10.24 & 1.37 & 9.15 & 1.74 & 7.99 \\
& & & & & &  \\
5 & 1.62 & 11.67 & 1.97 & 10.51 & 2.43 & 9.27 \\
6 & 2.20 & 13.06 & 2.61 & 11.84 & 3.15 & 10.53 \\
7 & 2.81 & 14.42 & 3.29 & 13.15 & 3.89 & 11.77 \\
8 & 3.45 & 15.76 & 3.98 & 14.43 & 4.66 & 12.99 \\
9 & 4.12 & 17.08 & 4.70 & 15.71 & 5.43 & 14.21 \\
& & & & & &  \\
10 & 4.80 & 18.39 & 5.43 & 16.96 & 6.22 & 15.41 \\
11 & 5.49 & 19.68 & 6.17 & 18.21 & 7.02 & 16.60 \\
12 & 6.20 & 20.96 & 6.92 & 19.44 & 7.83 & 17.78 \\
13 & 6.92 & 22.23 & 7.69 & 20.67 & 8.65 & 18.96 \\
14 & 7.65 & 23.49 & 8.46 & 21.89 & 9.47 & 20.13 \\
& & & & & &  \\
15 & 8.40 & 24.74 & 9.25 & 23.10 & 10.30 & 21.29 \\
16 & 9.15 & 25.98 & 10.04 & 24.30 & 11.14 & 22.45 \\
17 & 9.90 & 27.22 & 10.83 & 25.50 & 11.98 & 23.61 \\
18 & 10.67 & 28.45 & 11.63 & 26.69 & 12.82 & 24.76 \\
19 & 11.44 & 29.67 & 12.44 & 27.88 & 13.67 & 25.90 \\
& & & & & &  \\
20 & 12.22 & 30.89 & 13.25 & 29.06 & 14.53 & 27.05 \\
21 & 13.00 & 32.10 & 14.07 & 30.24 & 15.38 & 28.18 \\
22 & 13.79 & 33.31 & 14.89 & 31.41 & 16.24 & 29.32 \\
23 & 14.58 & 34.51 & 15.72 & 32.59 & 17.11 & 30.45 \\
24 & 15.38 & 35.71 & 16.55 & 33.75 & 17.97 & 31.58 \\
30 & 20.24 & 42.83 & 21.59 & 40.69 & 23.23 & 38.32\\
32 & 21.89 & 45.17 & 23.3 & 42.98 & 25 & 40.54\\
33 & 22.72 & 46.34 & 24.15 & 44.13 & 25.89 & 41.65\\
36 & 25.21 & 49.84 & 26.73 & 47.54 & 28.56 & 44.98\\
\hline
\end{tabular}
\end{center}
\caption{Poisson based Confidence Intervals}
\end{table}

\hypertarget{vacancy-rates-in-the-us-supreme-court}{%
\section{Vacancy rates in the US Supreme
Court}\label{vacancy-rates-in-the-us-supreme-court}}

Refer to the article \emph{Updating a Classic: The Poisson Distribution
and the Supreme Court Revisited}.

\begin{enumerate}
\def\labelenumi{\alph{enumi})}
\item
  What is the probability that a sitting President will appoint 3
  judges? (Reflect on this given the current situation).
\item
  Reproduce Table 1, i.e., determine how the \textbf{Probability} and
  \textbf{Expected} columns were calculated.
\item
  Update Table 1 and calculate an updated vacancy rate (expressed as
  vacancies per year) for the period 1933-2020.
\item
  Based only on your updated point-estimate of the rate {[}i.e.~without
  using any information on the health of the current court{]}, what is
  the probability that the next US president (the one who takes office
  in 2021) will be able to appoint 0, 1, 2, 3 and more than 3 new judges
  if (s)he stays in office for (i) four and (ii) eight years.
\end{enumerate}

\hypertarget{seismicity-before-and-after-hydraulic-fracturing-in-the-horn-river-basin-northeast-british-columbia}{%
\section{Seismicity before and after hydraulic fracturing in the Horn
River Basin, northeast British
Columbia}\label{seismicity-before-and-after-hydraulic-fracturing-in-the-horn-river-basin-northeast-british-columbia}}

Consult the article \emph{Investigation of regional seismicity before
and after hydraulic fracturing in the Horn River Basin, northeast
British Columbia} by Farahbod et al.~(2014).

\begin{enumerate}
\def\labelenumi{\alph{enumi})}
\item
  Using just the year 2006 data in the first four columns of Table 2,
  calculate separate event rates for the periods when hydraulic
  fracturing (i) was and (ii) was not in operation. Express each rate as
  the number of events per year, and accompany it with a 95\% CI.
\item
  Repeat the calculations for the individual years 2007 to 2011.
\item
  Repeat the calculations for the entire period 2006 to 2011.
\item
  Store your results in a tidy \texttt{data.frame}.
\item
  Are you comfortable using Poisson-based CIs for the entire period 2006
  to 2011? i.e.~does it look like the rate in non-HF days is homogeneous
  over the 6 years, i.e., do the 6 CIs largely overlap each other?
\item
  Does it look like the rate in HF days is homogeneous over the 6 years,
  i.e., do the 6 CIs largely overlap each other?
\end{enumerate}

\hypertarget{de-novo-mutation-rates}{%
\section{De-novo Mutation Rates}\label{de-novo-mutation-rates}}

The
\href{https://www.cancer.gov/publications/dictionaries/genetics-dictionary}
{US NCI Dictionary of Genetics Terms} defines a \textit{de novo}
mutation as

\begin{quote}
a genetic alteration that is present for the first time in one family
member as a result of a variant (or mutation) in a germ cell (egg or
sperm) of one of the parents, or a variant that arises in the fertilized
egg itself during early embryogenesis.
\end{quote}

These point mutations differ from the chromosome-level errors that lead
to Down Syndrome, and so the influences of the two parental ages might
also be expected to differ.

The article \emph{Rate of de novo mutations and the importance of
father's age to disease risk} is a good introduction to the fast-moving
research on this. The written news item
\href{http://www.nature.com/news/fathers-bequeath-more-mutations-as-they-age-1.11247}{(Nature News, 2012)},
and the audio interview with the senior author
\href{https://www.nature.com/nature/podcast/index-2012-08-23.html}{(Stefansson, 2012)}
provide a good lay introduction. That study involved genetic material
from 78 father-mother-offspring trios. The focus was on single
nucleotide polymorphism mutations and limited to the autosomal
chromosomes 1-22, thereby excluding the sex chromosome 23. Using several
rigorous criteria, a total of 4,933 \textit{de novo} mutations, or an
average of 63.2 per trio, were called.

\begin{enumerate}
\def\labelenumi{\alph{enumi})}
\tightlist
\item
  In five trios, a child of the offspring was also sequenced, thereby
  allowing the investigators to determine the parent of origin of each
  of the 348 \textit{de novo} mutations called. Using Table 1, calculate
  separate mutation rates for mutations from mothers and from fathers.
  Express each rate as the number of mutations per parent-year-lived,
  and accompany it with a 95\% CI.
\end{enumerate}

%\showmatmethods


\bibliography{pinp}
\bibliographystyle{jss}



\end{document}

