\documentclass[letterpaper,12pt,twoside,]{pinp}

%% Some pieces required from the pandoc template
\providecommand{\tightlist}{%
  \setlength{\itemsep}{0pt}\setlength{\parskip}{0pt}}

% Use the lineno option to display guide line numbers if required.
% Note that the use of elements such as single-column equations
% may affect the guide line number alignment.

\usepackage[T1]{fontenc}
\usepackage[utf8]{inputenc}

% pinp change: the geometry package layout settings need to be set here, not in pinp.cls
\geometry{layoutsize={0.95588\paperwidth,0.98864\paperheight},%
  layouthoffset=0.02206\paperwidth, layoutvoffset=0.00568\paperheight}

\definecolor{pinpblue}{HTML}{185FAF}  % imagecolorpicker on blue for new R logo
\definecolor{pnasbluetext}{RGB}{101,0,0} %



\title{Lab 009 - Logistic Regression}

\author[a]{EPIB607 - Inferential Statistics}

  \affil[a]{McGill University}

\setcounter{secnumdepth}{5}

% Please give the surname of the lead author for the running footer
\leadauthor{Bhatnagar}

% Keywords are not mandatory, but authors are strongly encouraged to provide them. If provided, please include two to five keywords, separated by the pipe symbol, e.g:
 

\begin{abstract}

\end{abstract}

\dates{This version was compiled on \today} 

% initially we use doi so keep for backwards compatibility
% new name is doi_footer


\begin{document}

% Optional adjustment to line up main text (after abstract) of first page with line numbers, when using both lineno and twocolumn options.
% You should only change this length when you've finalised the article contents.
\verticaladjustment{-2pt}

\maketitle
\thispagestyle{firststyle}
\ifthenelse{\boolean{shortarticle}}{\ifthenelse{\boolean{singlecolumn}}{\abscontentformatted}{\abscontent}}{}

% If your first paragraph (i.e. with the \dropcap) contains a list environment (quote, quotation, theorem, definition, enumerate, itemize...), the line after the list may have some extra indentation. If this is the case, add \parshape=0 to the end of the list environment.


This lab introduces simple logistic regression, a model for the
association of a binary response variable with a single predictor
variable. Logistic regression generalizes methods for two-way tables and
allows for the use of a numerical predictor.

\hypertarget{introduction}{%
\section{Introduction}\label{introduction}}

\emph{Odds and probabilities}

If the probability of an event \(A\) is \(p\), the odds of the event are
\[\dfrac{p}{1 - p}. \]

Given the odds of an event \(A\), the probability of the event is
\[\dfrac{\text{odds}}{1 + \text{odds}}. \]

\emph{Simple logistic regression}

Suppose that \(Y\) is a binary response variable and \(X\) is a
predictor variable, where \(Y = 1\) represents the particular outcome of
interest.

The model for a single variable logistic regression, where
\(p(x) = P(Y = 1 | X = x)\), is
\[\text{log} \left[ \dfrac{p(x)}{1 - p(x)} \right] = \beta_0 + \beta_1 x.\]

The estimated model equation has the form
\[\text{log} \left[ \dfrac{\hat{p}(x)}{1 - \hat{p}(x)} \right] = b_0 + b_1 x, \]
where \(b_0\) and \(b_1\) are estimates of the model parameters
\(\beta_0\) and \(\beta_1\).

\newpage

\hypertarget{background-information}{%
\section{Background Information}\label{background-information}}

Patients admitted to an intensive care unit (ICU) are either extremely
ill or considered to be at great risk of serious complications. There
are no widely accepted criteria for distinguishing between patients who
should be admitted to an ICU and those for whom admission to other
hospital units would be more appropriate. Thus, among different ICUs,
there are wide ranges in a patient's chance of survival. When studies
are done to compare effectiveness of ICU care, it is critical to have a
reliable means of assessing the comparability of the different patient
populations.

One such strategy for doing so involves the use of statistical modeling
to relate empirical data for many patient variables to outcomes of
interest. The following dataset consists of a sample of 200 subjects who
were part of a much larger study on survival of patients following
admission to an adult
ICU.\footnote{From Hosmer D.W., Lemeshow, S., and Sturdivant, R.X. \textit{Applied Logistic Regression}. 3$^{rd}$ ed., 2013.}
The major goal of the study was to develop a logistic regression model
to predict the probability of survival to hospital
discharge.\footnote{Lemeshow S., et al. Predicting the outcome of intensive care unit patients. \textit{Journal of the American Statistical Association} 83.402 (1988): 348-356.}

The following table provides a list of the variables in the dataset and
their description. The data are accessible as the \texttt{icu} dataset
in the \texttt{aplore3} package.

\begin{center}
\begin{tabular}{r|l}
\textbf{Variable} & \textbf{Description} \\
\hline
\texttt{id} & patient ID number \\
\texttt{sta} & patient status at discharge, either \texttt{Lived} or \texttt{Died} \\
\texttt{age} & age in years (when admitted) \\
\texttt{gender} & gender, either \texttt{Male} or \texttt{Female} \\
\texttt{can} & cancer part of current issue, either \texttt{No} or \texttt{Yes} \\
\texttt{crn} & history of chronic renal failure, either \texttt{No} or \texttt{Yes}\\
\texttt{inf} & infection probable at admission, either \texttt{No} or \texttt{Yes} \\
\texttt{cpr} & CPR prior to admission, either \texttt{No} or \texttt{Yes} \\
\texttt{sys} & systolic blood pressure at admission, in mm Hg \\
\texttt{hra} & heart rate at admission, in beats per minute \\
\texttt{pre} & previous admission to an ICU within 6 months, either \texttt{No} or \texttt{Yes} \\
\texttt{type} & type of admission, either \texttt{Elective} or \texttt{Emergency} \\
\texttt{fra} & long bone, multiple, neck, single area, or hip fracture, either \texttt{No} or \texttt{Yes} \\
\texttt{po2} & $PO_2$ from initial blood gases, either \texttt{60} or \texttt{<=60}, in mm Hg\\
\texttt{ph} & $pH$ from initial blood gases, either \texttt{>= 7.25} or \texttt{< 7.25} \\
\texttt{pco} & $PCO_2$ from initial blood gases, either \texttt{<= 45} or \texttt{>45}, in mm Hg \\
\texttt{bic} & $HCO_3$ (bicarbonate) from initial blood gases, either \texttt{>= 18} or \texttt{<18}, in mm Hg \\
\texttt{cre} & creatinine from initial blood gases, either \texttt{<= 2.0} or \texttt{> 2.0}, in mg/dL \\
\texttt{loc} & level of consciousness at admission, either \texttt{Nothing}, \texttt{Stupor}, or \texttt{Coma}
\end{tabular}
\end{center}

\newpage

\hypertarget{odds-and-probabilities}{%
\section{Odds and probabilities}\label{odds-and-probabilities}}

\begin{enumerate}
\def\labelenumi{\arabic{enumi}.}
\item
  Create a two-way table of survival to discharge by whether CPR was
  administered prior to admission. The template provides code for
  re-leveling the \texttt{sta} variable such that \texttt{0} corresponds
  to \texttt{Died} and \texttt{1} corresponds to \texttt{Lived}.

  \begin{enumerate}
  \def\labelenumii{\alph{enumii})}
  \item
    Calculate the odds of survival to discharge for those who did not
    receive CPR prior to ICU admission. Is someone who did not receive
    CPR prior to admission more likely to survive to discharge than to
    not survive to discharge?
  \item
    Calculate the odds of survival to discharge for those who received
    CPR prior to ICU admission. Is someone who received CPR prior to
    admission more likely to survive to discharge than not?
  \item
    Calculate the odds ratio of survival to discharge, comparing
    patients who receive CPR prior to admission to those who do not
    receive CPR prior to admission.
  \end{enumerate}
\item
  Creatinine level in the data are recorded as being either less than or
  equal to 2.0 mg/dL or greater than 2.0 mg/dL. A typical creatinine
  level is between 0.5 - 1.0 mg/dL, and elevated creatinine may be a
  sign of renal failure.

  \begin{enumerate}
  \def\labelenumii{\alph{enumii})}
  \item
    Calculate the odds of survival to discharge for patients who have a
    creatinine level less than or equal to 2.0 mg/dL. From the odds,
    calculate the probability of survival to discharge for these
    patients.
  \item
    Calculate the probability of survival to discharge for patients who
    have a creatinine level greater than 2.0 mg/dL. From the
    probability, calculate the odds of survival to discharge for these
    patients.
  \item
    Compute and interpret the odds ratio of survival to discharge,
    comparing patients with creatinine \(> 2.0\) mg/dL to those with
    creatinine \(\leq\) 2.0 mg/dL.
  \end{enumerate}
\end{enumerate}

\hypertarget{simple-logistic-regression}{%
\section{Simple logistic regression}\label{simple-logistic-regression}}

\begin{enumerate}
\def\labelenumi{\arabic{enumi}.}
\setcounter{enumi}{2}
\item
  Fit a logistic regression model to predict survival to discharge from
  prior CPR.

  \begin{enumerate}
  \def\labelenumii{\alph{enumii})}
  \item
    Write the model equation estimated from the data.
  \item
    Interpret the intercept. Confirm that the interpretation coheres
    with the answer to Question 1, part a).
  \item
    Interpret the slope coefficient. Compute the exponential of the
    slope coefficient and confirm that this matches the answer to
    Question 1, part c).
  \item
    Compute and interpret an odds ratio that summarizes the association
    between survival to discharge and prior CPR.
  \item
    Is there evidence of a statistically significant association between
    survival to discharge and prior CPR at \(\alpha = 0.05\)?
  \end{enumerate}
\end{enumerate}

\newpage

\begin{enumerate}
\def\labelenumi{\arabic{enumi}.}
\setcounter{enumi}{3}
\item
  Fit a logistic regression model to predict survival to discharge from
  an indicator of elevated creatinine.

  \begin{enumerate}
  \def\labelenumii{\alph{enumii})}
  \item
    Write the model equation estimated from the data.
  \item
    Interpret the intercept and slope coefficient.
  \item
    Compute and interpret an odds ratio that summarizes the association
    between survival to discharge and an indicator of elevated
    creatinine.
  \item
    Is there evidence of a statistically significant association between
    survival to discharge and an indicator of elevated creatinine at
    \(\alpha = 0.05\)?
  \end{enumerate}
\item
  Fit a logistic regression model to predict survival to discharge from
  age.

  \begin{enumerate}
  \def\labelenumii{\alph{enumii})}
  \item
    Write the model equation estimated from the data.
  \item
    Does the intercept have a meaningful interpretation in the context
    of the data?
  \item
    Interpret the slope coefficient.
  \item
    Calculate the odds of survival to discharge for a 70-year-old
    individual. Is a 70-year-old individual more likely to survive than
    not?
  \item
    Calculate the odds ratio of survival to discharge comparing a
    45-year-old individual to a 70-year-old individual.
  \end{enumerate}
\end{enumerate}

%\showmatmethods


\bibliography{pinp}
\bibliographystyle{jss}



\end{document}
