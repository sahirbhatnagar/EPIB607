\documentclass[10pt]{beamer}\usepackage[]{graphicx}\usepackage[]{color}
% maxwidth is the original width if it is less than linewidth
% otherwise use linewidth (to make sure the graphics do not exceed the margin)
\makeatletter
\def\maxwidth{ %
  \ifdim\Gin@nat@width>\linewidth
    \linewidth
  \else
    \Gin@nat@width
  \fi
}
\makeatother

\definecolor{fgcolor}{rgb}{0.345, 0.345, 0.345}
\newcommand{\hlnum}[1]{\textcolor[rgb]{0.686,0.059,0.569}{#1}}%
\newcommand{\hlstr}[1]{\textcolor[rgb]{0.192,0.494,0.8}{#1}}%
\newcommand{\hlcom}[1]{\textcolor[rgb]{0.678,0.584,0.686}{\textit{#1}}}%
\newcommand{\hlopt}[1]{\textcolor[rgb]{0,0,0}{#1}}%
\newcommand{\hlstd}[1]{\textcolor[rgb]{0.345,0.345,0.345}{#1}}%
\newcommand{\hlkwa}[1]{\textcolor[rgb]{0.161,0.373,0.58}{\textbf{#1}}}%
\newcommand{\hlkwb}[1]{\textcolor[rgb]{0.69,0.353,0.396}{#1}}%
\newcommand{\hlkwc}[1]{\textcolor[rgb]{0.333,0.667,0.333}{#1}}%
\newcommand{\hlkwd}[1]{\textcolor[rgb]{0.737,0.353,0.396}{\textbf{#1}}}%
\let\hlipl\hlkwb

\usepackage{framed}
\makeatletter
\newenvironment{kframe}{%
 \def\at@end@of@kframe{}%
 \ifinner\ifhmode%
  \def\at@end@of@kframe{\end{minipage}}%
  \begin{minipage}{\columnwidth}%
 \fi\fi%
 \def\FrameCommand##1{\hskip\@totalleftmargin \hskip-\fboxsep
 \colorbox{shadecolor}{##1}\hskip-\fboxsep
     % There is no \\@totalrightmargin, so:
     \hskip-\linewidth \hskip-\@totalleftmargin \hskip\columnwidth}%
 \MakeFramed {\advance\hsize-\width
   \@totalleftmargin\z@ \linewidth\hsize
   \@setminipage}}%
 {\par\unskip\endMakeFramed%
 \at@end@of@kframe}
\makeatother

\definecolor{shadecolor}{rgb}{.97, .97, .97}
\definecolor{messagecolor}{rgb}{0, 0, 0}
\definecolor{warningcolor}{rgb}{1, 0, 1}
\definecolor{errorcolor}{rgb}{1, 0, 0}
\newenvironment{knitrout}{}{} % an empty environment to be redefined in TeX

\usepackage{alltt}


%\input{slides_header.tex}
\usepackage{graphicx}
\usepackage{hyperref, url}
\hypersetup{colorlinks,citecolor=myorange,filecolor=red,linkcolor=brown,urlcolor=blue}

\usepackage{longtable,booktabs}
\usepackage{amssymb,amsmath}
\usepackage{animate}
\usepackage{subfig}
\usepackage{tikz}
\usetikzlibrary{shapes.geometric, arrows,shapes.symbols,decorations.pathreplacing}
\tikzstyle{startstop} = [rectangle, rounded corners, minimum width=3cm, minimum height=1cm, draw=black, fill=pinkish,text width=3.5cm]
\tikzstyle{startstop2} = [rectangle, rounded corners, minimum width=3cm, minimum height=1cm, draw=black, fill=background,text width=4.5cm]
\tikzstyle{startstop3} = [rectangle, rounded corners, minimum width=3cm, minimum height=1cm, draw=black, fill=beige,text width=3.0cm]
\tikzstyle{startstop4} = [rectangle, rounded corners, minimum width=3cm, minimum height=1cm, draw=black, fill=pinkish,text width=4.5cm]
\tikzstyle{io} = [trapezium, trapezium left angle=70, trapezium right angle=110, minimum width=2cm, minimum height=1cm, text centered, draw=black, fill=blue!30,text width=1.5cm]
\tikzstyle{process} = [rectangle, minimum width=1cm, minimum height=1cm, text centered, draw=black, fill=orange!30,text width=2cm]
\tikzstyle{decision} = [diamond, minimum width=2cm, minimum height=1cm, text centered, draw=black, fill=green!30]
\tikzstyle{arrow} = [thick,->,>=stealth]
\tikzstyle{both} = [thick,<->,>=stealth, red]


% used for tree of stats tests in 001-introduction
\tikzstyle{startstopstats} = [rectangle, rounded corners, minimum width=2cm, minimum height=.5cm,text centered, draw=black, fill=red!30]
\tikzstyle{iostats} = [trapezium, trapezium left angle=70, trapezium right angle=110, minimum width=2cm, minimum height=.5cm, text centered, draw=black, fill=blue!30]
\tikzstyle{processstats} = [rectangle, minimum width=1.5cm, minimum height=.5cm, text centered, draw=black, fill=orange!30]
\tikzstyle{processbigstats} = [rectangle, minimum width=1.5cm, minimum height=.5cm, text centered, draw=black, fill=orange!30,text width=1.6cm]
\tikzstyle{decisionstats} = [rectangle, minimum width=1cm, minimum height=1cm, text centered, draw=black, fill=green!30,text width=1.6cm]
\tikzstyle{decisionbigstats} = [rectangle, minimum width=1cm, minimum height=1cm, text centered, draw=black, fill=yellow!30,text width=2cm]

\usepackage{pifont}% http://ctan.org/pkg/pifont
\newcommand{\cmark}{\ding{51}}%
\newcommand{\xmark}{\ding{55}}%

\usepackage{ulem} % for strikeout

\usepackage{xcolor}
\usepackage{color, colortbl}
\definecolor{lightgray}{RGB}{200,200,200}
\definecolor{palegray}{RGB}{221,221,221}
\definecolor{myblue}{RGB}{0,89,179}
\definecolor{myorange}{rgb}{0.776,0.357,0.157}
\definecolor{gray}{RGB}{110,110,110}
\definecolor{darkgray}{RGB}{100,100,100}
\definecolor{lightgray}{RGB}{200,200,200}
\definecolor{palegray}{RGB}{221,221,221}
\definecolor{turquoise}{RGB}{81,193,188}
\definecolor{tomato}{RGB}{255,136,136}
\definecolor{mandarina}{RGB}{229,169,25}
\definecolor{foreground}{RGB}{81,141,193}
\definecolor{background}{RGB}{246,244,240}
\definecolor{highlight}{RGB}{229,169,25}
\definecolor{lowlight}{RGB}{200,200,200}
\definecolor{beige}{RGB}{255,255,240}
\definecolor{pinkish}{RGB}{255,223,247}
\definecolor{darktangerine}{rgb}{1.0, 0.66, 0.07}
\definecolor{deepink}{RGB}{255,20,147}
%\usepackage{shadethm}
%\colorlet{shadecolor}{blue!15}
%\colorlet{shadecolor}{palegray}
%\setlength{\shadeboxrule}{.4pt}

%\newshadetheorem{thm}{Theorem}
%\newshadetheorem{defm}{Definition}
%\newshadetheorem{exm}{Exercise}
%\newshadetheorem{remarkm}{Remark}
%\definecolor{shadethmcolor}{HTML}{EDF8FF}
%\definecolor{shadethmcolor}{RGB}{221,221,221}
%\definecolor{shaderulecolor}{HTML}{45CFFF}
%\definecolor{shaderulecolor}{RGB}{0,89,179}
%\setlength{\shadeboxrule}{.4pt}



\usepackage{epsfig}

\newcommand{\code}[1]{\texttt{#1}}
\newcommand{\blue}[1]{\textcolor{blue}{#1}}
\newcommand{\red}[1]{\textcolor{red}{#1}}

\usepackage{comment}

\makeatletter

\def \iqsssectiontitleheader {}

\newcommand{\iqsssectiontitle}[1]{
	\def \iqsssectiontitleheader{#1}
}

\@ifundefined{insertmainframenumber}
{%
	% \insertmainframenumber not defined
	\newcommand{\insertmainframenumber}{\inserttotalframenumber}
}
{%
	% \insertmainframenumber already defined
}%


\AtBeginSection[]{
	\title{\insertsectionhead}
	{
		%\definecolor{white}{RGB}{140,193,250}
		%\definecolor{white}{RGB}{200,200,200}
		%\definecolor{white}{RGB}{242,244,247}
		\definecolor{white}{RGB}{0,89,179}
		%\definecolor{iqss@orange}{rgb}{1,1,1}
		\ifnum \insertmainframenumber > \insertframenumber
		%\setbeamercolor{background canvas}{bg=myblue}
		%\setbeamercolor{normal text}{fg=black,bg=white}
		%\setbeamercolor{frametitle}{fg=red}
		%\setbeamercolor{section in toc}{fg=myblue, bg=white}
		%\setbeamercolor{subsection in toc}{fg=myblue, bg=white}
		\frame{
			\frametitle{\iqsssectiontitleheader}
			\tableofcontents[currentsection]
		}
		\else
		\frame{
			\frametitle{Backup Slides}
			\tableofcontents[sectionstyle=shaded/shaded,subsectionstyle=shaded/shaded/shaded]
		}
		\fi
	}
}
\makeatother
%\graphicspath{{/home/sahir/git_repositories/EPIB607/resources/assets/slides/figure/}}


\usepackage{fontspec}
%\setsansfont{Fira Sans}
%\setmonofont{Fira Mono}
%\setsansfont[ItalicFont={Fira Sans Light Italic},BoldFont={Fira Sans},BoldItalicFont={Fira Sans Italic}]{Fira Sans Light}
%\setmonofont[BoldFont={Fira Mono Medium}]{Fira Mono}

\def\installpath{/usr/local/share/texmf/fonts/opentype/libertinus/}
\setmainfont{LibertinusSerif}[
UprightFont    = *-Regular,
BoldFont       = *-Bold,
ItalicFont     = *-Italic,
BoldItalicFont = *-BoldItalic,
Ligatures      = TeX,
Extension      = .otf,
Path           = \installpath/
]

\setsansfont{LibertinusSerif}[
UprightFont    = *-Regular,
BoldFont       = *-Bold,
ItalicFont     = *-Italic,
BoldItalicFont = *-BoldItalic,
Ligatures      = TeX,
Extension      = .otf,
Path           = \installpath/
]


%\setmonofont{LibertinusSerif}[
%UprightFont    = *-Regular,
%BoldFont       = *-Bold,
%ItalicFont     = *-Italic,
%BoldItalicFont = *-BoldItalic,
%Ligatures      = TeX,
%Extension      = .otf,
%Path           = \installpath/
%]






\newcommand\Wider[2][3em]{%
	\makebox[\linewidth][c]{%
		\begin{minipage}{\dimexpr\textwidth+#1\relax}
			\raggedright#2
		\end{minipage}%
	}%
}


\newcommand {\framedgraphic}[1] {
	\begin{figure}
		\centering
		\includegraphics[width=\textwidth,height=0.9\textheight,keepaspectratio]{#1}
	\end{figure}
}


\newcommand {\framedgraphiccaption}[2] {
	\begin{figure}
		\centering
		\includegraphics[width=\textwidth,height=0.8\textheight,keepaspectratio]{#1}
		\caption{#2}
	\end{figure}
}




\setbeamercolor{itemize item}{fg=myblue}
\setbeamercolor{itemize subitem}{fg=myorange}
%\setbeamertemplate{itemize item}[square]
\setbeamertemplate{itemize item}[circle]
\setbeamertemplate{itemize subitem}[triangle]
\setbeamertemplate{blocks}[rounded][shadow=true]
\setbeamercolor{block body alerted}{bg=alerted text.fg!10}
\setbeamercolor{block title alerted}{bg=alerted text.fg!20}
\setbeamercolor{block body}{bg=structure!10}
\setbeamercolor{block title}{bg=structure!20}
\setbeamercolor{block body example}{bg=green!10}
\setbeamercolor{block title example}{bg=green!20}


\makeatletter
\newenvironment<>{proofs}[1][\proofname]{%
	\par
	\def\insertproofname{#1\@addpunct{.}}%
	\usebeamertemplate{proof begin}#2}
{\usebeamertemplate{proof end}}
\newenvironment<>{proofc}{%
	\setbeamertemplate{proof begin}{\begin{block}{}}
		\par
		\usebeamertemplate{proof begin}}
	{\usebeamertemplate{proof end}}
	\newenvironment<>{proofe}{%
		\par
		\pushQED{\qed}
		\setbeamertemplate{proof begin}{\begin{block}{}}
			\usebeamertemplate{proof begin}}
		{\popQED\usebeamertemplate{proof end}}
\makeatother


\makeatletter
\newenvironment<>{exams}[1][\proofname]{%
	\par
	\def\insertproofname{#1\@addpunct{.}}%
	\usebeamertemplate{example begin}#2}
{\usebeamertemplate{example end}}
\newenvironment<>{examc}{%
	\setbeamertemplate{exam begin}{\begin{block}{}}
		\par
		\usebeamertemplate{exam begin}}
	{\usebeamertemplate{exam end}}
	\newenvironment<>{exame}{%
		\par
		\pushQED{\qed}
		\setbeamertemplate{exam begin}{\begin{block}{}}
			\usebeamertemplate{exam begin}}
		{\popQED\usebeamertemplate{exam end}}
		\makeatother

%\definecolor{mycolor}{HTML}{F7F8E0}
%\declaretheorem[shaded={bgcolor=mandarina}]{theo}
%\declaretheorem[shaded={bgcolor=mycolor}]{propo}
%\declaretheorem[shaded={bgcolor=green!80!black!30}]{remark}

%\setbeamertemplate{navigation symbols}{\usebeamercolor[fg]{title in head/foot}\usebeamerfont{title in head/foot}\insertframenumber}


%\setbeamertemplate{footline}{}

\beamertemplatenavigationsymbolsempty % toggle off if you want navigation symbols at the bottom

\setbeamertemplate{footline}
{ \usebeamercolor[fg]{page number in head/foot}%
	\usebeamerfont{page number in head/foot}%
	\hspace{1em}\insertsectionhead%
	\hfill%
	\insertframenumber\,/\,\hyperlinkappendixstart{\insertmainframenumber}
	\ifnum \thepage = \insertframeendpage{\small .}\else{\phantom{\small .}}\fi
	\hspace{1em}
	\vskip2pt%
}

%\newtheorem{proposition}[theorem]{Proposition}
%\newtheorem{exercise}[theorem]{Exercise}
%\newtheorem{remark}[theorem]{Remark}


\usepackage{amsthm}
\usepackage{thmtools}

\setbeamertemplate{theorems}[ams style] 
%\setbeamertemplate{theorems}[numbered] 
%\setbeamertemplate{corollary}[numbered] 
\newtheorem{proposition}{Proposition}
\newtheorem{exercise}{Exercise}
\newtheorem{remark}{Remark}
\newtheorem{exam}{Example}
%\newtheorem{proof}{Proof}
%\newtheorem{corollaries}[theorem]{Corollary}
\newcommand*{\theorembreak}{\usebeamertemplate{theorem end}\framebreak\usebeamertemplate{theorem begin}}



\setlength{\emergencystretch}{3em} % prevent overfull lines
\providecommand{\tightlist}{%
	\setlength{\itemsep}{0pt}\setlength{\parskip}{0pt}}

\newcommand\AddButton{%
	\setbeamertemplate{background canvas}{%
		\begin{tikzpicture}[remember picture,overlay]
		\node[anchor=west] at ([yshift=5pt,xshift=0.1em]current page.south west)
		{\hyperlink{toc}{\beamergotobutton{back to TOC}}};
		\end{tikzpicture}%
	}%
}


\titlegraphic{\hfill\includegraphics[height=1cm]{/home/sahir/git_repositories/EPIB607/slides/mcgill_logo.png}}




\graphicspath{{/home/sahir/git_repositories/EPIB607/slides/figure/}}

%\let\oldShaded\Shaded
%\let\endoldShaded\endShaded
%\renewenvironment{Shaded}{\footnotesize\oldShaded}{\endoldShaded}

%\newcommand{\blue}[1]{\textcolor{blue}{#1}}
%\newcommand{\red}[1]{\textcolor{red}{#1}}


\usepackage{xparse}
\NewDocumentCommand\mylist{>{\SplitList{;}}m}
{
	\begin{itemize}
		\ProcessList{#1}{ \insertitem }
	\end{itemize}
}
\NewDocumentCommand\mynum{>{\SplitList{;}}m}
{
	\begin{enumerate}
		\ProcessList{#1}{ \insertitem }
	\end{enumerate}
}
\newcommand\insertitem[1]{\item #1}

\newcommand\FrameText[1]{%
	\begin{textblock*}{\paperwidth}(0pt,\textheight)
		\raggedright #1\hspace{.5em}
\end{textblock*}}
\IfFileExists{upquote.sty}{\usepackage{upquote}}{}
\begin{document}
	
	
	
	%\title{Introduction to Regression Trees}
	%\author{Sahir Bhatnagar \inst{1}}
	%\author[shortname]{Sahir Rai Bhatnagar, PhD Candidate (Biostatistics) }
	%\institute[shortinst]{Department of Epidemiology, Biostatistics and Occupational Health}
	
	\title{019 - Poisson Regression}
	\author{EPIB 607 - FALL 2020}
	\institute{
		Sahir Rai Bhatnagar\\
		Department of Epidemiology, Biostatistics, and Occupational Health\\
		McGill University\\
		
		\vspace{0.1 in}
		
		\texttt{sahir.bhatnagar@mcgill.ca}\\
		%\texttt{\url{https://sahirbhatnagar.com/EPIB607/}}
	}
	
	\date{slides compiled on \today}
	
	\maketitle
	
	%\section{Objectives}
	




\section{1. Breastfeeding and respiratory infection I}
\begin{frame}[fragile,plain]
\vspace*{-.91in}
%\textbf{3. Ratio depth of ocean depths in north vs south hemisphere}
\tiny
A total of 189,612 person-years of follow up were accumulated over the course of the study: 151,690
among infants who were being breastfed and 37,922 among infants not being breastfed. Over the
course of follow up the investigators identified 514,230 incident cases of respiratory infection among
breastfeeding infants and 140,312 among non-breastfeeding infants. Calculate the crude incidence rate difference and 95\% CI comparing infants who were not breastfed with those who were.
\begin{knitrout}\tiny
\definecolor{shadecolor}{rgb}{0.969, 0.969, 0.969}\color{fgcolor}\begin{kframe}
\begin{alltt}
\hlstd{fit} \hlkwb{<-} \hlkwd{glm}\hlstd{(cases} \hlopt{~ -}\hlnum{1} \hlopt{+} \hlstd{PT} \hlopt{+} \hlstd{PT}\hlopt{:}\hlstd{not_breastfed,} \hlkwc{family} \hlstd{=} \hlkwd{poisson}\hlstd{(}\hlkwc{link} \hlstd{= identity))}
\hlkwd{print}\hlstd{(}\hlkwd{summary}\hlstd{(fit),} \hlkwc{signif.stars} \hlstd{= F)}
\end{alltt}
\begin{verbatim}
## 
## Coefficients:
##                  Estimate Std. Error z value Pr(>|z|)
## PT                3.39001    0.00473   717.1   <2e-16
## PT:not_breastfed  0.31001    0.01095    28.3   <2e-16
## 
## (Dispersion parameter for poisson family taken to be 1)
## 
##     Null deviance:        Inf  on 2  degrees of freedom
## Residual deviance: 1.1195e-10  on 0  degrees of freedom
## AIC: 32.68
## 
## Number of Fisher Scoring iterations: 2
\end{verbatim}
\end{kframe}
\end{knitrout}

\end{frame}


\section{2. Breastfeeding and respiratory infection II}
\begin{frame}[fragile,plain]
	\vspace*{-1.291in}
	%\textbf{3. Ratio depth of ocean depths in north vs south hemisphere}
	\tiny
Calculate the crude incidence rate ratio and 95\% CI comparing infants who were not breastfed with those who were.
\begin{knitrout}\tiny
\definecolor{shadecolor}{rgb}{0.969, 0.969, 0.969}\color{fgcolor}\begin{kframe}
\begin{verbatim}
## 
## Coefficients:
##               Estimate Std. Error z value Pr(>|z|)
## (Intercept)    1.22083    0.00139   875.5   <2e-16
## not_breastfed  0.08751    0.00301    29.1   <2e-16
## 
## (Dispersion parameter for poisson family taken to be 1)
## 
##     Null deviance: 8.3002e+02  on 1  degrees of freedom
## Residual deviance: 1.1533e-10  on 0  degrees of freedom
## AIC: 32.68
## 
## Number of Fisher Scoring iterations: 2
\end{verbatim}
\end{kframe}
\end{knitrout}
	
\end{frame}





\section{3. Malaria control with bednets}

\begin{frame}[fragile,plain]
		\vspace*{-1.091in}
\tiny
See the 2018 Lancet article \textit{Efficacy of Olyset Duo, a bednet containing pyriproxyfen and permethrin, versus a permethrin-only net against clinical malaria in an area with highly pyrethroid-resistant vectors in rural Burkina Faso: a cluster-randomised controlled trial} by Tiono et. al. Reproduce the Rate ratio (95\% CI) in Table 2. Calculate the rate difference and 95\% CI comparing PPF-treated to Standard long-lasting insecticidal nets. Check the goodness of fit. 




\begin{knitrout}\tiny
\definecolor{shadecolor}{rgb}{0.969, 0.969, 0.969}\color{fgcolor}\begin{kframe}
\begin{verbatim}
## Call:
## glm(formula = cases ~ exposure + offset(log(years)), family = poisson(link = log), 
##     data = df)
## 
## Coefficients:
##             Estimate Std. Error z value Pr(>|z|)
## (Intercept)   0.6831     0.0243   28.09  < 2e-16
## exposure     -0.2669     0.0329   -8.12  4.6e-16
## 
## (Dispersion parameter for poisson family taken to be 1)
## 
##     Null deviance: 1381.2  on 23  degrees of freedom
## Residual deviance: 1316.0  on 22  degrees of freedom
## AIC: 1477
## 
## Number of Fisher Scoring iterations: 5
\end{verbatim}
\end{kframe}
\end{knitrout}

\end{frame}

%\section{Bootstrap confidence intervals}

\section{4. Population mortality rates in Denmark}

\begin{frame}[fragile,plain]
\tiny
\vspace*{-0.1in}
We can fit the following simple (multiplicative) rate ratio model to the patterns of mortality rates  for 1980-1984 and  2000-2004. The reference cell is females 70-74, 1980-84. $R$ = rate. $M$ = multiplier.

\begin{tabular}{|l c | l l  l  l  | l l l l | l |}
	\hline
	Year  & Age & \multicolumn{3}{c}{Female (F)} & &   \multicolumn{3}{c}{Male (M)} & \\ 
	\hline
	& 70-74 &  $R_{F}$ & & & & $R_{F}$ & & $\times M_{M}$  & \\
	1980- & 75-79 &  $R_{F}$ & $ \times M_{75}$ & &   & $R_{F}$ & $\times M_{75}$ & $\times M_{M}$ & \\
	1984 & 80-84 & $R_{F}$ & $ \times M_{80}$ &  & &  $R_{F}$ & $ \times M_{80}$ & $ \times M_{M}$ & \\
	& 85-89 & $R_{F}$ & $ \times M_{85}$ &  & &  $R_{F}$ & $ \times M_{85}$ & $ \times M_{M}$ & \\ 
	\hline
	& 70-74 &  $R_{F}$ &  & & $\times M_{20y}$  &  $R_{F}$ & & $  \times M_{M}$  & $\times M_{20y}$\\
	2000- & 75-79 &  $R_{F} $ & $\times M_{75}$ & & $\times M_{20y}$ &  $R_{F}$ & $ \times M_{75}$ & $ \times M_{M}$& $\times M_{20y}$ \\
	2004      & 80-84 & $R_{F}$ & $ \times M_{80}$ & & $\times M_{20y}$ &   $R_{F}$ & $ \times M_{80}$ & $ \times M_{M}$ & $\times M_{20y}$ \\
	& 85-89 & $R_{F}$ & $ \times M_{85}$ & \ \ \ & $\times M_{20y}$&   $R_{F}$ & $\times M_{85}$ & $\times M_{M}$ & $\times M_{20y}$ \\
	\hline
\end{tabular}

%The array called `r' in the R code ( which fits additive models to the rates and logs of the rates)can be used to calculate ratios.

\begin{knitrout}\tiny
\definecolor{shadecolor}{rgb}{0.969, 0.969, 0.969}\color{fgcolor}
\begin{tabular}{l|l|r|r|r|r|r|r}
\hline
Year & Age & Female\_deaths & Female\_PT & Female\_rate & Male\_deaths & Male\_PT & Male\_rate\\
\hline
1980-1984 & 70-74 & 15989 & 586883 & 0.02724 & 23810 & 456908 & 0.05211\\
\hline
1980-1984 & 75-79 & 20838 & 454143 & 0.04588 & 24707 & 300319 & 0.08227\\
\hline
1980-1984 & 80-84 & 24073 & 297679 & 0.08087 & 20319 & 167304 & 0.12145\\
\hline
1980-1984 & 85-89 & 20216 & 147772 & 0.13681 & 13524 & 74296 & 0.18203\\
\hline
2000-2004 & 70-74 & 13912 & 521562 & 0.02667 & 17360 & 436995 & 0.03973\\
\hline
2000-2004 & 75-79 & 19731 & 471946 & 0.04181 & 22477 & 341363 & 0.06584\\
\hline
2000-2004 & 80-84 & 25541 & 369990 & 0.06903 & 22992 & 217930 & 0.10550\\
\hline
2000-2004 & 85-89 & 27135 & 226798 & 0.11964 & 17444 & 104010 & 0.16772\\
\hline
2005-2009 & 70-74 & 12179 & 540569 & 0.02253 & 15782 & 472013 & 0.03344\\
\hline
2005-2009 & 75-79 & 17273 & 444474 & 0.03886 & 19547 & 344351 & 0.05676\\
\hline
2005-2009 & 80-84 & 23513 & 363534 & 0.06468 & 21781 & 230530 & 0.09448\\
\hline
2005-2009 & 85-89 & 26842 & 237877 & 0.11284 & 17811 & 114485 & 0.15557\\
\hline
\end{tabular}


\end{knitrout}

%The equation is for the rate in any given age-group in a given gender in a given calendar period:
\textcolor{white}{text}\newline

\begin{tabular}{c c c c c c c c c}
	Rate = & $\rule{1cm}{0.15mm}$ & $\times \rule{1cm}{0.15mm}$ & $\times \rule{1cm}{0.15mm}$ & $\times \rule{1cm}{0.15mm}$ & $\times \rule{1cm}{0.15mm}$ & $\times \rule{1cm}{0.15mm}$ \\
	& &   if  &  if &  if & if & if & \\
	& &  75-79 & 80-84 & 85-89 & male & 2000-04 \\  \\
	$\log[Rate] =$ & $\rule{1cm}{0.15mm}$ & $+ \rule{1cm}{0.15mm}$ & $+ \rule{1cm}{0.15mm}$ & $+ \rule{1cm}{0.15mm}$ & $+ \rule{1cm}{0.15mm}$ & $+ \rule{1cm}{0.15mm}$ \\
	& &   if  &  if &  if & if & if & \\
	& &  75-79 & 80-84 & 85-89 & male & 2000-04 \\ \\
	
	$\log[Rate] =$ &$\rule{1cm}{0.15mm}$& $+  \rule{1cm}{0.15mm}$ & $+   \rule{1cm}{0.15mm}$ & $+   \rule{1cm}{0.15mm}$ & $+   \rule{1cm}{0.15mm} $ & $+ \rule{1cm}{0.15mm}$ \\
	& &  $\times$  &  $\times$ &  $\times$ & $\times$ & $\times$ & \\
	& &  $I_{75-79}$ & $I_{80-84}$ & $I_{85-89}$ & $I_{male}$ & $I_{2000-04}$ \\
\end{tabular}

where each $`I'$ is a (0/1) indicator of the category in question. By using both the 0 and 1 values of each $I$, this 6-parameter equation  produces a fitted value for each of the $4\times2\times2=16$ cells.

%You can also think of $I_{75-79},$  $I_{80-84},$ and  $I_{85-89}$ as 
%`radio buttons':  at most 1 of them can be `on' at the same time, since there are 4 
%age levels in all.
\end{frame}



\end{document}
