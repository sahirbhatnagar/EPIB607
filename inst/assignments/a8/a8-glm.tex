\documentclass[letterpaper,12pt,twoside,]{pinp}

%% Some pieces required from the pandoc template
\providecommand{\tightlist}{%
  \setlength{\itemsep}{0pt}\setlength{\parskip}{0pt}}

% Use the lineno option to display guide line numbers if required.
% Note that the use of elements such as single-column equations
% may affect the guide line number alignment.

\usepackage[T1]{fontenc}
\usepackage[utf8]{inputenc}

% pinp change: the geometry package layout settings need to be set here, not in pinp.cls
\geometry{layoutsize={0.95588\paperwidth,0.98864\paperheight},%
  layouthoffset=0.02206\paperwidth, layoutvoffset=0.00568\paperheight}

\definecolor{pinpblue}{HTML}{185FAF}  % imagecolorpicker on blue for new R logo
\definecolor{pnasbluetext}{RGB}{101,0,0} %



\title{Assignment 8 - Poisson and Logistic Regression. Due December 1,
11:59pm 2021}

\author[a]{EPIB607 - Inferential Statistics}

  \affil[a]{McGill University}

\setcounter{secnumdepth}{5}

% Please give the surname of the lead author for the running footer
\leadauthor{Bhatnagar}

% Keywords are not mandatory, but authors are strongly encouraged to provide them. If provided, please include two to five keywords, separated by the pipe symbol, e.g:
 \keywords{  Rates |  Parameter contrasts |  Regression  }  

\begin{abstract}
In this assignment you will practice poisson and logistic regression.
State all assumptions. Provide confidence intervals with appropriate
units. Answers should be given in full sentences (DO NOT just provide
the number). All graphs and calculations are to be completed in an R
Markdown document using the template from previous assignments. You are
free to choose any function from any package to complete the assignment.
Concise answers will be rewarded. Be brief and to the point. Please
submit the compiled pdf report to Crowdmark. You need to save your
answers to each question in separate pdf files. You also need to upload
your code separately to Q3. See \url{https://crowdmark.com/help/} for
details.
\end{abstract}

\dates{This version was compiled on \today} 

% initially we use doi so keep for backwards compatibility
% new name is doi_footer


\begin{document}

% Optional adjustment to line up main text (after abstract) of first page with line numbers, when using both lineno and twocolumn options.
% You should only change this length when you've finalised the article contents.
\verticaladjustment{-2pt}

\maketitle
\thispagestyle{firststyle}
\ifthenelse{\boolean{shortarticle}}{\ifthenelse{\boolean{singlecolumn}}{\abscontentformatted}{\abscontent}}{}

% If your first paragraph (i.e. with the \dropcap) contains a list environment (quote, quotation, theorem, definition, enumerate, itemize...), the line after the list may have some extra indentation. If this is the case, add \parshape=0 to the end of the list environment.


\hypertarget{points-population-mortality-rates-in-denmark}{%
\section{(50 points) Population mortality rates in
Denmark}\label{points-population-mortality-rates-in-denmark}}

The following table contains mortality data for males and females in
Denmark for 4 age groups over three time periods.

\vspace*{0.3in}

\begin{tabular}{l|l|r|r|r|r|r|r}
\hline
Year & Age & Female\_deaths & Female\_PT & Female\_rate & Male\_deaths & Male\_PT & Male\_rate\\
\hline
1980-1984 & 70-74 & 15989 & 586882.8 & 0.0272439 & 23810 & 456908.21 & 0.0521111\\
\hline
1980-1984 & 75-79 & 20838 & 454142.7 & 0.0458843 & 24707 & 300318.92 & 0.0822692\\
\hline
1980-1984 & 80-84 & 24073 & 297678.6 & 0.0808691 & 20319 & 167303.51 & 0.1214499\\
\hline
1980-1984 & 85-89 & 20216 & 147771.7 & 0.1368057 & 13524 & 74295.83 & 0.1820291\\
\hline
2000-2004 & 70-74 & 13912 & 521561.9 & 0.0266737 & 17360 & 436994.92 & 0.0397259\\
\hline
2000-2004 & 75-79 & 19731 & 471945.5 & 0.0418078 & 22477 & 341362.82 & 0.0658449\\
\hline
2000-2004 & 80-84 & 25541 & 369989.9 & 0.0690316 & 22992 & 217929.72 & 0.1055019\\
\hline
2000-2004 & 85-89 & 27135 & 226798.1 & 0.1196439 & 17444 & 104009.58 & 0.1677153\\
\hline
2005-2009 & 70-74 & 12179 & 540568.6 & 0.0225300 & 15782 & 472012.84 & 0.0334355\\
\hline
2005-2009 & 75-79 & 17273 & 444474.2 & 0.0388616 & 19547 & 344351.34 & 0.0567647\\
\hline
2005-2009 & 80-84 & 23513 & 363534.1 & 0.0646789 & 21781 & 230530.24 & 0.0944822\\
\hline
2005-2009 & 85-89 & 26842 & 237877.3 & 0.1128397 & 17811 & 114485.04 & 0.1555749\\
\hline
\end{tabular}

\begin{enumerate}
\def\labelenumi{\alph{enumi}.}
\tightlist
\item
  (15 points) Come up with a suitable regression model for this data.
  Write down the regression equation in terms of parameters and
  determinants.
\item
  (15 points) Estimate the parameters of this model using the data in
  the table above. Provide the fitted regression equation. The data is
  provided in the \texttt{denmark.csv} file in myCourses.
\item
  (10 points) Interpret the parameter for gender. Are mortality rates
  significantly different in males compared with females?
\item
  (10 points) Perform a goodness of fit test for the fitted model in
  part (b). Is this a good fit?
\end{enumerate}

\newpage

\hypertarget{points-survival-of-patients-following-admission-to-an-adult-intensive-care-unit-icu}{%
\section{(50 points) Survival of patients following admission to an
adult intensive care unit
(ICU)}\label{points-survival-of-patients-following-admission-to-an-adult-intensive-care-unit-icu}}

The ICU study data set consists of a sample of 200 subjects who were
part of a much larger study on survival of patients following admission
to an adult intensive care unit (ICU). The major goal of this study was
to develop a logistic regression model to predict the probability of
survival to hospital discharge of these patients. A code sheet for the
variables in this data is provided in \texttt{icu\_codebase.pdf}. The
primary outcome variable is vital status at hospital discharge
(\texttt{sta}). Clinicians associated with the study felt that a key
determinant of survival was the type of admission, \texttt{type}. The
dataset can be loaded into \texttt{R} as follows:

\begin{Shaded}
\begin{Highlighting}[]
\FunctionTok{load}\NormalTok{(}\StringTok{"icu.rda"}\NormalTok{)}
\end{Highlighting}
\end{Shaded}

\begin{enumerate}
\def\labelenumi{\alph{enumi}.}
\tightlist
\item
  Write down the equation for the logistic regression model of
  \texttt{sta} on \texttt{type}. What characteristic of the outcome
  variable, \texttt{sta}, leads us to consider the logistic regression
  model as opposed to the usual linear regression model to describe the
  relationship between \texttt{sta} and \texttt{type}? What is the
  parameter of interest and what does it represent?\\
\item
  Use the \texttt{plot} function (in base R) to plot the relationship
  between \texttt{sta} and \texttt{type}. Interpret the plot.\\
\item
  Using a logistic regression routine of your choice, obtain the
  estimates of the parameters of the logistic regression model in part
  a. Interpret the estimate for the parameter of interest in the context
  of the problem and provide a 95\% confidence interval for this
  parameter. State your assumptions.\\
\item
  Using these estimates, write down the equation for the fitted values,
  that is, the estimated probabilities of the response.\\
\item
  Plot the fitted values as a function of \texttt{sta} using a boxplot.
  Interpret the boxplot.\\
\item
  Fit a model with \texttt{type} as the response and \texttt{sta} as the
  determinant. Comment on the differences and similarities of this model
  vs.~the one in part c.~Would you prefer one model over the other?
  Explain.
\end{enumerate}

%\showmatmethods


\bibliography{pinp}
\bibliographystyle{jss}



\end{document}
