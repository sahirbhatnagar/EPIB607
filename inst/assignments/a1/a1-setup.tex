\documentclass[letterpaper,12pt,twoside,]{pinp}

%% Some pieces required from the pandoc template
\providecommand{\tightlist}{%
  \setlength{\itemsep}{0pt}\setlength{\parskip}{0pt}}

% Use the lineno option to display guide line numbers if required.
% Note that the use of elements such as single-column equations
% may affect the guide line number alignment.

\usepackage[T1]{fontenc}
\usepackage[utf8]{inputenc}

% pinp change: the geometry package layout settings need to be set here, not in pinp.cls
\geometry{layoutsize={0.95588\paperwidth,0.98864\paperheight},%
  layouthoffset=0.02206\paperwidth, layoutvoffset=0.00568\paperheight}

\definecolor{pinpblue}{HTML}{185FAF}  % imagecolorpicker on blue for new R logo
\definecolor{pnasbluetext}{RGB}{101,0,0} %



\title{Assignment 1 - Setting up the computing environment and
Introduction to R. Due September 6, 2021.}

\author[a]{EPIB607 - Inferential Statistics}

  \affil[a]{Fall 2021, McGill University}

\setcounter{secnumdepth}{5}

% Please give the surname of the lead author for the running footer
\leadauthor{A1 Due September 6, 2021}

% Keywords are not mandatory, but authors are strongly encouraged to provide them. If provided, please include two to five keywords, separated by the pipe symbol, e.g:
 \keywords{  R |  RStudio |  R Markdown |  DataCamp  }  

\begin{abstract}

\end{abstract}

\dates{This version was compiled on \today} 

% initially we use doi so keep for backwards compatibility
% new name is doi_footer

\pinpfootercontents{Assignment 1 due Sepetember 6, 2021 by 11:59pm}

\begin{document}

% Optional adjustment to line up main text (after abstract) of first page with line numbers, when using both lineno and twocolumn options.
% You should only change this length when you've finalised the article contents.
\verticaladjustment{-2pt}

\maketitle
\thispagestyle{firststyle}
\ifthenelse{\boolean{shortarticle}}{\ifthenelse{\boolean{singlecolumn}}{\abscontentformatted}{\abscontent}}{}

% If your first paragraph (i.e. with the \dropcap) contains a list environment (quote, quotation, theorem, definition, enumerate, itemize...), the line after the list may have some extra indentation. If this is the case, add \parshape=0 to the end of the list environment.


\tableofcontents

\vspace*{1in}

Computing is an essential part of modern statistics. However, before
doing any data analysis, we must first install the necessary tools. In
this assignment you will first download R and RStudio and learn about
what these tools are. Then you will signup for a free DataCamp account
with your \texttt{mail.mcgill.ca} or \texttt{mcgill.ca} address. After
this, you will be asked to complete a series of DataCamp courses that
will guide you through some basic and some more advanced commands. You
will then be introduced to R Markdown which is a tool for creating
reproducible reports. All future assignments for this course must be
submitted in R Markdown format.

\newpage

\hypertarget{marking}{%
\section{Marking}\label{marking}}

Your progress and completion of these courses will be availble to us
automatically through the DataCamp website. \textbf{You do not need to
hand anything in for this Assignment}. You will receive full credits for
this assignment once we have seen that all tasks have been completed.

\hypertarget{install-r-and-rstudio}{%
\section{Install R and RStudio}\label{install-r-and-rstudio}}

See this \href{https://sahirbhatnagar.com/EPIB607/install.html}{Chapter}
from the course notes, which will guide you through installing both
\href{https://cran.r-project.org/}{R} and
\href{https://www.rstudio.com/products/rstudio/download/preview/}{RStudio}.
RStudio is a software application that facilitates how you interact with
\texttt{R}.

\hypertarget{basics-of-r-and-rstudio}{%
\section{Basics of R and RStudio}\label{basics-of-r-and-rstudio}}

Read the following two Chapters from the course notes:\\
1. \href{https://sahirbhatnagar.com/EPIB607/basics.html}{Basics of R and
RStudio}\\
2. \href{https://sahirbhatnagar.com/EPIB607/projects.html}{RStudio
Projects}

\hypertarget{sign-up-for-datacamp}{%
\section{Sign up for DataCamp}\label{sign-up-for-datacamp}}

Sign up for a free DataCamp account at
\href{https://www.datacamp.com/groups/shared_links/3dae8b789a824d0761a650b316c1d10a25cda8934ccf214aa7cd8366d07ed301}{this
link}. Note: you are required to sign up with a \texttt{@mail.mcgill.ca}
or \texttt{@mcgill.ca} email address.

\hypertarget{required-courses-to-complete}{%
\section{Required Courses to
Complete}\label{required-courses-to-complete}}

Once you have registered for DataCamp, you will see a series of courses
that have been assigned. These are required to complete by September 6,
2021.

\begin{enumerate}
\def\labelenumi{\arabic{enumi}.}
\tightlist
\item
  Introduction to \texttt{R}
\item
  Reporting with \texttt{R\ Markdown}
\end{enumerate}

\hypertarget{interactive-r-learning-environment-with-swirl-optional}{%
\section{Interactive R Learning Environment with Swirl
(Optional)}\label{interactive-r-learning-environment-with-swirl-optional}}

\texttt{swirl} is a software package for the R programming language that
turns the R console into an interactive learning environment. Users
receive immediate feedback as they are guided through self-paced lessons
in data science and R programming. Students from previous years found
this to be extremely helpful in their path to learning \texttt{R}.
Follow \href{https://swirlstats.com/students.html}{instructions on this
page} for how to get started.

%\showmatmethods





\end{document}
