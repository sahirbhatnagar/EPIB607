\documentclass[letterpaper,12pt,twoside,]{pinp}

%% Some pieces required from the pandoc template
\providecommand{\tightlist}{%
  \setlength{\itemsep}{0pt}\setlength{\parskip}{0pt}}

% Use the lineno option to display guide line numbers if required.
% Note that the use of elements such as single-column equations
% may affect the guide line number alignment.

\usepackage[T1]{fontenc}
\usepackage[utf8]{inputenc}

% pinp change: the geometry package layout settings need to be set here, not in pinp.cls
\geometry{layoutsize={0.95588\paperwidth,0.98864\paperheight},%
  layouthoffset=0.02206\paperwidth, layoutvoffset=0.00568\paperheight}

\definecolor{pinpblue}{HTML}{185FAF}  % imagecolorpicker on blue for new R logo
\definecolor{pnasbluetext}{RGB}{101,0,0} %


\usepackage{subfig}

\title{Assignment 4 - Central Limit Theorem, Confidence Intervals and
Bootstrap. Due October 15, 11:59pm 2021}

\author[a]{EPIB607 - Inferential Statistics}

  \affil[a]{Fall 2021, McGill University}

\setcounter{secnumdepth}{5}

% Please give the surname of the lead author for the running footer
\leadauthor{EPIB607}

% Keywords are not mandatory, but authors are strongly encouraged to provide them. If provided, please include two to five keywords, separated by the pipe symbol, e.g:
 

\begin{abstract}
All questions are to be answered in an R Markdown document using the
provided template and compiled to a pdf document. You are free to choose
any function from any package to complete the assignment. Concise
answers will be rewarded. Be brief and to the point. Each question is
worth 25 points. Label your graphs appropriately with proper titles and
axis labels. Justify your answers. You may compile your reoport to pdf
or to HTML. If you compile to HTML, then you must print the resulting
HTML to pdf. Please submit the compiled pdf report to Crowdmark. You
must also submit your code to Crowdmark. If you use the template, the
code from your assignment will automatically appear at the end. Upload
this code to Q5 in Crowdmark. You can upload a single pdf to Crowdmark,
and then select the pages for a given question. See
\url{https://crowdmark.com/help/} for details.
\end{abstract}

\dates{This version was compiled on \today} 

% initially we use doi so keep for backwards compatibility
% new name is doi_footer

\pinpfootercontents{Assignment 4}

\begin{document}

% Optional adjustment to line up main text (after abstract) of first page with line numbers, when using both lineno and twocolumn options.
% You should only change this length when you've finalised the article contents.
\verticaladjustment{-2pt}

\maketitle
\thispagestyle{firststyle}
\ifthenelse{\boolean{shortarticle}}{\ifthenelse{\boolean{singlecolumn}}{\abscontentformatted}{\abscontent}}{}

% If your first paragraph (i.e. with the \dropcap) contains a list environment (quote, quotation, theorem, definition, enumerate, itemize...), the line after the list may have some extra indentation. If this is the case, add \parshape=0 to the end of the list environment.


\hypertarget{template}{%
\section*{Template}\label{template}}
\addcontentsline{toc}{section}{Template}

Use the template from the previous assignment.

\hypertarget{points-concordance-between-pcr-based-extraction-free-saliva-and-nasopharyngeal-swabs-for-sars-cov-2-testing---part-i}{%
\section{(25 points) Concordance between PCR-based extraction-free
saliva and nasopharyngeal swabs for SARS-CoV-2 testing - PART
I}\label{points-concordance-between-pcr-based-extraction-free-saliva-and-nasopharyngeal-swabs-for-sars-cov-2-testing---part-i}}

This question is based on the article
\href{https://hrbopenresearch.org/articles/4-85/v1}{\emph{Concordance
between PCR-based extraction-free saliva and nasopharyngeal swabs for
SARS-CoV-2 testing}}. The data used to reproduce the results is provided
with the article and it provides Ct values for both test types
(Nasopharyngeal and Saliva). Download the data, and use the following
code to read it into \texttt{R}. Note that a Ct value of
\texttt{undetected} implies that no virus was found in the sample. In
the following \texttt{R} code, I specify \texttt{undetected} to be NA:

\newpage

\begin{Shaded}
\begin{Highlighting}[]
\FunctionTok{library}\NormalTok{(readxl)}
\FunctionTok{library}\NormalTok{(dplyr)}
\FunctionTok{library}\NormalTok{(here)}

\CommentTok{\# read symptomatic cohort data}
\NormalTok{dt\_symp }\OtherTok{\textless{}{-}}\NormalTok{ readxl}\SpecialCharTok{::}\FunctionTok{read\_xlsx}\NormalTok{(}
\NormalTok{  here}\SpecialCharTok{::}\FunctionTok{here}\NormalTok{(}\StringTok{"Ct\_values\_for\_matched\_NPS\_and\_saliva\_samples\_(symptomatic\_cohort).xlsx"}\NormalTok{), }
  \AttributeTok{na =} \StringTok{"undetected"}\NormalTok{, }
  \AttributeTok{col\_names =} \FunctionTok{c}\NormalTok{(}\StringTok{"ID"}\NormalTok{,}\StringTok{"Nasopharyngeal"}\NormalTok{,}\StringTok{"Saliva"}\NormalTok{), }
  \AttributeTok{skip =} \DecValTok{1}\NormalTok{, }
  \AttributeTok{col\_types =} \FunctionTok{c}\NormalTok{(}\StringTok{"text"}\NormalTok{, }\StringTok{"numeric"}\NormalTok{,}\StringTok{"numeric"}\NormalTok{)}
\NormalTok{) }\SpecialCharTok{\%\textgreater{}\%} 
\NormalTok{  dplyr}\SpecialCharTok{::}\FunctionTok{mutate}\NormalTok{(}\AttributeTok{cohort =} \StringTok{"Symptomatic"}\NormalTok{)}

\CommentTok{\# read asymptomatic cohort data}
\NormalTok{dt\_asymp }\OtherTok{\textless{}{-}}\NormalTok{ readxl}\SpecialCharTok{::}\FunctionTok{read\_xlsx}\NormalTok{(}
\NormalTok{  here}\SpecialCharTok{::}\FunctionTok{here}\NormalTok{(}\StringTok{"Ct\_values\_for\_matched\_NPS\_and\_saliva\_samples\_(asymptomatic\_cohort).xlsx"}\NormalTok{), }
  \AttributeTok{na =} \StringTok{"undetected"}\NormalTok{, }
  \AttributeTok{col\_names =} \FunctionTok{c}\NormalTok{(}\StringTok{"ID"}\NormalTok{,}\StringTok{"Nasopharyngeal"}\NormalTok{,}\StringTok{"Saliva"}\NormalTok{), }
  \AttributeTok{skip =} \DecValTok{1}\NormalTok{,}
  \AttributeTok{col\_types =} \FunctionTok{c}\NormalTok{(}\StringTok{"text"}\NormalTok{, }\StringTok{"numeric"}\NormalTok{,}\StringTok{"numeric"}\NormalTok{)}
\NormalTok{) }\SpecialCharTok{\%\textgreater{}\%} 
\NormalTok{  dplyr}\SpecialCharTok{::}\FunctionTok{mutate}\NormalTok{(}\AttributeTok{cohort =} \StringTok{"Asymptomatic"}\NormalTok{)}

\CommentTok{\# combine symptomatic and asymptomatic data together}
\NormalTok{dt }\OtherTok{\textless{}{-}}\NormalTok{ dplyr}\SpecialCharTok{::}\FunctionTok{bind\_rows}\NormalTok{(dt\_symp, dt\_asymp) }\SpecialCharTok{\%\textgreater{}\%} 
\NormalTok{  dplyr}\SpecialCharTok{::}\FunctionTok{mutate}\NormalTok{(}\AttributeTok{cohort =} \FunctionTok{factor}\NormalTok{(cohort))}
\end{Highlighting}
\end{Shaded}

\begin{enumerate}
\def\labelenumi{\alph{enumi})}
\tightlist
\item
  (5 points) Reproduce Table 2. Hint: you can use the following command
  to create a new variable which indicates if the individual was
  SARS-CoV-2 negative or positive. To answer this question, you can
  simply show your code and it's output.
\end{enumerate}

\begin{Shaded}
\begin{Highlighting}[]
\FunctionTok{ifelse}\NormalTok{(}\FunctionTok{is.na}\NormalTok{(dt}\SpecialCharTok{$}\NormalTok{Nasopharyngeal),}\StringTok{"negative"}\NormalTok{,}\StringTok{"positive"}\NormalTok{)}
\end{Highlighting}
\end{Shaded}

\begin{enumerate}
\def\labelenumi{\alph{enumi})}
\setcounter{enumi}{1}
\tightlist
\item
  (5 points) Reproduce the point estimates for Positive agreement and
  Negative agreement in Table 2. To answer this question, you can simply
  show your code and it's output.\\
\item
  (5 points) Can you determine which statistical procedure was used to
  calculate the confidence intervals for the point estimates in part b)?
  If yes, state the assumptions of the statistical procedure. If no,
  compute confidence intervals using a procedure of your choice and
  compare them with the ones given in the paper. State your
  assumptions.\\
\item
  (5 points) Consider the 3 figures shown in Figure 1. Comment on the
  dataset format that would have been used to create the three figures
  (e.g.~tidy format, wide format).\\
\item
  (5 points) Create a graphic which would support the claim that
  \emph{The difference in mean Ct values (0.132) was not statistically
  significant (p=0.860)}. You can either recreate a graph shown in the
  paper, or create your own. Justify why your figure supports the claim.
  Note: you \textbf{are not} being asked to perform any statistical
  test.
\end{enumerate}

\hypertarget{points-concordance-between-pcr-based-extraction-free-saliva-and-nasopharyngeal-swabs-for-sars-cov-2-testing---part-ii}{%
\section{(25 points) Concordance between PCR-based extraction-free
saliva and nasopharyngeal swabs for SARS-CoV-2 testing - PART
II}\label{points-concordance-between-pcr-based-extraction-free-saliva-and-nasopharyngeal-swabs-for-sars-cov-2-testing---part-ii}}

This question is based on the article
\href{https://hrbopenresearch.org/articles/4-85/v1}{\emph{Concordance
between PCR-based extraction-free saliva and nasopharyngeal swabs for
SARS-CoV-2 testing}}. The code to read in the data is shown in the
previous question. For the following questions, only use the symptomatic
cohort.

\begin{enumerate}
\def\labelenumi{\alph{enumi}.}
\tightlist
\item
  (5 points) Give a 95\% confidence interval for the mean Ct value for
  each test type (Nasopharyngeal and Saliva). State your assumptions.\\
\item
  (5 points) Based on these confidence intervals, are you convinced that
  the data show there is no difference in mean Ct values between both
  test types? Explain.\\
\item
  (5 points) Construct a 95\% confidence interval for the mean
  difference in Ct values between both test types using a canned
  function. Interpret this confidence interval. State the assumptions
  being made by this function.\\
\item
  (5 points) Construct a 95\% bootstrap confidence interval for the mean
  difference in Ct values between both test types. Compare it to the one
  obtained in part c.~State your assumptions.\\
\item
  (5 points) Compute a standard error for the mean difference in Ct
  values between both test types. Was the standard error that you
  computed used in any of the confidence intervals you constructed in
  parts a,c or d? Explain.
\end{enumerate}

\newpage

\hypertarget{points-concordance-between-pcr-based-extraction-free-saliva-and-nasopharyngeal-swabs-for-sars-cov-2-testing---part-iii}{%
\section{(25 points) Concordance between PCR-based extraction-free
saliva and nasopharyngeal swabs for SARS-CoV-2 testing - PART
III}\label{points-concordance-between-pcr-based-extraction-free-saliva-and-nasopharyngeal-swabs-for-sars-cov-2-testing---part-iii}}

This question is based on the article
\href{https://hrbopenresearch.org/articles/4-85/v1}{\emph{Concordance
between PCR-based extraction-free saliva and nasopharyngeal swabs for
SARS-CoV-2 testing}}. The code to read in the data is shown in the
previous question.

\begin{enumerate}
\def\labelenumi{\alph{enumi}.}
\tightlist
\item
  (5 points) For the asymptomatic cohort, calculate the sensitivity and
  specificity of the saliva test. Show your work.\\
\item
  (7.5 points) The reviewer reports are provided with the article.
  Specifically, Reviewer 2 has an issue with the very wide confidence
  interval for sensitivity of the saliva test. Without performing any
  calculations, do you agree or disagree with their statement.
  Explain.\\
\item
  (7.5 points) Are you able to calculate a 95\% bootstrap confidence
  interval for the sensitivity of the saliva test? If yes, compute it
  and compare it to the one given in the paper. If not, explain why.\\
\item
  (5 points) Do you think the 95\% confidence interval provided for the
  specificity is correct? Explain.
\end{enumerate}

\hypertarget{points-how-deep-is-the-ocean}{%
\section{(25 points) How deep is the
ocean?}\label{points-how-deep-is-the-ocean}}

This question is based on the in-class exercise on sampling
distributions using the depths of the ocean example.

\begin{enumerate}
\def\labelenumi{\alph{enumi}.}
\tightlist
\item
  (5 points) For your sample of \(n=5\) of depths of the ocean,
  calculate the 95\% Confidence interval using the \(\pm\) formula, the
  \texttt{qnorm} function, and using \(B=10000\) bootstrap samples.\\
\item
  (5 points) Plot all three confidence intervals on the same plot and
  comment on the difference/similarities between the 3 intervals. You
  may use the \texttt{compare\_CI} function provided below to produce
  the plot. This takes as input, the sample mean (\texttt{ybar}), and
  the CIs calculated from a,b,c in the form of a numeric vector of
  length 2 into the arguments \texttt{PM}, \texttt{QNORM} and
  \texttt{BOOT}, respectively.\\
\item
  (10 points) Repeat parts a and b using your sample of size \(n=20\).
  Comment on the difference/similarities between the \(n=5\) and
  \(n=20\) graph.\\
\item
  (5 points) In the in-class exercise, there were a total of
  approximately \(N=60\) students who participated. In your own words,
  briefly explain the difference between the \(N=60\), and the \(n=5\)
  or \(n=20\). What effect did these different `n' have on the sampling
  distribution of the sample means?
\end{enumerate}

\newpage

\begin{Shaded}
\begin{Highlighting}[]
\NormalTok{compare\_CI }\OtherTok{\textless{}{-}} \ControlFlowTok{function}\NormalTok{(ybar, PM, QNORM, BOOT, }
                       \AttributeTok{col =} \FunctionTok{c}\NormalTok{(}\StringTok{"\#E41A1C"}\NormalTok{,}\StringTok{"\#377EB8"}\NormalTok{,}\StringTok{"\#4DAF4A"}\NormalTok{)) \{}

\NormalTok{  dt }\OtherTok{\textless{}{-}} \FunctionTok{data.frame}\NormalTok{(}\AttributeTok{type =} \FunctionTok{c}\NormalTok{(}\StringTok{"plus\_minus"}\NormalTok{, }\StringTok{"qnorm"}\NormalTok{, }\StringTok{"bootstrap"}\NormalTok{),}
                   \AttributeTok{ybar =} \FunctionTok{rep}\NormalTok{(ybar, }\DecValTok{3}\NormalTok{),}
                   \AttributeTok{low =} \FunctionTok{c}\NormalTok{(PM[}\DecValTok{1}\NormalTok{], QNORM[}\DecValTok{1}\NormalTok{], BOOT[}\DecValTok{1}\NormalTok{]),}
                   \AttributeTok{up =} \FunctionTok{c}\NormalTok{(PM[}\DecValTok{2}\NormalTok{], QNORM[}\DecValTok{2}\NormalTok{], BOOT[}\DecValTok{2}\NormalTok{])}
\NormalTok{  )}
  
  \FunctionTok{plot}\NormalTok{(dt}\SpecialCharTok{$}\NormalTok{ybar, }\DecValTok{1}\SpecialCharTok{:}\FunctionTok{nrow}\NormalTok{(dt), }\AttributeTok{pch =} \DecValTok{20}\NormalTok{, }\AttributeTok{ylim =} \FunctionTok{c}\NormalTok{(}\DecValTok{0}\NormalTok{, }\DecValTok{5}\NormalTok{), }
       \AttributeTok{xlim =} \FunctionTok{range}\NormalTok{(}\FunctionTok{pretty}\NormalTok{(}\FunctionTok{c}\NormalTok{(dt}\SpecialCharTok{$}\NormalTok{low, dt}\SpecialCharTok{$}\NormalTok{up))),}
       \AttributeTok{xlab =} \StringTok{"Depth of ocean (m)"}\NormalTok{, }\AttributeTok{ylab =} \StringTok{"Confidence Interval Type"}\NormalTok{,}
       \AttributeTok{las =} \DecValTok{1}\NormalTok{, }\AttributeTok{cex.axis =} \FloatTok{0.8}\NormalTok{, }\AttributeTok{cex =} \DecValTok{3}\NormalTok{)}
  
  \FunctionTok{abline}\NormalTok{(}\AttributeTok{v =} \DecValTok{37}\NormalTok{, }\AttributeTok{lty =} \DecValTok{2}\NormalTok{, }\AttributeTok{col =} \StringTok{"black"}\NormalTok{, }\AttributeTok{lwd =} \DecValTok{2}\NormalTok{)}
  \FunctionTok{segments}\NormalTok{(}\AttributeTok{x0 =}\NormalTok{ dt}\SpecialCharTok{$}\NormalTok{low, }\AttributeTok{x1 =}\NormalTok{ dt}\SpecialCharTok{$}\NormalTok{up,}
           \AttributeTok{y0 =} \DecValTok{1}\SpecialCharTok{:}\FunctionTok{nrow}\NormalTok{(dt), }\AttributeTok{lend =} \DecValTok{1}\NormalTok{,}
           \AttributeTok{col =}\NormalTok{ col, }\AttributeTok{lwd =} \DecValTok{4}\NormalTok{)}
  
  \FunctionTok{legend}\NormalTok{(}\StringTok{"topleft"}\NormalTok{,}
         \AttributeTok{legend =} \FunctionTok{c}\NormalTok{(}\FunctionTok{eval}\NormalTok{(}\FunctionTok{substitute}\NormalTok{( }\FunctionTok{expression}\NormalTok{(}\FunctionTok{paste}\NormalTok{(mu,}\StringTok{" = "}\NormalTok{,}\DecValTok{37}\NormalTok{)))),}
                    \FunctionTok{sprintf}\NormalTok{(}\StringTok{"plus/minus CI: [\%.f, \%.f]"}\NormalTok{,PM[}\DecValTok{1}\NormalTok{], PM[}\DecValTok{2}\NormalTok{]),}
                    \FunctionTok{sprintf}\NormalTok{(}\StringTok{"qnorm CI: [\%.f, \%.f]"}\NormalTok{,QNORM[}\DecValTok{1}\NormalTok{], QNORM[}\DecValTok{2}\NormalTok{]),}
                    \FunctionTok{sprintf}\NormalTok{(}\StringTok{"bootstrap CI: [\%.f, \%.f]"}\NormalTok{,BOOT[}\DecValTok{1}\NormalTok{], BOOT[}\DecValTok{2}\NormalTok{])),}
         \AttributeTok{lty =} \FunctionTok{c}\NormalTok{(}\DecValTok{1}\NormalTok{, }\DecValTok{1}\NormalTok{,}\DecValTok{1}\NormalTok{,}\DecValTok{1}\NormalTok{),}
         \AttributeTok{col =} \FunctionTok{c}\NormalTok{(}\StringTok{"black"}\NormalTok{,col), }\AttributeTok{lwd =} \DecValTok{4}\NormalTok{)}
\NormalTok{\}}

\CommentTok{\# example of how to use the function:}
\FunctionTok{compare\_CI}\NormalTok{(}\AttributeTok{ybar =} \DecValTok{36}\NormalTok{, }\AttributeTok{PM =} \FunctionTok{c}\NormalTok{(}\DecValTok{28}\NormalTok{, }\DecValTok{40}\NormalTok{), }\AttributeTok{QNORM =} \FunctionTok{c}\NormalTok{(}\DecValTok{25}\NormalTok{,}\DecValTok{40}\NormalTok{), }\AttributeTok{BOOT =} \FunctionTok{c}\NormalTok{(}\DecValTok{31}\NormalTok{, }\DecValTok{38}\NormalTok{))}
\end{Highlighting}
\end{Shaded}


%\showmatmethods


\bibliography{pinp}
\bibliographystyle{jss}



\end{document}
