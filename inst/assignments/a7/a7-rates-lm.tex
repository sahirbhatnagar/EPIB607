\documentclass[letterpaper,12pt,twoside,]{pinp}

%% Some pieces required from the pandoc template
\providecommand{\tightlist}{%
  \setlength{\itemsep}{0pt}\setlength{\parskip}{0pt}}

% Use the lineno option to display guide line numbers if required.
% Note that the use of elements such as single-column equations
% may affect the guide line number alignment.

\usepackage[T1]{fontenc}
\usepackage[utf8]{inputenc}

% pinp change: the geometry package layout settings need to be set here, not in pinp.cls
\geometry{layoutsize={0.95588\paperwidth,0.98864\paperheight},%
  layouthoffset=0.02206\paperwidth, layoutvoffset=0.00568\paperheight}

\definecolor{pinpblue}{HTML}{185FAF}  % imagecolorpicker on blue for new R logo
\definecolor{pnasbluetext}{RGB}{101,0,0} %


\usepackage{subfig}

\title{Assignment 7 - Sample Size, Proportions, Rates, Linear
Regression. Due November 19, 11:59pm 2021}

\author[a]{EPIB607 - Inferential Statistics}

  \affil[a]{Fall 2021, McGill University}

\setcounter{secnumdepth}{5}

% Please give the surname of the lead author for the running footer
\leadauthor{EPIB607}

% Keywords are not mandatory, but authors are strongly encouraged to provide them. If provided, please include two to five keywords, separated by the pipe symbol, e.g:
 

\begin{abstract}
All questions are to be answered in an R Markdown document using the
provided template and compiled to a pdf document. You are free to choose
any function from any package to complete the assignment. Concise
answers will be rewarded. Be brief and to the point. Each question is
worth 25 points. Label your graphs appropriately with proper titles and
axis labels. Justify your answers. You may compile your reoport to pdf
or to HTML. If you compile to HTML, then you must print the resulting
HTML to pdf. Please submit the compiled pdf report to Crowdmark. You
must also submit your code to Crowdmark. If you use the template, the
code from your assignment will automatically appear at the end. Upload
this code to Q5 in Crowdmark. You can upload a single pdf to Crowdmark,
and then select the pages for a given question. See
\url{https://crowdmark.com/help/} for details.
\end{abstract}

\dates{This version was compiled on \today} 

% initially we use doi so keep for backwards compatibility
% new name is doi_footer

\pinpfootercontents{Assignment 7}

\begin{document}

% Optional adjustment to line up main text (after abstract) of first page with line numbers, when using both lineno and twocolumn options.
% You should only change this length when you've finalised the article contents.
\verticaladjustment{-2pt}

\maketitle
\thispagestyle{firststyle}
\ifthenelse{\boolean{shortarticle}}{\ifthenelse{\boolean{singlecolumn}}{\abscontentformatted}{\abscontent}}{}

% If your first paragraph (i.e. with the \dropcap) contains a list environment (quote, quotation, theorem, definition, enumerate, itemize...), the line after the list may have some extra indentation. If this is the case, add \parshape=0 to the end of the list environment.


\hypertarget{template}{%
\section*{Template}\label{template}}
\addcontentsline{toc}{section}{Template}

Use the template from the previous assignment.

\hypertarget{points-regen-cov-antibody-combination-and-outcomes-in-outpatients-with-covid-19}{%
\section{(25 points) REGEN-COV Antibody Combination and Outcomes in
Outpatients with
Covid-19}\label{points-regen-cov-antibody-combination-and-outcomes-in-outpatients-with-covid-19}}

This question is based on the paper \emph{REGEN-COV Antibody Combination
and Outcomes in Outpatients with Covid-19} by Weinreich et al.~2021
published in the New England Journal of Medicine (available on
myCourses).

\begin{enumerate}
\def\labelenumi{\alph{enumi})}
\item
  (10 points) Use a simulation based approach (as shown in class) to
  reproduce the sample size calculation for the average change from
  baseline in viral shedding from day 1 to day 22 (details can be found
  in Section 11.2 of the study protocol also available on myCourses).
  Specifically focus on the sample size of 20 patients per arm in phase
  1. Hint: you are being asked to show (via simulation) that the study
  is indeed powered at the stated value to detect the stated effect.
\item
  (10 points) The sample size section of the protocol continues further:
\end{enumerate}

\begin{quote}
Assuming a 10\% dropout rate and standard deviation of 2.1 log10
copies/mL (Cao, 2020), a sample size of 50 patients per arm in phase 2
will have at least 80\% power to detect a difference of 1.25 log10
copies/mL.
\end{quote}

Using simulations, reproduce this power calculation.

\begin{enumerate}
\def\labelenumi{\alph{enumi})}
\setcounter{enumi}{2}
\tightlist
\item
  (5 points) Finally, the protocol says:
\end{enumerate}

\begin{quote}
If a standard deviation of 3.8 log10 copies/mL is assumed (Wang, 2020c),
the detectable difference would be 2.27 log10 copies/mL
\end{quote}

Use the \texttt{pwr} R package
(\url{https://cran.r-project.org/web/packages/pwr/index.html}) to
reproduce this power calculation.

\hypertarget{points-simulation-study-for-confidence-intervals-of-proportions}{%
\section{(25 points) Simulation study for confidence intervals of
proportions}\label{points-simulation-study-for-confidence-intervals-of-proportions}}

The goal of this simulation study is to estimate the coverage
probabilities of confidence intervals for the binomial proportion. In
class we saw at least three methods to calculate the confidence interval
for a proportion: 1) normal approximation, 2) the exact method
(Clopper-Pearson) and 3) the plus 4 method.

\begin{enumerate}
\def\labelenumi{\alph{enumi})}
\item
  (10 points) Simulate 100 trials from a Binomial(n = 10, \(\pi\) = 0.1)
  distribution using the \texttt{stats::rbinom} function. For each of
  these trials, calculate the 95\% confidence interval for the
  proportion using each of the three methods mentioned above. For each
  of the three methods: plot the confidence intervals and color each of
  them by whether they covered the truth or not. Hint: see the R code
  for the slides on one sample rates.
\item
  (5 points) For each of the three methods, calculate the coverage
  probability, i.e., the percentage of intervals which contain the true
  population proportion \(\pi\). Describe your findings and comment on
  the differences between methods in terms of coverage probability.
\item
  (10 points) Repeat the simulation study in a), but this time, with
  different combinations of \(n\) and \(\pi\). Visualize the coverage
  probabilities as a function of \(n\) and \(\pi\) for each of the three
  methods. Describe your findings and comment on the differences between
  methods in terms of coverage probabilities as a function of \(n\) and
  \(\pi\). What happens to the coverage probabilities when the expected
  number of events increases? Explain.
\end{enumerate}

\hypertarget{points-concordance-between-pcr-based-extraction-free-saliva-and-nasopharyngeal-swabs-for-sars-cov-2-testing---part-i}{%
\section{(25 points) Concordance between PCR-based extraction-free
saliva and nasopharyngeal swabs for SARS-CoV-2 testing - PART
I}\label{points-concordance-between-pcr-based-extraction-free-saliva-and-nasopharyngeal-swabs-for-sars-cov-2-testing---part-i}}

This question is based on the article
\href{https://hrbopenresearch.org/articles/4-85/v2}{\emph{Concordance
between PCR-based extraction-free saliva and nasopharyngeal swabs for
SARS-CoV-2 testing}}. The data used to reproduce the results is provided
with the article and it provides Ct values for both test types
(Nasopharyngeal and Saliva). Download the data, and use the following
code to read it into \texttt{R}. Note that a Ct value of
\texttt{undetected} implies that no virus was found in the sample. In
the following \texttt{R} code, I specify \texttt{undetected} to be NA:

\begin{Shaded}
\begin{Highlighting}[]
\FunctionTok{library}\NormalTok{(readxl)}
\FunctionTok{library}\NormalTok{(dplyr)}
\FunctionTok{library}\NormalTok{(here)}

\CommentTok{\# read symptomatic cohort data}
\NormalTok{dt\_symp }\OtherTok{\textless{}{-}}\NormalTok{ readxl}\SpecialCharTok{::}\FunctionTok{read\_xlsx}\NormalTok{(}
\NormalTok{  here}\SpecialCharTok{::}\FunctionTok{here}\NormalTok{(}\StringTok{"Ct\_values\_for\_matched\_NPS\_and\_saliva\_samples\_(symptomatic\_cohort).xlsx"}\NormalTok{), }
  \AttributeTok{na =} \StringTok{"undetected"}\NormalTok{, }
  \AttributeTok{col\_names =} \FunctionTok{c}\NormalTok{(}\StringTok{"ID"}\NormalTok{,}\StringTok{"Nasopharyngeal"}\NormalTok{,}\StringTok{"Saliva"}\NormalTok{), }
  \AttributeTok{skip =} \DecValTok{1}\NormalTok{, }
  \AttributeTok{col\_types =} \FunctionTok{c}\NormalTok{(}\StringTok{"text"}\NormalTok{, }\StringTok{"numeric"}\NormalTok{,}\StringTok{"numeric"}\NormalTok{)}
\NormalTok{) }\SpecialCharTok{\%\textgreater{}\%} 
\NormalTok{  dplyr}\SpecialCharTok{::}\FunctionTok{mutate}\NormalTok{(}\AttributeTok{cohort =} \StringTok{"Symptomatic"}\NormalTok{)}

\CommentTok{\# read asymptomatic cohort data}
\NormalTok{dt\_asymp }\OtherTok{\textless{}{-}}\NormalTok{ readxl}\SpecialCharTok{::}\FunctionTok{read\_xlsx}\NormalTok{(}
\NormalTok{  here}\SpecialCharTok{::}\FunctionTok{here}\NormalTok{(}\StringTok{"Ct\_values\_for\_matched\_NPS\_and\_saliva\_samples\_(asymptomatic\_cohort).xlsx"}\NormalTok{), }
  \AttributeTok{na =} \StringTok{"undetected"}\NormalTok{, }
  \AttributeTok{col\_names =} \FunctionTok{c}\NormalTok{(}\StringTok{"ID"}\NormalTok{,}\StringTok{"Nasopharyngeal"}\NormalTok{,}\StringTok{"Saliva"}\NormalTok{), }
  \AttributeTok{skip =} \DecValTok{1}\NormalTok{,}
  \AttributeTok{col\_types =} \FunctionTok{c}\NormalTok{(}\StringTok{"text"}\NormalTok{, }\StringTok{"numeric"}\NormalTok{,}\StringTok{"numeric"}\NormalTok{)}
\NormalTok{) }\SpecialCharTok{\%\textgreater{}\%} 
\NormalTok{  dplyr}\SpecialCharTok{::}\FunctionTok{mutate}\NormalTok{(}\AttributeTok{cohort =} \StringTok{"Asymptomatic"}\NormalTok{)}

\CommentTok{\# combine symptomatic and asymptomatic data together}
\NormalTok{dt }\OtherTok{\textless{}{-}}\NormalTok{ dplyr}\SpecialCharTok{::}\FunctionTok{bind\_rows}\NormalTok{(dt\_symp, dt\_asymp) }\SpecialCharTok{\%\textgreater{}\%} 
\NormalTok{  dplyr}\SpecialCharTok{::}\FunctionTok{mutate}\NormalTok{(}\AttributeTok{cohort =} \FunctionTok{factor}\NormalTok{(cohort))}
\end{Highlighting}
\end{Shaded}

\begin{enumerate}
\def\labelenumi{\alph{enumi})}
\item
  (8 points) For the symptomatic cohort, was there a difference in mean
  Ct values? Use an appropriate regression model to answer this
  question. Write the regression equation in terms of population
  parameters and define all parameters in your model. What is the
  parameter of interest?
\item
  (5 points) Estimate the regression parameters using the data provided.
  Report the estimated coefficient for the parameter of interest and
  interpret it in the context of the problem.
\item
  (5 points) Reproduce the p-value for the parameter of interest from
  the regression output and interpret it. What assumptions were used in
  calculating the p-value. (Note: you are being asked to show how the
  p-value was calculated. Do not simply copy the value from the
  regression output.)
\item
  (7 points) Use a permutation test to calculate the p-value for the
  parameter of interest and compare it with the one obtained in part c).
  Briefly discuss this comparison.
\end{enumerate}

\newpage

\hypertarget{points-5-each-physical-activity-in-nhanes}{%
\section{(25 points, 5 each) Physical activity in
NHANES}\label{points-5-each-physical-activity-in-nhanes}}

This problem uses data from the
\href{https://cran.r-project.org/web/packages/NHANES/NHANES.pdf}{National
Health and Nutrition Examination Survey (NHANES)}, a survey conducted
annually by the US Centers for Disease Control (CDC). The dataset is
available from the \texttt{NHANES} package.

Regular physical activity is important for maintaining a healthy weight,
boosting mood, and reducing risk for diabetes, heart attack, and stroke.
In this problem, you will be exploring the relationship between weight
(\texttt{Weight}) and physical activity (\texttt{PhysActive}) using the
NHANES data. Weight is measured in kilograms. The variable
\texttt{PhysActive} is coded \texttt{Yes} if the participant does
moderate or vigorous-intensity sports, fitness, or recreational
activities, and \texttt{No} if otherwise. The objective is to compare
weight between physically active and those who are not.

\begin{enumerate}
\def\labelenumi{\alph{enumi})}
\item
  Explore the data.

  \begin{enumerate}
  \def\labelenumii{\roman{enumii}.}
  \item
    Identify how many individuals are physically active.
  \item
    Create a plot that shows the association between weight and physical
    activity. Describe what you see.
  \end{enumerate}
\item
  Provide an appropriate regression model for the stated objective and
  state the parameter of interest. Give the regression equation in terms
  of population parameters and be sure to define each of the parameters
  in your model.
\item
  Fit a linear regression model to estimate the regression parameters.
  Report the estimated coefficients from the model and interpret each of
  them in the context of the problem.
\item
  Report a 95\% confidence interval for the parameter of interest and
  interpret the interval in the context of the problem. Based on the
  interval, is there sufficient evidence at \(\alpha = 0.05\) to reject
  the null hypothesis of no association between weight and physical
  activity? State the assumptions used for calculating the 95\%
  confidence interval.
\item
  Provide a 95\% bootstrap confidence interval for the parameter of
  interest and compare it to the one in part d). Briefly discuss the
  comparison.
\end{enumerate}

%\showmatmethods


\bibliography{pinp}
\bibliographystyle{jss}



\end{document}
