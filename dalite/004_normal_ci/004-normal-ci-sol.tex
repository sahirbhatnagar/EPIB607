\documentclass[letterpaper,12pt,twoside,printwatermark=false]{pinp}

%% Some pieces required from the pandoc template
\providecommand{\tightlist}{%
  \setlength{\itemsep}{0pt}\setlength{\parskip}{0pt}}

% Use the lineno option to display guide line numbers if required.
% Note that the use of elements such as single-column equations
% may affect the guide line number alignment.

\usepackage[T1]{fontenc}
\usepackage[utf8]{inputenc}

% The geometry package layout settings need to be set here...
\geometry{layoutsize={0.95588\paperwidth,0.98864\paperheight},%
          layouthoffset=0.02206\paperwidth,%
		  layoutvoffset=0.00568\paperheight}

\definecolor{pinpblue}{HTML}{185FAF}  % imagecolorpicker on blue for new R logo
\definecolor{pnasbluetext}{RGB}{101,0,0} %



\title{DALITE Q4 - Normal Curve Calculations and Confidence Intervals.
Solutions.}

\author[a]{EPIB607 - Inferential Statistics}

  \affil[a]{Fall 2019, McGill University}

\setcounter{secnumdepth}{5}

% Please give the surname of the lead author for the running footer
\leadauthor{Bhatnagar and Hanley}

% Keywords are not mandatory, but authors are strongly encouraged to provide them. If provided, please include two to five keywords, separated by the pipe symbol, e.g:
 \keywords{  Normal calculations |  Confidence intervals |  Central Limit Theorem (CLT)  }  

\begin{abstract}
This DALITE quiz will cover the normal curve calculations and confidence
intervals.
\end{abstract}

\dates{This version was compiled on \today} 

% initially we use doi so keep for backwards compatibility
\doifooter{\url{https://sahirbhatnagar.com/EPIB607/}}
% new name is doi_footer

\pinpfootercontents{DALITE Q4 due October 1, 2019 by 5pm}

\begin{document}

% Optional adjustment to line up main text (after abstract) of first page with line numbers, when using both lineno and twocolumn options.
% You should only change this length when you've finalised the article contents.
\verticaladjustment{-2pt}

\maketitle
\thispagestyle{firststyle}
\ifthenelse{\boolean{shortarticle}}{\ifthenelse{\boolean{singlecolumn}}{\abscontentformatted}{\abscontent}}{}

% If your first paragraph (i.e. with the \dropcap) contains a list environment (quote, quotation, theorem, definition, enumerate, itemize...), the line after the list may have some extra indentation. If this is the case, add \parshape=0 to the end of the list environment.


\hypertarget{normal-calculations}{%
\section{Normal Calculations}\label{normal-calculations}}

Cholesterol levels among fourteen-year-old boys are roughly Normal with
mean 170 and standard deviation 30 milligrams per deciliter (mg/dl). In
a SRS of 4 fourteen-year-old boys, the probability that the average
cholesterol level is 200 mg/dl or more is close to (simply provide the
corresponding \texttt{R} code used to answer this question in your
rationale)

\begin{enumerate}
\def\labelenumi{\alph{enumi}.}
\tightlist
\item
  \textbf{0.023 (Correct)}
\item
  0.159
\item
  0.977
\end{enumerate}

\hypertarget{correct-rationales}{%
\subsection{Correct rationales}\label{correct-rationales}}

\begin{itemize}
\tightlist
\item
  \texttt{stats::pnorm(q=200,\ mean=170,\ sd=(30/sqrt(4)),\ lower.tail\ =\ FALSE)}
\end{itemize}

\hypertarget{incorrect-rationales}{%
\subsection{Incorrect rationales}\label{incorrect-rationales}}

\begin{itemize}
\tightlist
\item
  \texttt{stats::pnorm(q=\ 200,\ mean=170,\ sd=\ 30)}
\item
  \texttt{stats::pnorm(q=200,\ mean\ =\ 170,\ sd\ =\ 30,\ lower.tail\ =\ FALSE)}
\item
  \texttt{1-mosaic::xpnorm(q=200,\ mean\ =\ 170,\ sd=30)}
\end{itemize}

\hypertarget{normal-calculations-2}{%
\section{Normal calculations 2}\label{normal-calculations-2}}

Suppose that the distribution of heights of all male students on your
campus is Normal with mean 70 inches and standard deviation 2.8 inches.
How large a simple random sample (SRS) do you need to reduce the
standard deviation of the mean to 0.5?

\begin{enumerate}
\def\labelenumi{\alph{enumi}.}
\tightlist
\item
  31.36
\item
  \textbf{32 (Correct)}
\item
  6
\item
  12
\end{enumerate}

\hypertarget{correct-rationales-1}{%
\subsection{Correct rationales}\label{correct-rationales-1}}

\begin{itemize}
\tightlist
\item
  If we want to reduce the standard deviation of the sample mean to 0.5
  inches, then we must choose \(n\) to satisfy \(2.8/ \sqrt{n} = 0.5\).
  Solving for \(n\) gives \(n = (2.8/0.5)^2 = 31.36\). So we need 32
  people (cant have 31.36 people).
\item
  Standard deviation of the mean is equal to standard deviation of the
  population over the square root of \(n\). \(0.5 = 2.8 / \sqrt{n}\).
  Solve for \(n\). \(n = 31.36\) Round up to nearest whole person to get
  standard deviation of the mean equal to 0.5.
\item
  We cannot take \(n=31\) because then our standard deviation of the
  mean would not be at least 0.5
\end{itemize}

\hypertarget{incorrect-rationales-1}{%
\subsection{Incorrect rationales}\label{incorrect-rationales-1}}

\begin{itemize}
\tightlist
\item
  \(2.8/\sqrt{n}=0.5 \to\) \(n=(2.8/0.5)^2 \to\) \(n=31.36\)
\end{itemize}

\hypertarget{confidence-interval-1}{%
\section{Confidence Interval 1}\label{confidence-interval-1}}

A study reports the mean change in HDL of adults eating raw garlic six
days a week for six months. The margin of error for a 95\% confidence
interval is given as plus or minus 6 milligrams per deciliter of blood
(mg/dl). This means that

\begin{enumerate}
\def\labelenumi{\alph{enumi}.}
\tightlist
\item
  we can be certain that the study result is within 6 mg/dl of the truth
  about the population
\item
  we could be certain that the study result is within 6 mg/dl of the
  truth about the population if the conditions for inferences were
  satisfied
\item
  \textbf{the study used a method that gives a result within 6 mg/dl of
  the truth about the population in 95\% of all samples (Correct)}
\end{enumerate}

\hypertarget{correct-rationales-2}{%
\subsection{Correct rationales}\label{correct-rationales-2}}

\begin{itemize}
\tightlist
\item
  Confidence interval comments about the method of obtaining the result,
  not the result itself. A 95\% confidence interval implies that we can
  be certain that if this method was repeated, 95\% of the time it will
  be within 6 mg/dl of the population mean.
\item
  You cannot determine whether the true mean was within or outside the
  CI. We do not know the true mean. The confidence interval's purpose is
  to provide a certain level of confidence on the method.
\item
  The confidence interval of 95\% means that the test will give accurate
  results (within 6 milligrams per decilitre of blood) 95\% of the time
\item
  The confidence level states the probability that the method will give
  a correct result. Since the confidence level is 95\%, we can only be
  certain that 95\% of the time the method will correctly capture the
  true mean.
\end{itemize}

\hypertarget{incorrect-rationales-2}{%
\subsection{Incorrect rationales}\label{incorrect-rationales-2}}

\begin{itemize}
\tightlist
\item
  If conditions were satisfied(all statistics such as mean, sd, and n
  were calculated), the result interprets that 95\% of the data will
  fall within the CI range.
\item
  A 95\% confidence interval indicates that 95\% of observations will
  fall within the given margin
\item
  As per the definition of confidence interval
\item
  1.96 x sigma/square root of n
\item
  95\% of the population distribution is contained in the confidence
  interval
\end{itemize}

\hypertarget{confidence-intervals-2}{%
\section{Confidence Intervals 2}\label{confidence-intervals-2}}

A laboratory scale is known to have a standard deviation of \(\sigma\) =
0.001 gram in repeated weighings. Scale readings in repeated weighings
are Normally distributed, with mean equal to the true weight of the
specimen. Three weighings of a specimen on this scale give 3.412, 3.416
and 3.414 grams. Answer both questions below:

\begin{enumerate}
\def\labelenumi{\roman{enumi})}
\tightlist
\item
  A 95 \% confidence interval for the true weight is
\item
  The margin of error for a 99\% confidence interval would be
\end{enumerate}

\begin{enumerate}
\def\labelenumi{\alph{enumi}.}
\item
  \begin{enumerate}
  \def\labelenumii{\roman{enumii})}
  \tightlist
  \item
    3.414 +/- 0.00113
  \item
    smaller
  \end{enumerate}
\item
  \begin{enumerate}
  \def\labelenumii{\roman{enumii})}
  \tightlist
  \item
    3.414 +/- 0.00113
  \item
    about the same
  \end{enumerate}
\item
  \textbf{i) 3.414 +/- 0.00113 ii) larger (Correct)}
\item
  \begin{enumerate}
  \def\labelenumii{\roman{enumii})}
  \tightlist
  \item
    3.414 +/- 0.00065
  \item
    larger
  \end{enumerate}
\item
  \begin{enumerate}
  \def\labelenumii{\roman{enumii})}
  \tightlist
  \item
    3.414 +/- 0.00196
  \item
    larger
  \end{enumerate}
\end{enumerate}

\hypertarget{correct-rationales-3}{%
\subsection{Correct rationales}\label{correct-rationales-3}}

\begin{itemize}
\item
  \begin{enumerate}
  \def\labelenumi{\roman{enumi})}
  \tightlist
  \item
    3.414 \(\pm\) 1.96 \(\times\) \(0.001/\sqrt{3} = 3.414 \pm 0.00113\)
    ii) larger because 99\% corresponds to a z value of 2.58
  \end{enumerate}
\item
  The answer is C because the true weight for a 95\% confidence interval
  type is 3.414 \(\pm\) 0.00113, the margin of error for a 99\%
  confidence interval would be larger because for me to be right 99\% of
  the time the interval will have to be a lot larger than if i were to
  be right 95\% of the time
\end{itemize}

\hypertarget{incorrect-rationales-3}{%
\subsection{Incorrect rationales}\label{incorrect-rationales-3}}

\begin{itemize}
\item
  \begin{enumerate}
  \def\labelenumi{\roman{enumi})}
  \setcounter{enumi}{1}
  \tightlist
  \item
    About the same since we are still only using n=3
  \end{enumerate}
\item
  \begin{enumerate}
  \def\labelenumi{\roman{enumi})}
  \setcounter{enumi}{1}
  \tightlist
  \item
    Would be larger, as including more people
  \end{enumerate}
\end{itemize}

\hypertarget{confidence-intervals-3}{%
\section{Confidence Intervals 3}\label{confidence-intervals-3}}

You calculate a 95\% confidence interval of 27 \(\pm\) 2 centimeters
(cm) for the mean needle length of Torrey pine trees. You ask a friend
to explain this result. He believes it means that ``95\% of all Torrey
pine needles have lengths between 25 and 29 cm.'' Is he right? or wrong?
Explain your answer in the rationale.

\begin{enumerate}
\def\labelenumi{\alph{enumi}.}
\tightlist
\item
  He is right
\item
  \textbf{He is wrong (Correct)}
\end{enumerate}

\hypertarget{correct-rationales-4}{%
\subsection{Correct rationales}\label{correct-rationales-4}}

\begin{itemize}
\tightlist
\item
  We can't know that. the mean is either 27 \(\pm\) 2cm or it's not.
  What the 95\% confidence interval means is that 95\% of the time, 27
  \(\pm\) 2cm will contain the true value of the mean needle length
\item
  The 95\% confidence interval suggests that we are 95\% certain that
  the true mean of the population is between those two numbers - not
  that 95\% of the individual values will fall between these numbers.
\item
  It means that this confidence interval has a 95\% chance to capture
  the population mean needle length of Torrey pine trees. It cannot
  indicate the population distribution.
\end{itemize}

\hypertarget{incorrect-rationales-4}{%
\subsection{Incorrect rationales}\label{incorrect-rationales-4}}

\begin{itemize}
\tightlist
\item
  The parameter in this case is the mean needle length of all Torrey
  pine trees. By the definition of a CI, this has a probability of 0.95
  of being within the interval. So he is right.
\item
  Because if they repeatedly took samples we would see that 95\% of them
  would contain the population mean of 27.
\item
  95\% of samples trees will have lengths between 25 and 29 cm
\item
  He is wrong because a number that lies within one standard deviation
  of the mean which would be 25 and 29 is a 68\% confidence interval
  while numbers lying 2 standard deviations from the mean is a 95\%
  confidence interval (meaning a range of 23 to 31)
\item
  25cm-29cm would be the 68\% confidence interval
\item
  Assuming the inferences are met (normally distributed, etc.), we have
  said with a 95\% confidence interval (i.e.~using a method that is
  right 95\% of the time) that 95\% of all lengths of Torrey pine
  needles are between 25 and 29cm in length.
\end{itemize}

\hypertarget{confidence-intervals-4}{%
\section{Confidence Intervals 4}\label{confidence-intervals-4}}

A New York Times poll on women's issues -- which interviewed 1025 women
randomly selected from the United States excluding Alaska and Hawaii--
in which 47\% of the women said they do not get enough time for
themselves; the poll reported a margin of error of \(\pm\) 3 percentage
points for 95\% confidence in the conclusions.

Which of the following statements best explains what ``95\% confidence''
means.

\begin{enumerate}
\def\labelenumi{\alph{enumi}.}
\tightlist
\item
  This poll is accurate 19 times out of 20. (NO. This poll is either
  accurate or its not)
\item
  95\% chance that the info is correct for between 44 and 50\% of women.
  (NO. 95\% confidence in the procedure that produced the interval
  44-50)
\item
  In 95 of 100 comparable polls, expect 44 - 50\% of women will give the
  same answer. (NO. Same answer? as what?)
\item
  \textbf{If this same poll were repeated many times, then 95 of every
  100 such polls would give a range that included 47\%. (NO. Estimate
  will be between \(\mu - \textrm{margin}\) and
  \(\mu + \textrm{margin}\) in 95\% of applications.)}
\item
  It means that 47\% give or take 3\% is an accurate estimate of the
  population mean 19 times out of 20 such samplings. (NO. 95\% of
  applications of CI give correct answer. How can the same interval 47\%
  \(\pm\) 3 be accurate in 19 but not in the other 1?)
\end{enumerate}

\hypertarget{correct-rationales-5}{%
\subsection{Correct rationales}\label{correct-rationales-5}}

\begin{itemize}
\tightlist
\item
  the procedure that yielded this sample will be accurate (contain the
  true population mean) within 3\% in 95/100 samplings. The answer that
  best represents this is D (still not true, because 47\% should be
  replaced by true population \%).
\item
  Cannot assume that E is true because the true mean will be within 3\%,
  not necessarily 47\%, whereas in D, it doesn't specify the range that
  will include 47\% so it is more correct than E
\item
  None of the answers are really good at explaining what a 95\%
  confidence interval means - but D is the only one that comes close to
  explaining that the interval would give a range that included 47\%. I
  mostly based this on a process of elimination.
\item
  This is the most complete explanation because it invokes replicability
  of the results and confidence in the methods rather than the results
\end{itemize}

\hypertarget{incorrect-rationales-5}{%
\subsection{Incorrect rationales}\label{incorrect-rationales-5}}

\begin{itemize}
\tightlist
\item
  CI is the method used. If the poll were repeated, the sample mean
  would be within the range stated 95\% of the time.
\item
  There is 95\% probability that the calculated confidence interval will
  include the mean (47\%).
\item
  95 out of 100 times, 47\% will be in this range--this is what the
  confidence interval states.
\item
  95\% of the samples will contain the true parameter value (which in
  this case is 47\%).
\item
  Definition of confidence interval is that with 95\% confidence, our
  true mean lies within the margin of error, and 95\% of the time, our
  sample mean will also lie within that margin of error.
\item
  this is CI
\item
  the answer is D because ``95\% confidence'' means that you are 95\%
  confident that the value you have will be right
\item
  95\% confidence indicates confidence level- reflects that method will
  give these same results 95\% of the time
\item
  when you do a poll of this size 100 times, 95\% will contain the value
  47\% +/- 3\% points
\item
  While we can not be sure whether the poll is accurate at reflecting
  the true mean, we can be 95\% sure that our methodology if we repeat
  the poll many times, 95 of 100 polls would include a range that
  include 47\%.
\item
  If we were to repeat the same sample many times, we are using the same
  methodology, 95 of every 100 polls would give a mean of 47\% +/- 3\%
  which includes 47 by default. Given the margin of error of +/- 3\%,
  any number found between the minimum 44\% and maximum 50\% would also
  include 47\% in their range by default.
\item
  E explains the best because 95\% of the times the mean would be
  between 47 \(\pm\) 3
\end{itemize}

%\showmatmethods


\bibliography{pinp}
\bibliographystyle{jss}



\end{document}

